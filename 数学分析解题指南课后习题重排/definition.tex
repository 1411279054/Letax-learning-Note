\linespread{1.3}%1.3倍行距
\usepackage{tabularx}
\usepackage{enumerate}
\usepackage{tikz}
\newcommand*{\circled}[1]{\lower.7ex\hbox{\tikz\draw (0pt, 0pt)%
		circle (.5em) node {\makebox[1em][c]{\small #1}};}}
%\newcolumntype{Z}{>{\centering\arraybackslash}X}%设置 Z 居中的效果
%\setlist[enumerate]{leftmargin=0pt}%设置列表环境左边的间距
%\newcommand{\ee}{\mathrm{e}}%常见符号
%\newcommand{\dd}{\,\mathrm{d}}
%\newcommand{\ii}{\mathrm{i}}
%\newcommand{\Ln}{\mathrm{Ln\,}}
%\newcommand{\Res}{\,\mathrm{Res}}
%\newcommand{\LL}{\mathscr{L}}
%\newcommand{\EE}{\mathbb{E}\,}
%\renewcommand{\leq}{\leqslant}
%\renewcommand{\geq}{\geqslant}
%\newcommand{\Cov}{\mathrm{Cov}}
%
%%\everymath{\displaystyle}
%
%%排选择题
%\newcommand{\fourch}[4]{\\\begin{tabular}{*{4}{@{}p{4cm}}}(A)~#1 & (B)~#2 & (C)~#3 & (D)~#4\end{tabular}} % 一行
%\newcommand{\twoch}[4]{\\\begin{tabular}{*{2}{@{}p{8cm}}}(A)~#1 & (B)~#2\end{tabular}\\\begin{tabular}{*{2}{@{}p{8cm}}}(C)~#3 & (D)~#4\end{tabular}}  %两行
%\newcommand{\onech}[4]{\\(A)~#1 \\ (B)~#2 \\ (C)~#3 \\ (D)~#4}  % 四行