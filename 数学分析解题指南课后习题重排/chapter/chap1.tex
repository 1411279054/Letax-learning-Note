\chapter{分析基础}
\section{实数共理、确界、不等式}
\centering{\textbf{练习题}}
\begin{enumerate}
	\item 设$\mathrm{max}\{a+b\text{ },|a-b|\}<\tfrac{1}{2}$, 求证:\,$|\mathrm{a}|<\tfrac{1}{2},  |b|<\tfrac{1}{2}.$
	\begin{solution}
		$2|a|=|a+b+a-b|\le |a+b|+|a-b|\le 2\mathrm{max}\{a+b\text{ },|a-b|\}<1$ $\therefore|a|<\tfrac{1}{2}$\\
		$2|b|=|a+b-(a-b)|\le |a+b|+|a-b|\le 2\mathrm{max}\{a+b\text{ },|a-b|\}<1$ $\therefore|b|<\tfrac{1}{2}$                         
	\end{solution}
	\item 求证: 对$\mathbf{\forall}a,b\in \boldsymbol{R}$,有$\mathrm{max}\{ |a+b| ,|a-b| , |1-b| \} \ge \tfrac{1}{2}$.
	\begin{solution}
		$2 = |a+b-(a-b)+2(1-b)|\le |a+b|+|a-b|+2|1-b|\le 4\mathrm{max}\{|a+b|,|a-b|,|1-b|\} $ \\
		$\therefore \mathrm{max}\{|a+b|,|a-b|,|1-b|\}\ge \tfrac{1}{2} $
	\end{solution}
	\item 求证: 对$\forall a,b\in \boldsymbol{R}$,有\\
	$\mathrm{max} \{a,b\} = \tfrac{a\,+b}{2}+\tfrac{|a - b|}{2}$,\ $\mathrm{min} \{a,b\} = \tfrac{a+b}{2}-\tfrac{|a-b|}{2}$ ;\\
	并解释其几何意义.
	\begin{solution}
	易知,$\mathrm{max}\{a,b\}+\mathrm{min}\{a,b\}=a+b$ \circled{1}\quad $\mathrm{max}\{a,b\}-\mathrm{min}\{a,b\}=|a-b|$\circled{2}\\
	由\circled{1}、\circled{2}得$\mathrm{max} \{a,b\} = \tfrac{a\,+b}{2}+\tfrac{|a - b|}{2}$\quad
	$\mathrm{min} \{a,b\} = \tfrac{a+b}{2}-\tfrac{|a-b|}{2}$\\
	几何意义:$\mathrm{max}\{a,b\}$指的是$a,b$中较大的那个, $\mathrm{min}\{a,b\}$指的是$a,b$中较小的那个。
	\end{solution}
	\item 设$f\left( x \right) $在集合$X$上有界,求证:
	$$
	|f\left( x \right)-f\left( y\right)  | \le \underset{x\in X}{\mathrm{sup}}f\left( x\right)  -\underset{x\in X}{\mathrm{inf}}f\left( x \right)\quad (\forall\,x, y \in X)
	$$ 
	\begin{solution}
			$f(x)-f(y) \le \underset{x \in X}{\mathrm{sup}}f(x) - \underset{x \in X}{\mathrm{inf}}f(x)$ $\therefore |f(x)-f(y)| \le |\underset{x \in X}{\mathrm{sup}}f(x) - \underset{x \in X}{\mathrm{inf}}f(x)|=\underset{x \in X}{\mathrm{sup}}f(x) - \underset{x \in X}{\mathrm{inf}}f(x)$
	\end{solution}
	\item 设 $f(x)$,$g(x)$在集合$X$上有界, 求证:\\
	
	$$\circled{1}
	\underset{x\in X}{\mathrm{inf}}\{f(x)\}\,+\,\underset{x\in X}{\mathrm{inf}}\{g(x)\} \le
	\underset{x\in X}{\mathrm{inf}}\{f(x)+g(x)\} \le 	\underset{x\in X}{\mathrm{inf}}\{f(x)\}\,+\,\underset{x\in X}{\mathrm{sup}}\{g(x)\}
	$$ 
	$$\circled{2}
		\underset{x\in X}{\mathrm{sup}}\{f(x)\}\,+\,\underset{x\in X}{\mathrm{inf}}\{g(x)\} \le 	\underset{x\in X}{\mathrm{sup}}\{f(x)+g(x)\} \le \underset{x\in X}{\mathrm{sup}}\{f(x)\}\,+\,	\underset{x\in X}{\mathrm{sup}}\{g(x)\}
	$$
	\begin{solution}
		\textcolor{green}{\circled{1}} 易知, $\underset{x\in X}{\mathrm{sup}}\{f(x)\}\,+\,\underset{x\in X}{\mathrm{inf}}\{g(x)\}\le f(x)+g(x)\ (\forall x \in X)$, $\therefore \underset{x\in X}{\mathrm{inf}}\{f(x)\}\,+\,\underset{x\in X}{\mathrm{inf}}\{g(x)\} \le
		\underset{x\in X}{\mathrm{inf}}\{f(x)+g(x)\}$, 又$\because \underset{x\in X}{\mathrm{inf}}\{f(x)+g(x)\} \le f(x)+g(x) \le f(x)\,+\,\underset{x\in X}{\mathrm{sup}}\{g(x)\} $, 即$\underset{x\in X}{\mathrm{inf}}\{f(x)+g(x)\} - \underset{x\in X}{sup}g(x)\le f(x), (\forall x \in X)$, $\therefore \underset{x\in X}{\mathrm{inf}}\{f(x)+g(x)\} \le 	\underset{x\in X}{\mathrm{inf}}\{f(x)\}\,+\,\underset{x\in X}{\mathrm{sup}}\{g(x)\}$, 所以,
		$
		\underset{x\in X}{\mathrm{inf}}\{f(x)\}\,+\,\underset{x\in X}{\mathrm{inf}}\{g(x)\} \le
		\underset{x\in X}{\mathrm{inf}}\{f(x)+g(x)\} \le 	\underset{x\in X}{\mathrm{inf}}\{f(x)\}\,+\,\underset{x\in X}{\mathrm{sup}}\{g(x)\}
		$
		\textcolor{green}{\circled{2}} 类似上面做法.

		
	\end{solution}
\end{enumerate}
\section{函数}
\centering{\textbf{练习题}}
\begin{enumerate}
\item 设$f(x)=|1+x|-|1-x|$.
	\begin{enumerate}
		\item 求证:$f(x)$是奇函数;
		\item 求证:$|f(x)|\le 2$.
		\item 求$\underset{n\text{次}}{\underbrace{(f\circ f \circ \cdots \circ f)}}\left(	x \right) .$
	\end{enumerate}
	\begin{solution}
		\begin{enumerate}[(1)]
			\item $f(x)=f(-x)$,$\therefore f(x)$是奇函数.
			\item $f(x)=|1+x|-|1-x|\le|1+x+1-x|=2$ 
			\item 易知,$f(x)$是一个分段函数,
			$f(x)=\left\{\begin{array}{lcc}
					-2 &  & x < -1\\
					2x & &-1 \le  x\le 1\\
					2 & &x \ge 1
				
			\end{array}\right.
			$, 下面当$-1 \le  x\le 1$时,$f(x) = 2x$  $\therefore (f\circ f)(x) =\left\{\begin{array}{lcc}
			-2 &  & x < \tfrac{-1}{2}\\
			4x & & \tfrac{-1}{2} \le  x\le \tfrac{1}{2}\\
			2 & &x \ge \tfrac{1}{2}
			
		\end{array}\right. $ \quad $\therefore$ 可得,$(f\circ f \circ \cdots \circ f)(x)=\left\{\begin{array}{lcc}
		-2 &  & x < \tfrac{-1}{2^(n-1)}\\
		2^(n-1)x & & \tfrac{-1}{2^(n-1)} \le  x\le \tfrac{1}{2^(n-1)}\\
		2 & &x \ge \tfrac{1}{2^(n-1)}
		
	\end{array}\right.$
		\end{enumerate}
	\end{solution}
\item 设$f(x)$在(0,+$\infty$)上定义, $a>0,b>0$.求证:
	\begin{enumerate}
		\item 若$\frac{f(x)}{x}$单调下降, 则$f(a+b)\le f(a)+f(b)$;
		\item 若$\frac{f(x)}{x}$单调上升, 则$f(a+b)\ge f(a)+f(b)$
	\end{enumerate}
	\begin{solution}
		\begin{enumerate}[(1)]
			\item 由已知得, $\frac{f(x)}{x}$单调下降\quad$\therefore \frac{f(a+b)}{a+b} \le \tfrac{f(a)}{a},\ \frac{f(a+b)}{a+b} \le \frac{f(b)}{b}$,$\therefore af(a+b) \le (a+b)f(a),\ bf(a+b)\le (a+b)f(b)$, 可得$f(a+b)\le f(a)+f(b)$.
			\item 与第一小题类似.
		\end{enumerate}
	\end{solution}
\item 利用上题证明: 当$a>0,b>0$时,有
	\begin{enumerate}
		\item	当$p>1$时, $\left( a+b\right) ^p \ge a^p+b^p$;
		\item 当$0<p<1$时,$(a+b)^p\le a^p+b^p$.
	\end{enumerate}
	\begin{solution}
		\begin{enumerate}[(1)]
			\item 令$f(x)=x^p,\ \frac{f(x)}{x}=x^{p-1}$, $ \because p>1 , p-1>0\ \therefore x^{p-1} $单调递增,由第二题可得$f(a+b)\ge f(a)+f(b)$\ $\therefore \left( a+b\right) ^p \ge a^p+b^p$
			\item 与第一小题类似
		\end{enumerate}
	\end{solution}
\item 设$f(x)$在\textbf{R}上定义, 且$f(f(x))\equiv x$.
	\begin{enumerate}
		\item 问这种函数有几个?
		\item 若$f(x)$为单调增加函数, 问这种函数有几个?
	\end{enumerate}
	\begin{solution}
		\begin{enumerate}[(1)]
			\item 令$y = f(x),\ x = f^{-1}(y)\ \because f(f(x))\equiv x \therefore f(y) \equiv f^{-1}(y)$, 说明其原函数等于反函数,说明函数图像关于直线$y=x$对称,其这样的函数有无数多个.
			\item 一个, $f(x) \equiv x$
		\end{enumerate}
	\end{solution}
\item 求证:若$y=f(x)(x\in (-\infty, +\infty))$是奇函数, 并且它的图像关于直线$x=b(b>0)$对称, 则函数$f(x)$是周期函数并求其周期.
\begin{solution}
	$\because f(x)$是奇函数, $\therefore f(x) = -f(-x)$,又$\therefore f(x)$关于直线$x=b(b>0)$对称, $f(b+x)=f(b-x)$, 即$f(b+b+x)=f(-x)=-f(x)$, $f(x+2b)=-f(x)=-f(x+2b-2b)=f(x-2b)$, $\therefore f(x+4b)=f(x)$, 因此$f(x)$是周期函数, 其周期是$4b$.
\end{solution}
\item 设$f: X\rightarrow Y$时满射, $g: Y\rightarrow Z$.求证:$g\circ f: X\circ Z$.有反函数的充分必要条件为$f$和$g$都有反函数存在, 且$(g\circ f)^{-1}=f^{-1} \circ g^{-1}$.
\begin{solution}
	$g\circ f:X\circ Z$有反函数,说明$g\circ f$一一对应,即$f$和$g$都一一对应, 所以, $f$和$g$存在反函数, 令$(g\circ f)$的反函数为$H$, 假设$H(a)=b$,有$(g\circ f)(b) = a$,左乘$g^{-1}$, 即$f(b) = g^{-1}(a)$, 再左乘$f^{-1}$, 即$b = (f^{-1}\circ g^{-1})(x)$\ $\therefore H=f^{-1}\circ g^{-1}, (g\circ f)^{-1} = f^{-1}\circ g^{-1}$.
\end{solution}
\end{enumerate}
\section{序列极限}
\centering{\textbf{练习题}}
\begin{enumerate}
	\item 设$x_n>0, \lim\limits_{n\rightarrow \infty}x_n=a$.
	\begin{enumerate}
		\item 当$a\ne 0$时, 求证:$\lim\limits_{n \rightarrow \infty}\frac{x_{n+1}}{x_{n}}=1$
		\item 举例说明当$a=0$时,$\lim\limits_{n \rightarrow \infty}\frac{x_{n+1}}{x_{n}} \ne 1$ 可能成立;
		\item 举例说明当$a=1$时, $\lim\limits_{n \rightarrow \infty}(x_n)^n\ne 1$可能成立.
	\end{enumerate}
	\begin{solution}
		\begin{enumerate}
			\item 由已知条件知: $\lim\limits_{n\rightarrow \infty}x_n=a$.根据$\varepsilon-N$定义知, $\forall\ \varepsilon>0, \exists\, N \in N_+$, 当$n>N$时, 有$|a_n-a|<\varepsilon$\ $\because \ n>N$, 那么$n+1>N$, $\therefore \forall \varepsilon>0, \exists \,N$,有$n+1>N$\  $\therefore |a_{n+1}-a|<\varepsilon,\ \therefore \lim\limits_{n\rightarrow +\infty}x_{n+1}=a$\ \ 
			$\therefore\lim\limits_{n \rightarrow \infty}\frac{x_{n+1}}{x_{n}}=1$.
			\item 例: $a=\frac{1}{2^n}$.
			\item 例: $x_n=\frac{n+1}{n}$.
		\end{enumerate}
		
	\end{solution}
	\item 设$0<x_1<1, x_{n+1}=1-\sqrt{1-x_n}$, 求$\lim\limits_{n \rightarrow \infty}x_n$和$\lim\limits_{n\rightarrow \infty}\frac{x_{n+1}}{x_n}$.
	\begin{solution}
		令$\lim\limits_{n\rightarrow +\infty}x_n=a$,其中$a<1$, 那么$\lim\limits_{n\rightarrow +\infty}x_{n+1}=a$, 又由已知表达式得$a=1-\sqrt{1-a}$, 解得: $a=0$.又$\because\lim\limits_{n\rightarrow \infty}\tfrac{x_{n+1}}{x_n}=\frac{1-\sqrt{1-x_n}}{x_n}$,根据洛必达得$\lim\limits_{n\rightarrow \infty}\frac{x_{n+1}}{x_n}=\frac{1}{2}$.
	\end{solution} 
	\item 设$c>1$, 求序列$\sqrt{c}, \sqrt{c\sqrt{c}},\sqrt{c\sqrt{c\sqrt{c}}},\cdots$的极限.
	\begin{solution}
		根据表达式可得, $x_{n+1}=\sqrt{c\sqrt{x_n}}$, $x_{n+1}-x_{n}=\sqrt{c\sqrt{x}}-\sqrt{x}=\sqrt{\sqrt{x}}(\sqrt{c}-x_n^{\frac{3}{4}})$.假设: $x_n<c^\frac{3}{2}$.下面用归纳法来证明:\\
		当$n=1$时, $x_1=\sqrt{c}<c^\frac{3}{2}$\\
		当$n=k$时, 假设$x_n<c^\frac{3}{2}$\\
		那么$n=k+1$时,$x_{n+1}=\sqrt{c\sqrt{x_n}}<\sqrt{c\cdot c^\frac{3}{2}}=c^\frac{5}{4}<c^\frac{3}{2}$, $\therefore x_n<c^{\frac{3}{2}}$, 且$x_{n+1}-x_{n}>0$, 由单调有界定理可知数列${x_n}$存在极限。令$\lim\limits_{x_n}=a$
		$\therefore \lim\limits_{x\rightarrow +\infty}x_{n+1}=a,a=\sqrt{ca},a=c$\ $\therefore \lim\limits_{x_n}=c$
	\end{solution}
	\item 设$A>0,x_1>0,x_{n+1}=\frac{1}{2}(x_n+\frac{A}{x_n})\,(n=1,2,\cdots)$
	\begin{enumerate}
		\item 求证: $x_n$单调下降且有界;
		\item 求$\lim\limits_{n\rightarrow \infty}x_n$.
	\end{enumerate}
	\begin{solution}
		\begin{enumerate}
			\item $x_{n+1}=\frac{1}{2}(x_n+\frac{A}{x_n})\ge \sqrt{A}$, 说明$x_n$有下界.\\
			$\because x_{n+1}-x_n=\frac{1}{2}(x_n+\frac{A}{x_n})-x_n=\frac{1}{2}(\frac{A}{x_n}-x_n)=\frac{1}{2}\frac{A-x_n^2}{x_n}$, 又$\because x_n \ge \sqrt{A}$\ $\therefore x_{n+1}-x_n<0$, 所以, $x_n$单调递减且有下界.
			\item 令$\lim\limits_{n \rightarrow \infty}x_n=a$\ $\therefore a=\sqrt{A}$\ 所以$\lim\limits_{n \rightarrow \infty}x_n=\sqrt{A}$.
		\end{enumerate}
		
		
	\end{solution}
	\item 设$F_0=F_1=1,F_{n+1}=F_n+F_{n-1}$, 求证:$\lim\limits_{n\rightarrow \infty}\tfrac{F_{n-1}}{F_n}=\frac{\sqrt{5}-1}{2}$.
	\begin{solution}
		令$x_n=\frac{F_{n-1}}{F_n}(x_n>0)$, 根据表达式有$\frac{1}{x_{n+1}}=1+x_n\ \therefore \frac{x_n}{x_{n+1}}=x_n(1+x_n)>1,\text{即}x_n>x_{n+1}$, 根据单调有界定理可得${x_n}$存在极限,令$\lim\limits_{n \rightarrow \infty}x_n=a(a>0)\ \therefore \frac{1}{a}=1+a$, 解得: $a=\frac{\sqrt{5}-1}{2}$, 即$\lim\limits_{n\rightarrow \infty}\frac{F_{n+1}}{F_n}=\frac{\sqrt{5}-1}{2}$.
	\end{solution}
	\item 求证:
	\begin{enumerate}
		\item $\frac{1}{2\sqrt{n+1}}<\sqrt{n+1}-\sqrt{n}<\frac{1}{2\sqrt{n}}$;
		\item 序列$x_n=1+\frac{1}{\sqrt{2}}+\cdots+\frac{1}{n}-2\sqrt{n}$的极限存在.
	\end{enumerate}
	\begin{solution}
		\begin{enumerate}
			
		 \item 	$\sqrt{n+1}-\sqrt{n}=\frac{1}{\sqrt{n+1}+\sqrt{n}}$, 而$ \frac{1}{2\sqrt{n+1}}\le \frac{1}{\sqrt{n+1}+\sqrt{n}}\le \frac{1}{2\sqrt{n}}$\ 
			$\therefore\ \frac{1}{2\sqrt{n+1}}<\sqrt{n+1}-\sqrt{n}<\frac{1}{2\sqrt{n}}$
		\item $x_{n+1}=1+\frac{1}{\sqrt{2}}+\cdots+\frac{1}{\sqrt{n}}+\frac{1}{\sqrt{n+1}}-2\sqrt{n+1},\ x_n=1+\frac{1}{\sqrt{2}}+\cdots+\frac{1}{\sqrt{n}}-2\sqrt{n}$.\\
		$x_{n+1}-x_n=\frac{1}{\sqrt{n+1}}+2\sqrt{n}-2\sqrt{n+1}<\frac{1}{\sqrt{n+1}}-\frac{1}{\sqrt{n+1}}=0$.$\therefore \ x_{n+1}<x_n$,数列${x_n}$是递减数列.\\
		下证: $x_n>-2$(数学归纳法)\\
		$n=1$时, $x_1=-1>-2$成立\\
		假设$n=k$时, $x_n>-2$\\
		那么当$n=k+1$时,$x_{n+1}=1+\frac{1}{\sqrt{2}}+\cdots+\frac{1}{\sqrt{n+1}}-2\sqrt{n+1}=x_n+\frac{1}{\sqrt{n+1}}-2\sqrt{n+1}+2\sqrt{n}>$
		\end{enumerate}
		
		
	\end{solution}
	\item 设$0<a_1<b_1$, 令
	$$
		a_{n+1} = \sqrt{a_n\cdot b_n},\ b_{n+1}=\tfrac{a_n+b_n}{2}\ (n=1,2,\cdots)
	$$求证: 序列${a_n},{b_n}的极限存在$.
	\begin{solution}
		
	\end{solution}
	\item 求证: 如下序列的极限存在.
	$$
	\lim\limits_{n\rightarrow \infty}(1+\frac{1}{2^x}(1+\frac{1}{3^2})\cdots(1+\frac{1}{n^2})).
	$$
	\item 求证: 如下序列的极限存在:
	$$
	\lim\limits_{n\rightarrow \infty} \left[\frac{(2n)!!}{(2n-1)!!}\right]^2\frac{1}{2n+1}.
	$$
	\item 设$c>0$,求序列
	$$
	\sqrt{c}, \sqrt{c+\sqrt{c}},\sqrt{c+\sqrt{c+\sqrt{c}}}, \cdots
	$$
	的极限.
	\item 设$x_n=a_1+a_2+\cdots+a_n$,求证: 若$\tilde{x}=|a_1|+|a_2|+\cdots+|a_n|$极限存在,则${x_n}$的极限也存在.
	\item 设$x_n=a_1+a_2+\cdots+a_n,y_n=b_1+b_2+\cdots+b_n,z_n=c_1+c_2+\cdots+c_n$, 且$c_n\le a_n\le b_n(n=1,2,\cdots)$; 又设${y_n},{z_n}$极限存在.求证: ${x_n}$极限也存在.
	\item 设序列${x_n}$满足$|x_{n+1}-x_n|\le q|x_n-x_{n-1}|(n=1,2,\cdots)$, 其中$0<q<1$.求证:
	序列${x_n}$的极限存在.
	\item 设$f(x)$在$(-\infty,+\infty)$上满足条件:
	$$|f(x)-f(y)|\le q|x-y|\quad (\forall x,y \in (-\infty,+\infty))$$
	其中$0<q<1$.对$\forall x_1\in (-\infty,+\infty)$, 令$x_{n+1}=f(x_n)(n=1,2,\cdots)$.求证: 序列${x_n}$的极限存在, 且极限值是$f(x)$的不动点.
	\item 设$x_0=a,x_1=b(b>a)$, 用如下公式定义序列的项: $$
	x_{2n}=\frac{x_{2n-1}+2x_{2n-2}}{3},\ x_{2n+1}=\frac{2x_{2n}+x_{2n-1}}{3}\quad (n = 1,2,\cdots)$$
	求证: 序列${x_n}$极限存在.
\end{enumerate}
\section{函数极限与连续概念}
\centering{\textbf{练习题}}
\begin{enumerate}
	\item 设在正实轴上, $h(x)\le f(x)\le g(x)$, 且广义极限$$
	\lim\limits_{x\rightarrow \infty}h(x) = A = \lim\limits_{x\rightarrow \infty}g(x)$$存在.求证: $\lim\limits_{x\rightarrow \infty}f(x)=A$(分别讨论$A=+\infty,-\infty,有限数三种情形$).
	\item 设$\lim\limits_{x\rightarrow a}f(x)=+\infty,\lim\limits_{x\rightarrow a}g(x)=A(>0)$, 求证:$$
	\lim\limits_{x\rightarrow a}f(x)g(x) = +\infty$$
	\item 设$0<x_n<+\infty$, 且满足$x_n+\frac{4}{x^2}<3$, 求证: 极限$\lim\limits_{x_n}存在, 并求出此极限值$.
	\item 设$f(x)$是$(-\infty,+\infty)$上的周期函数, 又$$
	\lim\limits_{x\rightarrow +\infty}f(x)=0$$ 求证:$f(x)\equiv 0$.
	\item 设$f(x),g(x)$在$(a,+\infty)$上定义,$g(x)$单调上升, 且$$
	\lim\limits_{x\rightarrow +\infty}g(f(x))=+\infty.$$
	求证: $\lim\limits_{x\rightarrow +\infty}f(x)=+\infty$, $\lim\limits_{x\rightarrow +\infty}g(x)=+\infty$.
	\item 设$x_n=\tfrac{1}{1 \cdot n}+\tfrac{1}{2 \cdot (n-1)}+\cdots+\tfrac{1}{(n-1) \cdot 2}+\tfrac{1}{n \cdot 1}$, 求$\lim\limits_{n\rightarrow \infty}x_n$.
	\item 设$\lim\limits_{n\rightarrow \infty}\tfrac{a_1+a_2+\cdots+a_n}{n}=a$, 求证: $\lim\limits_{n\rightarrow \infty}\tfrac{a_n}{n}=0$
	\item 设${x_n}$满足$\lim\limits_{n\rightarrow \infty}(x_n-x_{n-2})=0$, 求证: $\lim\limits_{n\rightarrow \infty}\tfrac{x_n}{n}=0$.
	\item 适当定义$f(0)$, 使函数$f(x)=(1-2x)^\frac{1}{x}$在点$x=0$处连续.
	\item 设$ f(x),g(x)\in C[a,b]$,求证:
	\begin{enumerate}
		\item $|f(x)|\in C[a,b];$\
		\item $max\{f(x),g(x)\}\in C[a,b]$;
		\item $min\{f(x),g(x)\}\in C[a,b]$.
	\end{enumerate}
	\item 设$f(x)\in C[a,b]$单调上升, 且$a<f(x)<b\ (\forall x\in [a,b])$.对$\forall x_1 \in [a,b]$, 由递推公式$x_{n+1}=f(x_n)(n=1,2,\cdots)$产生序列$\{x_n\}$.求证: 极限$\lim\limits_{n\rightarrow \infty}x_n$存在, 且其极限值$c$满足$c=f(c)$.
	\item 设序列$\{x_n\}$由如下迭代产生:
	$$
	x_1 = \tfrac{1}{2}, x_{n+1} = x_{n}^2 + x_n \quad (n = 1,2,\cdots)
	$$求证:$\lim\limits_{n\rightarrow \infty}(\frac{1}{1+x_1}+\frac{1}{1+x_2}+\cdots+\frac{1}{1+x_n})=2$
	\item 求出函数$f(x)=\frac{1}{1+\frac{1}{x}}$的间断点, 并判断间断点的类型.
\end{enumerate}
\section{闭区间上连续函数的性质}
\centering{\textbf{练习题}}
\begin{enumerate}
	\item 设$f(x)\in C[a,b]$, 且$|f(x)|$在$[a,b]$上单调.求证: $f(x)$在$[a,b]$上不变号.
	\item 设$f(x)\in C(-\infty,+\infty)$, 且严格单调, 又
	$$
	\lim\limits_{x\rightarrow -\infty}f(x) = 0,\quad \lim\limits_{x\rightarrow +\infty}f(x)=+\infty$$
	求证: 方程$f^3(x)-6f^2(x)+9f(x)-3$有且仅有三个根.
	\item 设$f_n(x)=x^n+x$.求证:
	\begin{enumerate}
		\item 对任意自然数$n>1$, 方程$f_n(x)=1$在$(\frac{1}{2},1)$内有且仅有一个根;
		\item 若$c_n\in (\frac{1}{2},1)$是$f_n(x)=1\text{的根, 则}\lim\limits_{n \rightarrow \infty}c_n$存在, 并求此值.
	\end{enumerate}
	\item 设$f(x)$在$[a,b]$上无界, 求证: $\exists c\in[a,b]$, 使得对$\forall \delta>0$, 函数$f(x)$在$[c-\delta,c+\delta]\cap[a,b]$上无界.
	\item 设${x_n}$为有界序列.求证:${x_n}$以$a$为极限的充分必要条件是:\ ${x_n}$的任一收敛子序列都有相同的极限值$a$.
	\item 设$f(x),g(x)\in C[a,b]$.求证:
	$$\underset{a\leq x\le b}{\mathrm{max}}|f(x)+g(x)|\le \underset{a\le b}{\mathrm{max}}|f(x)|+\underset{a\le x \le b}{\mathrm{max}}|g(x)|.$$
	\item 设$f(x)\in C[a,b]$, 且有唯一的取到$f(x)$最大值的点$x^*$, 又设使得$\lim\limits_{n \rightarrow  \infty}f(x)=f(x^*)$.求证:$\lim\limits_{n \rightarrow \infty}x_n=x^*$.
	\item 设$f(x)\in C[0,+\infty)$, 又设对$\forall l\in \textbf{R}$, 方程$f(x)=l$在$[0,+\infty)$上只有有限个解或无解.求证:
	\begin{enumerate}
		\item 如果$f(x)$在$[0, +\infty)$上有界, 则极限$\lim\limits_{x\rightarrow +\infty}f(x)$存在;
		\item 如果$f(x)$在$[0,+\infty]$上无界, 则$\lim\limits_{n\rightarrow +\infty}=+\infty$.
	\end{enumerate}
	\item 设$f(x)\in C(-\infty,+\infty)$,存在$\lim\limits_{x\rightarrow \pm\infty}=+\infty$, 且$f(x)$的最小值$f(a)<a$.求证:$f(f(x))$至少在两个点处取到最小值.
	\item 设$f(x)$在$[a,b]$上定义, $x_0\in[a,b]$.如果对$\forall \varepsilon >0, \exists \delta >0 $, 当$|x-x_0|<\delta$时, 有$f(x)<f(x_0)+\varepsilon$, 那么称$f(x)$在点$x_0$处\textbf{上半函数}.如果$f(x)$在$[a,b]$上每一点都上半连续,则称$f(x)$为$[a,b]$上的一个半连续函数.求证:$[a,b]$上的上半连续函数一定有上界.
	\item 证明下列函数在实数轴上一致连续: \\
	(1)\ $f(x)=\sqrt{1+x^2}$\qquad (2)\ $f(x)=sinx$
	\item 证明下列函数在实数轴上不一致连续: \\
	(1)\ $f(x)=xsinx$;\qquad(2)\ $f(x)=sinx^2$.
	\item 设$f(x)$在$[0,+\infty)$上一致连续, 对$\forall h\ge 0,\lim\limits_{n\rightarrow \infty}f(h+n)=A$(有限数).求证:$\lim\limits_{n\rightarrow +\infty}f(x)=A$.
	\item 设存在常数$L>0$, 使得$f(x)$在$[a,b]$上满足$$
	|f(x)-f(y)|\le L|x-y|,\quad \forall x,y\in [a,b].
	$$求证: $f(x)$在$[a,b]$上一致连续.
	\item 设函数$f(x),g(x)$在$(a,b)$内一致连续.求证: $f(x)+g(x)$与$f(x)\cdot g(x)$都在$(a,b)$内一致连续.
	\item 设$f(x)$在$(a,b)$内一致连续, 值域含于区间$(a,d)$, 又$g(x)$在$(c,d)$内一致连续.求证: $g(f(x))$在$(a,b)$内一致连续.
	\item 设$f(x)\in C(-\infty,+\infty)$, 且是周期为$T$的周期函数.求证: $f(x)$在实轴上一致连续.
	
\end{enumerate}