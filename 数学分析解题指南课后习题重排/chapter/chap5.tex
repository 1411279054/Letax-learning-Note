\chapter{多元函数积分学}
\section{欧式空间、多元函数的极限与连续}
\centering{\textbf{练习题}}
\begin{enumerate}
	\item $\forall \bm{x},\bm{y}\in R^m$.证明: $|\bm{x}+\bm{y}|^2+|\bm{x}-\bm{y}|=2(|\bm{x}|^2+|\bm{y}|^2)$, 并说明等式的几何意义.
	\item 证明下列三个命题等价:
	\begin{enumerate}
		\item $\bm{x\cdot y}=0$;
		\item $|\bm{x-y}|^2=|\bm{x}|^2+|\bm{y}|^2$;
		\item $|\bm{x}-\bm{y}|=|\bm{x}+\bm{y}|$.
	\end{enumerate}
	\item 设$\bm{z}\in \bm{R}^m$为常向量, $c$为常数, 证明:
	\begin{enumerate}
	\item 	$H=\{\bm{x}|\bm{x}\in \bm{R}^m, \bm{x}\cdot \bm{z}<c\}$是开集;
	\item $\{\bm{x}|\bm{x}\in \bm{R}^m, \bm{x}\cdot \bm{z}\ge c\}$是闭集.
	\end{enumerate}
	\item 试画出下列集合$\Omega$的图形:
	\begin{enumerate}
		\item $\{(x,y)|y>0,x>y,x<1\}$;
		\item $\{(x,y)|0\le y\le 2 ,2y\le x\le 2y+2 \}$
		\item $\{(x,y)|1\le xy \le 2,\frac{1}{2}\le \frac{y}{x}\le 1\}$;
		\item $\{(x,y,z)|0<x<y<z<1\}$.
	\end{enumerate} 
	\item 证明:
	\begin{table}[H]
		\begin{tabular}{ll}
			$(1)\ (A\cup B)^\circ \supset A^\circ\cup B^\circ$;\qquad \qquad \qquad \qquad \qquad&(2)\ $\overline{A\cap B}\subset \overline{A}+\overline{B}$。
		\end{tabular}
	\end{table}
 	\item 设$A,B$为$\bm{R}^m$中的有界集. 证明:
 	\begin{table}[H]
 		\begin{tabular}{ll}
 			(1)\ $\partial(A\cup B)\subset \partial A \cup \partial B$;\qquad \qquad \qquad \qquad \qquad &(2)\ $\partial(A\cap B)\subset \partial A \cup \partial B$. 
 		\end{tabular}
 	\end{table}
 	\item 设$A,B$是$\bm{R}^m$中不相交的闭集, 求证: 存在开集$W$和$V$, 满足$A\subset W, B\subset V$, 而 $W\cap V=\varnothing$.
 	\item 设$E\subset \bm{R}^m$, 证明: $E=\{\bm{x}|\rho(\bm{x},E)=0\}$.
 	\item 设$F_1,F_2,\cdots,F_n,\cdots$是$\bm{R}^m$中的有界闭集列, 满足$F_n \supset F_{n+1}(n=1,2,\cdots)$, 又$F_n$的直径$d_n=d(F_n)\rightarrow 0\ (n\rightarrow \infty)$. 求证: 存在惟一的一点$
 	\bm{x_0}\in \bm{R}^m$, 使得$\bm{x_0}\in \bigcap\limits_{n=1}^{\infty}F_n$.
 	\item 设$E,F$为$\bm{R}^m$中的闭集, $E,F$中至少有一为有界集, 求证: $\exists \bm{x}\in E,\bm{y}\in F$, 使得$\rho(x,y)=\rho(E,F)$.
 	\item 设$D$为$\bm{R}^m$中的凸集, 证明: $\overline{D}$也是凸集.
 	\item 证明:
 	\begin{enumerate}
 		\item $|\bm{x}-2(\bm{x}\cdot \bm{a})\frac{\bm{a}{|\bm{a}|^2}}| = |\bm{x}|\ (\bm{a}\ne 0)$;
		 \item $|\frac{\bm{x}}{|\bm{x}|^2}|-\frac{\bm{y}}{|\bm{y}|^2}| = \frac{|\bm{x}-\bm{y}|}{|\bm{x}||\bm{y}|}$;
		 \item 设$|\bm{x}||\bm{y}-\bm{x}/|\bm{x}|^2|=|\bm{y}||\bm{x}-\bm{y}/|\bm{y}|^2|$.
 	\end{enumerate}
 	\item 确定并画出下列函数的定义域, 指出后两题的等位面是什么曲面(或曲线):
 	\begin{table}[H]
 	\begin{tabular}{ll}
 		(1)\ $u=\sqrt{1-x^2}+\sqrt{1-y^2}$;\qquad \qquad \qquad \qquad \qquad &(2)\ $u=\sqrt{\frac{2x-x^2-y^2}{x^2+y^2-x}}$;\\
 		(3)\ $u=\mathrm{arcsin}\frac{y}{x}$;\qquad  \qquad \qquad \qquad \qquad &(4)\  $u=\mathrm{ln}(-1-x^2-y^2+z^2)$.
 	\end{tabular}	
 	 \end{table}
  \item 求下列函数的极限:
  \begin{table}[H]
  	\begin{tabular}{ll}
  		(1)\ $\lim\limits_{(x,y)\rightarrow (0,0)}\frac{\mathrm{e}^x+\mathrm{e}^y}{\mathrm{cos}x+\mathrm{sin}y}$;\qquad \qquad \qquad \qquad \qquad &(2)\ $\lim\limits_{(x,y)\rightarrow (0,0)\frac{x^2y^{3/2}}{x^4+y^2}}$;\\
  		(3)\ $\lim\limits_{x\rightarrow +\infty \atop y\rightarrow +\infty}(x^2+y^2)\mathrm{e}^{-(x+y)}$;\qquad \qquad \qquad \qquad \qquad &(4)\ $\lim\limits_{(x,y)\rightarrow (0,0)}\frac{\mathrm{sin}(x^3+y^3)}{x^2+y^2}$.
  	\end{tabular}
  \end{table}

\item 对下列函数$f(x,y)$, 证明$\lim\limits_{(x,y)\rightarrow (0,0)}f(x,y)$不存在:
\begin{table}[H]
	\begin{tabular}{ll}
		(1)\ $f(x,y)=\frac{x^2}{x^2+y^2}$;\qquad \qquad \qquad \qquad \qquad &(2)\ $f(x,y)=\frac{x^2y^2}{x^3+y^3}$.
	\end{tabular}
\end{table}
\item 问下列函数是否在全平面连续, 为什么?
\begin{enumerate}
	\item $f(x,y)=\begin{cases}
	\frac{x^2-y^2}{x^2+y^2},\qquad&x^2+y^2 \ne 0,\\
	0,\qquad& x^2+y^2 = 0;
	\end{cases}$
	\item $f(x,y)=\begin{cases}
	\frac{\mathrm{sin}(xy)}{x},\qquad &x\ne 0,\\
	y,\qquad &x=0;
	\end{cases}$
	\item $f(x,y)=\begin{cases}
	\frac{x^2}{y^2}\mathrm{e}^-\frac{x^4}{y^2},\qquad& y\ne 0,\\
	0,\qquad& y=0;
	\end{cases}$
	\item $f(x,y)=\begin{cases}
	y^2\mathrm{ln}(x^2+y^2),\qquad& x^2+y^2\ne 0,\\
	0,\qquad& x^2+y^2=0
	\end{cases}$
\end{enumerate}
\item 设函数$f(x,y)$在半开平面$x>0$上连续, 且对$\forall y_0$, 极限
$$\lim\limits_{x\rightarrow 0^+ \atop y\rightarrow y_0}f(x,y) = \varphi(y_0)$$
存在. 当函数$f$在$y$轴上补充定义$\varphi(y)$后, 证明: 函数$f(x,y)$在闭半平面$x\ge 0$上连续.
\item 设函数$f(x,y)$在开半平面$x>0$上一致连续. 证明:
\begin{enumerate}
	\item $\forall y_0$, 极限$\lim\limits_{x \rightarrow 0^+ \atop y \rightarrow y_0}=\varphi(y_0)$存在;
	\item 函数在$y$轴上补充定义$\varphi (y)$后, 所得函数$f(x,y)$在$x\ge 0$上一致连续. 
\end{enumerate}
\item 设$u=f(\bm{x})$在$\bm{x_0}\in \bm{R}^m$点连续, 且$f(\bm{x_0})>0$.证明: 存在$\bm{x_0}$的一个领域$U(\bm{x_0};\delta)$, 使得$f(\bm{x})$在$U(\bm{x_0};\delta)$上取正值.
\item 设$E$是$\bm{R}^m$中任意点集, 求证: $\rho(\bm{x},E)$在$\bm{R}^m$上一致连续.
\item 设$f(\bm{x})\in C(\bm{R}^m,\bm{R})$, 对任意实数$\alpha$, 作集合
$$ G={\bm{x}|f(\bm{x}>\alpha)},\quad F={\bm{x}|f(\bm{x})\ge\alpha}.$$
求证: $G$是$\bm{R}^m$中的开集, $F$是$\bm{R}^m$中的闭集.
\item 设$\bm{x}\in \bm{R}^m, \bm{x}=(x_1,x_2,\cdots,x_m)$. 求证L
\begin{enumerate}
	\item $\exists a>0, b>0$, 使得$a|\bm{x}|\le \sum\limits_{i=1}^{m}|x_i|\le b|\bm{x}|$;
		\item $\exists a>0, b>0$, 使得$a|\bm{x}|\le \underset{1\le i \le m}{\mathrm{max}}|x_i|\le b|\bm{x}|$.
\end{enumerate}
\item 设$A$是$m\times m$矩阵, $\mathrm{det}A\ne 0$, 求证: $\exists a>0$, 使得
$$ |A\bm{x}|\ge a|\bm{x}|\ (\forall \bm{x}\in \bm{R}^m).$$
\item 设$\overline{\Omega}\subset\bm{R}^m$是有界闭区域, $f(\bm{x})\in C(\overline{\Omega},\bm{R}^m)$, 且是单叶的. 求证:$f^{-1}(x)$在$f(\overline{\Omega})$上连续.
\item 设$f(x,y)$除直线$x=a$与$y=b$外有定义, 且满足:
\begin{enumerate}
	\item $\lim\limits_{y\rightarrow b}f(x,y)=\varphi (x)$存在;
	\item $\lim\limits_{x\rightarrow a}f(x,y)=\phi (y)$一致存在(即$\forall \epsilon >0,\exists \delta (\epsilon)>0,\text{当}0<|x-a|<\delta, \text{时}, \forall y\ne b,\text{有}|f(x,y)-\phi(y)|<\epsilon$).
	证明:
	\begin{enumerate}
		\item 累次极限$\lim\limits_{x\rightarrow a}\lim\limits_{y\rightarrow b}f(x,y)=\lim\limits_{x\rightarrow a}\phi(x)=c$存在。
		\item 累次极限$\lim\limits_{y\rightarrow b}\lim\limits_{x\rightarrow a}f(x,y)=\lim\limits_{y\rightarrow b}\phi(y)=c$
		\item 全面极限$\lim\limits_{(x,y)\rightarrow (a,b)}f(x,y)=c$.
	\end{enumerate}
	
\end{enumerate}
\end{enumerate}

\section{偏导数与微分}
\centering{\textbf{练习题}}
\begin{enumerate}
	\item 求下列函数的偏导数:
	\begin{table}[H]
		\begin{tabular}{ll}
			(1)\ $u=\frac{x}{\sqrt{x^2+y^2}}$;\qquad \qquad \qquad \qquad \qquad &(2)\ $u=\mathrm{tan}\frac{x^2}{y}$;\\
			(3)\ $u=\mathrm{sin}(x\mathrm{cos}y)$;\qquad \qquad \qquad \qquad \qquad &(4)\ 
			$u=\mathrm{e}^\frac{x}{y}$;\\
			(5)\ $u=\mathrm{ln}\sqrt{x^2+y^2}$;\qquad \qquad \qquad \qquad \qquad &(6)\ $
			u=\mathrm{arctan}\frac{x+y}{1-xy}$;\\
			(7)\ $u=(\frac{x}{y})^z$;\qquad \qquad \qquad \qquad \qquad &(8)\ $u=\mathrm{arccos}\frac{z}{\sqrt{x^2+y^2}}$ 
		\end{tabular}
	\end{table}
\item 设置$f(x,y)$在园$\Omega$上的偏导数$f'_x, f'_y$存在且有界. 证明:$f(x,y)$在$\Omega$上一致连续. 若$\Omega$是任意区间, 问区间是否成立. 考察例子
$$ f(x,y) = \mathrm{arctan}(\frac{y}{x}),$$
$\Omega$用极坐标表示为$1<r<2,0<\theta<2\pi$.
\item 设
$$ f(x,y)=\begin{cases}
\frac{x^2y}{x^2+y^2},\quad & x^2+y^2 \ne 0,\\
0, x^2+y^2 = 0. 
\end{cases}
$$
证明:
\begin{enumerate}
	\item $f(x,y)$在$(0,0)$点连续;
	\item $f'_x(0,0),f'_y(0,0)$存在;
	\item $f'_x(x,y)$, $f'_y(x,y)$在$(0,0)$点不连续;
	\item $f(x,y)$在$(0,0)$点不可微.
\end{enumerate}
\item 设$$f(x,y)=\begin{cases}
\frac{\mathrm{sin}(xy)}{x},\quad &x\ne0,\\
y,\quad &x=0,
\end{cases}$$
证明: $f(x,y)$在平面上可微.
\item 求下列复合函数的偏导数:
\begin{table}[H]
	\begin{tabular}{ll}
		(1)\ $u=f(\frac{xz}{y})$;\qquad \qquad \qquad \qquad &(2)\ $u=f(x+y,z)$;\\
		(3)\ $u=f(x,xy,xyz)$;\qquad \qquad \qquad \qquad &(4)\ $u=f(x+y+z,x^2+y^2+z^2)$;\\
		(5)\ $u=f(\frac{x}{y},\frac{y}{z})$;\qquad \qquad \qquad \qquad & (6)\ $u=f(x^2+y^2,x^2-y^2,2xy)$.
	\end{tabular}
\end{table}
\item 设$u=x^nf(\frac{y}{x},\frac{z}{x})$, 其中$f$可微. 证明$u$满足方程:
$$ x\frac{\partial f}{\partial x}+y\frac{\partial u}{\partial y} +z \frac{\partial u}{\partial z}=n\cdot u.$$
\item 证明: $f(x,y,z)$为$n$次齐次函数的充要条件是
$$ x\frac{\partial f}{\partial x}+y\frac{\partial f}{\partial y}+z\frac{\partial f}{\partial z}= nf(x,y,z).$$
\item 作自变量变换: $x=\sqrt{vw},y=\sqrt{wu},z=\sqrt{uv}$, 它把函数$f(x,y,z)$变为$F(u,v,w)$.证明:
$$xf'_x+yf'_y+af'_z=uF'_u+vF'_v+wF'_w.$$
\item 令$\xi =2xy,\eta =x^2-y^2$, 解下列方程(解可含任意函数):
\begin{table}[H]
	\begin{tabular}{ll}
		(1)\ $y\frac{\partial u}{\partial x}+x\frac{\partial u}{\partial y}=0$; \qquad \qquad \qquad \qquad \qquad &(2)\ $x\frac{\partial u}{\partial x}-y\frac{\partial u}{\partial y}=0$.
	\end{tabular}
\end{table}
\item 令$\xi = x,\eta = y-x,\zeta =z-x$, 求方程$\frac{\partial u}{\partial x}+\frac{\partial u}{\partial y}+\frac{\partial u}{\partial z}=0$的解.
\item 设$u(x,y),v(x,y)$为连续可微函数, 且满足方程组
$$ \frac{\partial u}{\partial x}=\frac{\partial v}{\partial y},\ \frac{\partial u}{\partial y}=-\frac{\partial v}{\partial x}.$$
作自变量变换: $x=r\mathrm{cos}\theta, y=r\mathrm{sin}\theta$, 证方程组变成为
$$ \frac{\partial u}{\partial r}=\frac{1}{r}\frac{\partial v}{\partial \theta}, \ \frac{1}{r}\frac{\partial u}{\partial \theta}=-\frac{\partial v}{\partial r}.$$
\item 再对上题所得方程组做变换:$R=\sqrt{u^2+v^2}, \varPhi=\mathrm{arctan}\frac{v}{u}$.证明方程组变为
$$\frac{\partial \mathrm{ln}R}{\partial r}=\frac{1}{r}\frac{\partial \varPhi}{\theta},\ \frac{1}{r}=\frac{\partial \mathrm{ln}R}{\partial \theta}=-\frac{\partial \varPhi}{\partial r}$$.
\item 设$f(x,y)=x^2-xy+y^2, (x_0,y_0)=(1,1)$.
\begin{enumerate}
	\item 若方向$\bm{l}$与基$\bm{e_1},\bm{e_2}$的夹角为$\frac{\pi}{3}$和$\partial{\pi}{6}$, 求方向导数$\frac{\partial f(1,1)}{\partial \bm{l}}$;
	\item 求在怎样的方向上方向导数$\frac{\partial f(1,1)}{\partial \bm{l}}$有最大值、 最小值、等于零.
\end{enumerate}
\item 设$u=f(x,y,z)$, 令
$$ x=r\mathrm{sin}\varphi cos\theta,\ y=r\mathrm{sin}\varphi \mathrm{sin}\theta,\ z= r\mathrm{cos}\varphi.$$
在$(x,y,z)$点作三个互相正交的向量$\bm{e}_r,\bm{e}_\varphi,\bm{e}_\theta$. 向量$\bm{e}_r$表示$\varphi, \theta$固定沿着$r$增加的方向, 其余两个作类似理解. 证明:
$$\frac{\partial u}{\partial \bm{e}_r}=\frac{\partial u}{\partial r}, \frac{\partial u}{\partial \bm{e}_\varphi}, \frac{\partial u}{\partial \bm{e}_\theta}=\frac{1}{r\mathrm{sin}\varphi}\frac{\partial u}{\partial \theta}.$$
\item 设在第一卦限上连续可微函数$u(x,y),v(x,y)$满足方程组
$$\frac{\partial u}{\partial x} = \frac{\partial v}{\partial y}, \ \frac{\partial u}{\partial y}=-\frac{\partial u}{\partial x},$$
且$u$只是$\sqrt{x^2+y^2}$的函数, 试求出$u(x,y)$和$v(x,y)$.
\item 设$f(\bm{x})$定义在凸函数$\Omega \subset \bm{R}^m$上, 对$\forall \bm{x}_1,\bm{x}_2\in \Omega , t\in [0,1]$, 满足$$
f[t\bm{x}_1+(1-t)\bm{x}_2]\le tf(\bm{x}_1) + (1-t)f(\bm{x}_2),$$
则称$f$为凹函数. 若$f(\bm{x})$是凸域$\Omega$上的可微凹函数, 证明:
\begin{enumerate}
	\item $f(\bm{x})-f(\bm{x_0})\ge \frac{f[\bm{x_0+t(\bm{x-x_0})}]-f(\bm{x_0})}{t}, \ \bm{x},\bm{x_0}\in \Omega$;
	\item $f(\bm{x})\ge f(\bm{x_0})+\mathrm{D}f(\bm{x_0}(\bm{x-x_0}))$.
\end{enumerate}
\item 求下列函数的二阶偏导数:
\begin{table}[H]
	\begin{tabular}{ll}
		(1)\ $u=xy+\frac{y}{x}$;\qquad \qquad \qquad \qquad \qquad &(2)\ $u=(xy)^z$.
	\end{tabular}
\end{table}
\item 对下列函数求指定阶的偏导数:
\begin{enumerate}
	\item $u=x^4+y^4-2x^2y^2$, 求所有三阶偏导数;
	\item $u=x^3\mathrm{sin}y+y^3\mathrm{sin}x$, 求$\frac{\partial^6 u}{\partial x^3 \partial y^3}$;
	\item $u=\mathrm{e}^{xyz}$, 求$\frac{\partial^3u}{\partial x \partial y \partial z}$;
	\item $u=\mathrm{ln}\sqrt{x^2+y^2}$, 求$\frac{\partial^4u}{\partial x^2\partial y^2}$.
\end{enumerate}
\item 求高阶导数:
\begin{table}[H]
	\begin{tabular}{ll}
		(1)\ $u=(x-a)^p(y-b)^q$, 求$\frac{\partial^{p+q}u}{\partial x^p\partial y^q}$;\qquad \qquad \qquad \qquad &(2)\ $u=\frac{x+y}{x-y}(x\ne y)$, 求$\frac{\partial^{m+n}u}{\partial x^m\partial y^n}$;\\
		(3)\ $u=\mathrm{ln}(ax+by)$, 求$\frac{\partial^{m+n}u}{\partial x^m\partial y^n}$;\qquad \qquad \qquad \qquad &(4)\ $u=xyz\mathrm{e}^{x+y+z}$, 求$\frac{\partial^{p+q+r}u}{\partial x^p\partial y^q\partial z^r}$.
	\end{tabular}
\end{table}
\item 求下列函数的二阶偏导数:
\begin{table}[H]
	\begin{tabular}{ll}
		(1)\ $u=f(x+y,xy)$;\qquad \qquad \qquad \qquad &(2)\ $u=f(x+y+z,x^2+y^2+z^2)$;\\
		(3)\ $u=f(\frac{x}{y}, \frac{y}{z})$;\qquad \qquad \qquad \qquad & (4)\ $u=f(x^2+y^2+z^2)$.
	\end{tabular}
\end{table}
\item 验证下列函数满足调和方程
$$\nabla^2u=\frac{\partial^2u}{\partial x^2}+\frac{\partial^2u}{\partial y^2}=0.$$
\begin{table}[H]
	\begin{tabular}{ll}
	\qquad \qquad (1)\ $u=\mathrm{ln}\sqrt{x^2+y^2}$;\qquad\qquad\qquad \qquad \qquad \qquad & (2)\ $u=\mathrm{arctan}\frac{y}{x}$.
	\end{tabular}
\end{table}
\item 证明: 函数$u=\frac{1}{2a\sqrt{\pi t}}\mathrm{e}^{-\frac{(x-b)^2}{4a^2t}}$($a,b$为实数)当$t>0$时满足方程
$$
\frac{\partial u}{\partial t}=a^2\frac{\partial^2 u}{\partial x^2}.$$
\item 设$x=f(u,v),y=g(u,v)$满足方程$$
\frac{\partial f}{\partial u} = \frac{\partial g}{\partial v},\ \frac{\partial f}{\partial v}=-\frac{\partial g}{\partial u},$$
又设$w=w(x,y)$满足方程$\frac{\partial^2 w}{\partial x^2}+\frac{\partial^2 w}{\partial y^2}=0$.证明:
\begin{enumerate}
	\item 函数$\omega=\omega[f(u,v),g(u,v)]$满足方程: $\frac{\partial^2\omega}{\partial u^2}+\frac{\partial^2\omega}{\partial v^2}=0$;
	\item $\frac{\partial^2(fg)}{\partial u^2}+\frac{\partial^2(fg)}{\partial v^2}=0$.
\end{enumerate}
\item 作变量替换$\xi = x+t,\eta=x-t$, 求解方程$\frac{\partial^2 u}{\partial t^2}=\frac{\partial^2 u}{\partial t^2}=\frac{\partial^2 u}{\partial x^2}$, 并验证之.
\item 求下列函数在$(0,0)$点领域展开为带皮亚诺余项的四阶泰勒公式:
\begin{table}[H]
	\begin{tabular}{ll}
		(1)\ $u=\mathrm{sin}(x^2+y^2)$;\qquad \qquad \qquad \qquad &(2)\ $u=\sqrt{1+x^2+y^2}$;\\
		(3)\ $u=\mathrm{ln}(1+x)\mathrm{ln}(1+y)$;\qquad \qquad \qquad \qquad & (4)\ $u=\mathrm{e}^{\mathrm{cos}y}$. 
	\end{tabular}
\end{table}
\item 设函数$f(x,y)$满足$\frac{\partial^2 f}{\partial x^2}=y$, $\frac{\partial^2 f}{\partial x \partial y}=x+y$, $\frac{\partial^2 f}{\partial y^2}=x$, 试求出函数$f(x,y)$.
\item 设$\Omega$为含原点的凸域, $u=f(x,y)$在$\Omega$上可微, 且满足
$$ x\frac{\partial f}{\partial x}+y\frac{\partial f}{\partial y} = 0$$
求证: $f(x,y)$在$\Omega$上恒为常数. 若$\Omega$不含原点, 问$f(x,y)$是否为常数. 考察例子$u=\mathrm{arctan}\frac{y}{x}$.
\item 求下列函数$f(\bm{x})\ (x\in \bm{R}^m)$的微分:
\begin{enumerate}
	\item $f(\bm{x})=(\bm{Ax-b})\cdot(\bm{Ax-b})$, 其中$A$为$n\times m$矩阵, $b\in \bm{R}^n$;
	\item $f(\bm{x})=\frac{1}{|\bm{x}|}$.
\end{enumerate}
\item 设$f:\bm{R}^m\rightarrow \bm{R}^l$, $g:\bm{R}^m\rightarrow \bm{R}^n$是可微函数. 试用复合函数求导公式, 证明公式
$$ Df(\bm{x})g(\bm{x}) = f(\bm{x})Dg(\bm{x}) + g(\bm{x})Df(\bm{x}).$$
\item 设$f(\bm{x}=\frac{\bm{x}}{|\bm{x}|})$, $\bm{x}\in \bm{R}^m$.
\begin{enumerate}
	\item 求$Df(\bm{x})$;
	\item 取方向$\bm{l}=\frac{\bm{x}}{|\bm{x}|}$, 求方向导数$\frac{\partial f}{\partial \bm{l}}$;
	\item 取方向$l$满足$\bm{l}\cdot \bm{x}=0$, 求方向导数$\frac{\partial f}{\partial \bm{l}}$;
	\item 求导数的范数$||Df(\bm{x})||$.
\end{enumerate}
\item 求下列变换的雅可比行列式:
\begin{enumerate}
		\item $x_1=r\mathrm{cos}\theta, x_2=r\mathrm{sin}\theta$, 求$\frac{\partial (x_1,x_2)}{\partial (r,\theta)}$.
		\item $x_1=r\mathrm{cos}\theta_1, x_2=r\mathrm{sin}\theta_1\mathrm{cos}\theta_2, x_3=r\mathrm{sin}\theta_1\mathrm{sin}\theta_2$, 求$\frac{\partial (x_1,x_2,x_3)}{\partial{(r,\theta_1,\theta1_2)}}$;
		\item $
		\begin{cases}
		x_1 = r\mathrm{cos}\theta_1,&\qquad \\
		x_2 = r\mathrm{sin}\theta_1\mathrm{cos}\theta_2, &\qquad \\ 
		x_3 = r\mathrm{sin}\theta_1\mathrm{sin}\theta_2\mathrm{sin}\theta_3,& r\ge 0, 0<\theta_1,\cdots,\theta_{m-2}<\pi, \\
		\qquad \vdots \qquad&0<\theta_{m-1}<2\pi \\
		x_{m-1} = r\mathrm{sin}\theta_1\mathrm{sin}\theta_2\mathrm{sin}\theta_3\cdots\mathrm{sin}\theta_{m-2}\mathrm{cos}\theta_{m-1},&\qquad \\
		x_m = r\mathrm{sin}\theta_1\mathrm{sin}\theta_2\mathrm{sin}\theta_3\cdots\mathrm{sin}\theta_{m-1},,&\qquad \\
		\end{cases}
		$\\
		试求用数学归纳法求$\frac{\partial(x_1,x_2,\cdots,x_m)}{\partial (r,\theta_1,\cdots,\theta_{m-1})}$.
\end{enumerate}
\item 设$\Omega$为$\bm{R}^m$中的凸区域, $f(\bm{x})\in C^2(\Omega,\bm{R})$.若$f$的海色矩阵$H_f(\bm{x})$是半正定的. 证明: $f(\bm{x})$是$\Omega$上的凹函数.
\end{enumerate}

\section{反函数与隐函数}

\section{切空间与极值}

\section{含参积分的定积分}

\section{含参积分的广义积分}