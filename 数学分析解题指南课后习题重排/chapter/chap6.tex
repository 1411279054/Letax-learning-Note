\chapter{多元函数积分学}
\section{重积分的概念与性质、重积分化累次积分}
\centering{\textbf{练习题}}
\begin{enumerate}
	\item 试求$\bm{R}^2$中点集$E=\{(x,y)|0\le x\le1,0\le y\le 1, x\text{和}y\text{至少有一为有理数}\}$的内容度和外容度. 问$E$是否是可测图形?
	\item 设$A,B,C$是$\bm{R}^m$中的可测图形, 证明:
	\begin{enumerate}
		\item $V(A\backslash B)=V(A)-V(A\cap B)$;
		\item $V(A\cup B)=V(A)+V(B)-V(A\cap B)$;
		\item $V(A\cup B\cup C)=V(A)+V(B)+V(C)-V(A\cap B)-V(A\cap C)-V(B\cap C)+V(A\cup B\cup C)$.
	\end{enumerate}
\item 举例说明$\bm{R}^m$中两个点集$E_1$和$E_2$都不是可测函数, 但是$E_1\cup E_2, E_1\cap E_2$都是可测函数.问是否还可能有$E_1\backslash E_2$也是可测图形.
\item 设$\Omega$为$\bm{R}^m$中一可测图形.证明: $\Omega^\circ$和$\overline{\Omega}$为可测图形, 且$V(\Omega^\circ)=V(\Omega)=V(\overline{\Omega})$.
\item 在$\bm{R}^2$的区域$D=\{(x,y)||x|\le 1,|y|\le 1\}$上给定函数
$$ f(x,y)=\begin{cases}
1,\qquad & \text{当}x,y\text{都是有理数}, \\
2,\qquad & \text{当}x,y\text{当}x,y\text{至少有一是无理数}.
\end{cases}$$
问$f(x,y)$是否在$D$上可积.
\item 设$\bm{R}^m$中的开集$\Omega$为可测图形, $f:\Omega\rightarrow \bm{R},f\in C(\Omega)$, 且$f(\bm{x})\ge 0\ (\bm{x}\in \Omega)$, 但不恒为零. 证明: $\displaystyle{\int_{\Omega}f(\bm{x})\mathrm{d}V}>0$. 如果$\Omega$不是开集, 上述论证是否正确? 举例说明.
\item 设定义在可测图形$\Omega \subset \bm{R}^m$上的两个函数$f,g$ 有界、可积, 而且$g(\bm{x})$在$\Omega$上之值非负. 令$$
m=\underset{x\in \Omega}{\mathrm{inf}\{f(\bm{x})\}},\ M = \underset{x\in \Omega}{\mathrm{sup}}\{f(\bm{x})\}.$$
证明:
\begin{enumerate}
	\item $F(t)=\displaystyle{\int_{\Omega}[f(\bm{x}-t)]g(\bm{x})\mathrm{d}V}$是$[m,M]$上的连续函数;
	\item 存在$\mu\in [m,M]$, 使得
	$$ \displaystyle{\int_{\Omega}f\cdot g\mathrm{d}V=\mu\cdot \int_{\Omega}g\mathrm{d}V}.$$
	\item 设$f(x)\in R[-1,1]$, 证明: $f(x-y)\in R([0\times1]\times[0,1])$.
	\item 设$\Omega \subset \bm{R}^m$为测度图形, $Q$为长方体, $\Omega \subset Q^\circ,f(\bm{x})\in R(\Omega)$. 定义$$
	F(\bm{x}) = \begin{cases}
	f(\bm{x}),\qquad &\bm{x}\in \Omega\\
	0,\qquad \bm{x}\in Q\backslash \Omega.
	\end{cases}$$
	求证: $F(\bm{x})\in R(Q)$.
	
\end{enumerate}
\item 设$\Omega$为$\bm{R}^m$中点集, $Q$为长方体, $\Omega \subset Q^\circ$.定义函数
$$ \chi(\bm{x})=\begin{cases}
1,\qquad &\bm{x} \in \Omega\\
0,\qquad &\bm{x} \in \Omega\backslash \Omega.
\end{cases}$$
若$\chi(\bm{x})$在$Q$上可积, 证明: $\Omega$为可测图形.
\item 在下列积分中改变积分的顺序:
\begin{table}[H]
	\begin{tabular}{ll}
		\qquad(1)\ $\displaystyle{\int_{0}^{3}\mathrm{d}x\int_{0}^{\mathrm{ln}x}f(x,y)\mathrm{d}y}$.\qquad \qquad \qquad \qquad \qquad &(2)\ $\displaystyle{\int_{0}^{2}\mathrm{d}y\int_{y^2}^{3y}f(x,y)\mathrm{d}x}$.\\
		\qquad(3)\ $\displaystyle{\int_{-1}^{1}\mathrm{d}x\int_{-\sqrt{1-x^2}}^{1-x^2}f(x,y)\mathrm{d}y}$;\qquad \qquad \qquad \qquad \qquad &(4)\ $\displaystyle{\int_{1}^{2}\mathrm{d}x\int_{\sqrt{x}}^{2}f(x,y)\mathrm{d}y}$.
	\end{tabular}
\end{table}
\item 计算下列二重积分:
\begin{enumerate}
	\item $\Omega$是由$y^2=2px\ (p>0)$与$x=\frac{p}{2}$围成的区域, 求
	$$ \displaystyle{\underset{\Omega}{\iint }x^my^k\mathrm{d}x\mathrm{d}y}\ \ (m>0,k\text{为正整数});$$
	\item $\Omega=\{(x,y)|0\le x\le y^2,0\le y \le 2+x,x\le 2\}$, 求$\displaystyle{\underset{\Omega}{\iint}xy\mathrm{d}x\mathrm{d}y}$;
	\item $\Omega$是由$y=\sqrt{1-x^2},y=0$围成, 求$\displaystyle{\underset{\Omega}{\iint}(x^2+3xy^2)\mathrm{d}x\mathrm{d}y}$;
	\item $\Omega$是由$y=\mathrm{e}^x, y=1,x=0$及$x=1$围成, 求$\displaystyle{\underset{\Omega}{\iint}(x+y)\mathrm{d}x\mathrm{d}y}$;
	\item $\Omega$是以$(1,1),(2,3),(3,1)$和$(4,3)$为顶点的四边形, 求$$
	\displaystyle{\underset{\Omega}{\iint}(x+y)\mathrm{d}x\mathrm{d}y};$$
	\item $\Omega$是由$y=x^2,y=4x$和$y=4$围成, 求$\displaystyle{\underset{\Omega}{\iint}\mathrm{sin}x\mathrm{d}x\mathrm{d}y}$.
\end{enumerate}
\item 计算下列积分:
\begin{table}[H]
	\begin{tabular}{ll}
		\qquad	(1)\ $\displaystyle{\int_{0}^{\frac{\pi}{2}}\mathrm{d}y\int_{y}^{\frac{\pi}{2}}\frac{\mathrm{sin}x}{x}\mathrm{d}x}$;\qquad \qquad \qquad \qquad &(2)\ $\displaystyle{\int_{0}^{1}\mathrm{d}y\int_{y}^{1}\mathrm{e}^{-x^2}\mathrm{d}x}$.
	\end{tabular}
\end{table}
\item 设在$D=[a,b]\times[c,d]$上定义的二元函数$f(x,y)\in C^2(D)$, 证明:
\begin{enumerate}
		\item $\displaystyle{\underset{D}{\iint}f''_{xy}(x,y)\mathrm{d}x\mathrm{d}y}=\displaystyle{\underset{D}{\iint}f''_{yx}(x,y)\mathrm{d}x\mathrm{d}y}$;
		\item 利用(1)证明$f''_{xy}(x,y)=f''_{yx}(x,y),(x,y)\in D$(这里不准用偏导与秩序无关定理).
\end{enumerate}
\item 设$f(x),g(x)\in R[a,b], D=[a,b]\times[a,b]$, 考虑$[f(x)g(y)-g(x)f(y)]^2$在$D$上的重积分, 证明:
$\left(\displaystyle{\int_{a}^{b}f(x)g(x)\mathrm{d}x} \right)^2\le \displaystyle{\int_{a}^{b}f^2(x)\mathrm{d}x\cdot \int_{a}^{b}g^2(x)\mathrm{d}x}$.
\item 求下列立体$\Omega$的体积:
\begin{enumerate}
	\item $\Omega$是由曲线$z=xy,x+y+z=1$和$z=0$围成;
	\item $\Omega$是由$y^2+z^2=1,|x+y|=1,|x-y|=1$围成.
\end{enumerate}
\item 证明: 若$b>a>0$, 则有
\begin{table}[H]
	\begin{tabular}{ll}
	\qquad	(1)\ $\lim\limits_{T\rightarrow \infty}\displaystyle{\int_{0}^{T}\mathrm{d}x\int_{a}^{b}\mathrm{e}^{-xy}\mathrm{d}y=\mathrm{ln}\frac{b}{a}}$;\qquad \qquad \qquad \qquad & (2)\ $\displaystyle{\int_{0}^{+\infty}\frac{\mathrm{e}^{-ax}-\mathrm{e}^{-bx}}{x}\mathrm{d}x=\mathrm{ln}\frac{b}{a}}$.
	\end{tabular}
\end{table}
\item 设$f(t)$在$t\ge 0$上连续可微, 而且$\displaystyle{\int_{0}^{+\infty}\frac{f(t)}{t}\mathrm{d}t}$收敛. 证明:当$b>a>0$时, 有
\begin{enumerate}
	\item $\lim\limits_{T\rightarrow \infty}\int_{0}^{T}\mathrm{d}x\int_{a}^{b}f'(xy)\mathrm{d}y=-f(0)\mathrm{ln}\frac{b}{a}$;
	\item $\displaystyle{\int_{0}^{+\infty}\frac{f(ax)-f(bx)}{x}\mathrm{d}x=f(0)\mathrm{ln}\frac{b}{a}}$.
\end{enumerate}
\item 设$f(x,y)$在$x^2+y^2\le R^2$上可积, $0<h<R$, 令
$$	F(\xi,\eta) = \underset{(x-\xi)^2+(y-\eta)^2\le h^2}{\iint}f(x,y)\mathrm{d}x\mathrm{d}y.$$
证明:$F(\xi,\eta)$在$\xi^2+\eta^2\le (R-h)^2$上连续.
\item 证明下列三重积分化为累次积分的顺序(只写出$\mathrm{d}x,\mathrm{d}z$互换的顺序):
\begin{table}[H]
	\begin{tabular}{ll}
	\qquad	(1)\ $\displaystyle{\int_{0}^{1}\mathrm{d}x\int_{0}^{1-x}\mathrm{d}y\int_{0}^{x+y}}\mathrm{d}z$;\qquad \qquad \qquad \qquad &(2)\ $\displaystyle{\int_{-1}^{1}\mathrm{d}x\int_{-\sqrt{1-x^2}}^{\sqrt{1-x^2}}\mathrm{d}y\int_{\sqrt{x^2+y^2}}^{1}f\mathrm{d}z}$.
	\end{tabular}
\end{table}
\item 计算下列三重积分:
\begin{enumerate}
		\item $\displaystyle{\underset{\Omega}{\iiint}xy^2z^3\mathrm{d}x\mathrm{d}y\mathrm{d}z}$,\ $\Omega$是由曲面$z=xy,y=x,x=1,z=0$所围成;
		\item $\displaystyle{\underset{\Omega}{\iiint}\frac{\mathrm{d}x\mathrm{d}y\mathrm{d}z}{(1+x+y+z)^3}}$, $\Omega$是由曲面$x+y+z=1,x=0,y=0,z=0$所围成;
		\item $\displaystyle{\underset{\Omega}{\iiint}\mathrm{cos}az\mathrm{d}x\mathrm{d}y\mathrm{d}z}$, $\Omega:x^2+y^2+z^2\le R^2$;
		\item $\displaystyle{\underset{\Omega}{\iiint}(1+x^4)\mathrm{d}x\mathrm{d}y\mathrm{d}z}$, $\Omega$是由曲面$x^2=y^2+z^2,x=2,x=1$所围成.
\end{enumerate}
\item 计算三重积分
$$I=\displaystyle{\int_{0}^{1}\mathrm{d}x\int_{x}^{1}\mathrm{d}y\int_{y}^{1}y\sqrt{1+z^4}\mathrm{d}z}.$$
\end{enumerate}
\section{重积分的变换}
\centering{\textbf{练习题}}
\begin{enumerate}
	\item 计算下列积分:
	\begin{enumerate}
		\item $\displaystyle{\underset{\Omega}{\iint}(x^2+y^2)\mathrm{d}x\mathrm{d}y}$, $\Omega$是由曲线$(x^2+y^2)^2=2xy$围成;
		\item $\displaystyle{\underset{\Omega}{\iint}x\mathrm{d}x\mathrm{d}y}$, $\Omega$是由阿基米德螺线$r=\theta$和半射线$\theta=\pi$围成;
		\item $\displaystyle{\underset{\Omega}{\iint}xy\mathrm{d}x\mathrm{d}y}$, $\Omega$是由对数螺线$r=\mathrm{e}^\theta$和半射线$\theta = 0, \theta=\frac{\pi}{2}$围成.
	\end{enumerate}
\item 求下列曲面围成的体积:
\begin{enumerate}
	\item $z=xy,x^2+y^2=a^2,z=0$;
	\item $z=x^2+y^2,x+y+z=1$;
	\item $x^2+y^2+z^2=a^2,x^2+y^2\le a|x|(a>0)$.
\end{enumerate}
\item 求下列积分:
\begin{enumerate}
	\item $\displaystyle{\underset{\Omega}{\iint}\sqrt{1-\frac{x^2}{a^2}-\frac{y^2}{b^2}}\mathrm{d}x\mathrm{d}y}$, $\Omega$是由$\frac{x^2}{a^2}+\frac{y^2}{b^2}=1$围成;
	\item $\displaystyle{\underset{\Omega}{\iint}(x^2+y^2)\mathrm{d}x\mathrm{d}y}$, $\Omega$是由$x^4+y^4=1$围成;
	\item $\displaystyle{\underset{\Omega}{\iint}(x+y)\mathrm{d}x\mathrm{d}y}$, $\Omega$是由$y=4x^2,y=9x^2,x=4y^2,x=9y^2$围成;
	\item  $\displaystyle{\underset{\Omega}{\iint}xy\mathrm{d}x\mathrm{d}y}$, $\Omega$是由$xy=2,xy=4,y=x,y=2x$围成.
\end{enumerate}
\item $D$是以$(x_1,y_1),(x_2,y_2),(x_3,y_3)$为顶点, 面积为$A\ (>0)$的三角形, 求
$$\underset{\Omega}{\iint}x^2\mathrm{d}x\mathrm{d}y$$.
\item 
\begin{enumerate}
	\item 计算积分$$ I = \displaystyle{\underset{x^2+y^2\le R^2}{\iint}\mathrm{ln}\frac{1}{\sqrt{(x-h)^2+y^2}\mathrm{d}x\mathrm{d}y}\ (h>R)}$$
	\item 写出圆的单层位势
	$$u(a,b)=\displaystyle{\underset{x^2+y^2\le R^2}{\iint}\mathrm{ln}\frac{1}{\sqrt{(x-a)^2+(y-b)^2}}\mathrm{d}x\mathrm{d}y}\ \ (a^2+b^2>R^2)$$
\end{enumerate}
\item 设$f(x,y)$在$x^2+y^2\le 1$上连续可微, 求
$$ I=\displaystyle{\underset{x^2+y^2\le 1}{\iint}\frac{xf'_y-yf'_x}{\sqrt{x^2+y^2}}\mathrm{d}x\mathrm{d}y}$$
\item 给定积分$I=\displaystyle{\underset{D}{\iint} [(\frac{\partial f}{\partial x})^2+(\frac{\partial f}{\partial y})^2]\mathrm{d}x\mathrm{d}y}$, 作正则变换$x=u(x,y),y=y(u,v)$, 区域$D$变为$\Omega$, 如果变换满足:
$$ \frac{\partial x}{\partial u}=\frac{\partial y}{\partial v},\ \frac{\partial x}{\partial v}=-\frac{\partial y}{\partial u},$$
证明:
$$ I = \displaystyle{\underset{\Omega}{\iint}[(\frac{\partial f}{\partial u})^2+(\frac{\partial f}{\partial v})^2]\mathrm{d}u\mathrm{d}v}.$$
\item 求下列积分:
\begin{enumerate}
	\item $\displaystyle{\underset{\Omega}{\iiint}(x^2+y^2)^2\mathrm{d}x\mathrm{d}y\mathrm{d}z}$, $\Omega$由曲面$z=x^2+y^2,z=1,z=2$围成;
	\item $\displaystyle{\underset{\Omega}{\iiint}(\sqrt{x^2+y^2})^3\mathrm{d}x\mathrm{d}y\mathrm{d}z}$, $\Omega$由曲面$x^2+y^2=9,x^2+y^2=16,z^2=x^2+y^2,z\ge 0$围成;
	\item $\displaystyle{\underset{\Omega}{\iiint}(x^2+y^2)\mathrm{d}x\mathrm{d}y\mathrm{d}z}$, $\Omega$由曲面$z=16(x^2+y^2), z=4(x^2+y^2), z=16$围成.
\end{enumerate}
\item 求下列积分:
\begin{enumerate}
	\item $\displaystyle{\underset{\Omega}{\iiint}x^3\mathrm{d}x\mathrm{d}y\mathrm{d}z}$, $\Omega$由$x^2+y^2+z^2\le 1,x\ge 0,y\ge 0,z\ge 0$所确定;
	\item $\displaystyle{\underset{\Omega}{\iiint}(\sqrt{x^2+y^2+z^2})^5\mathrm{d}x\mathrm{d}y\mathrm{d}z}$, $\Omega$由不等式$x^2+y^2+z^2\le 2z$所确定;
	\item $\displaystyle{\underset{\Omega}{\iiint}x^2\mathrm{d}x\mathrm{d}y\mathrm{d}z}$, $\Omega$是由$x^2+y^2\le z^2, x^2+y^2+z^2\le 8$所确定.
\end{enumerate}
\item 求下列积分:
\begin{enumerate}
	\item $\Omega$由$z=\frac{x^2+y^2}{a},z=\frac{x^2+y^2}{b},xy=c,xy=d,y=\alpha x,y=\beta x$围成(其中$0<a<b,0<c<d,0<\alpha<\beta$), 求
	$$ \displaystyle{\underset{\Omega}{\iiint}x^2y^2z\mathrm{d}x\mathrm{d}y\mathrm{d}z};$$
	\item $\Omega$由$x=az^2,x=bz^2(z>0,0<a<b), x=\alpha y,x=\beta y\ (0<\alpha<\beta)$以及$x=h (>0)$围成, 求$$
	\displaystyle{\underset{\Omega}{\iiint}y^4\mathrm{d}x\mathrm{d}y\mathrm{d}z}.$$
	\item $\Omega$由$\frac{x^2}{a^2}+\frac{y^2}{b^2}+\frac{z^2}{c^2}=1$围成, 求
	$$ \displaystyle{\underset{\Omega}{\iiint}\mathrm{e}^{\sqrt{\frac{x^2}{a^2}+\frac{y^2}{b^2}+\frac{z^2}{c^2}}}\mathrm{d}x\mathrm{d}y\mathrm{d}z}.$$
\end{enumerate}
\item 设一元函数$f(t)\in C[0,+\infty)$.令
$$ F(t)=\displaystyle{\underset{\Omega}{\iiint}f(\frac{x^2}{a^2}+\frac{y^2}{b^2}+\frac{z^2}{c^2})\mathrm{d}x\mathrm{d}y\mathrm{d}z},$$
其中$\Omega_t=\{(x,y,z)|\frac{x^2}{a^2}+\frac{y^2}{b^2}+\frac{z^2}{c^2}\le t^2\}$.证明:
\begin{table}[H]
	\begin{tabular}{ll}
		\qquad (1)\ $F(t)\in C^{1}[0,+\infty)$;\qquad \qquad \qquad &(2)\ 求出$F'(t)$的表达式.
	\end{tabular}
\end{table}
\item 设$\Omega$是由平面$x+y+z=1, y=0, z=0, x=0$围成的区域. 证明
$$ \displaystyle{\underset{\Omega}{\iiint}x^py^qz^s(1-x-y-z)^t\mathrm{d}x\mathrm{d}y\mathrm{d}z}=\frac{\Gamma(p+1)\Gamma(q+1)\Gamma(s+1)\Gamma(t+1)}{\Gamma(p+q+s+t+4)},$$
其中$p \ge 0,q\ge 0,s\ge 0,t\ge 0$.
\item 设$\Omega$是以$(x_i,y_i,z_i)\ (i=1,2,3,4)$为顶点, 体积为$V(>0)$的四面体, 求$$
\displaystyle{\underset{\Omega}{\iiint}x\mathrm{d}x\mathrm{d}y\mathrm{d}z.}$$
\item 用广义球坐标求$n$维球的体积, 即求$x_1^2,x_2^2+\cdots+x_n^2\le \bm{R}^2$的体积. 所谓广义球坐标即为
$$\begin{cases}
x_1=r\mathrm{cos}\theta_1,&\qquad \qquad r\ge0,\\
x_2=r\mathrm{sin}\theta_1\mathrm{cos}\theta_2,& \qquad \qquad 0\le \theta_i\le \pi,\\
\qquad \vdots&\qquad\qquad\qquad i=1,\cdots,n-2,\\
x_{n-1}=r\mathrm{sin}\theta_1\mathrm{sin}\theta_2\cdots\mathrm{sin}\theta_{\theta_{n-2}}\mathrm{cos}\theta_{n-1},&\qquad \qquad  0\le \theta_{n-1}\le 2\pi.\\
x_n=r\mathrm{sin}\theta_1\mathrm{sin}\theta_2\cdots\mathrm{sin}\theta_{n-2}\mathrm{sin}\theta_{n-1},&\qquad
\end{cases}$$
\item 求$n$面体: $x_i\ge 0(i=1,2,\cdots,n),x_1+x_2+\cdots+x_n\le a\ (a>0)$的容积.
\item 证明
$$\displaystyle{\underset{\underset{x_i\ge 0(i=1,\cdots,n) }{0\le \sum\limits_{i=1}^{n}x_i\le a}}{\int\dotsi\int}f(x_1+x_2+\cdots+x_n)\mathrm{d}x_1\mathrm{d}x_2\cdots\mathrm{d}x_n}=\int_{0}^{a}f(x)\frac{x^{n-1}}{(n-1)!\mathrm{d}x}.$$
\end{enumerate}

\section{曲线积分与格林公式}
\centering{\textbf{练习题}}
\begin{enumerate}
	\item 求下列第一型曲线积分:
	\begin{enumerate}
		\item $\displaystyle{\int_{L}y^2\mathrm{d}s,\ L}$为摆线的一拱: $x=a(t-\mathrm{sin}t),\  y=a(1-\mathrm{sin}t)$,\  $y=a(1-\mathrm{cos}t)\ \  
	\text{其中},(0\le t\le 2\pi)$;
		\item $\displaystyle{\int_{L}(x^{\frac{4}{3}}+y^{\frac{4}{3}})\mathrm{d}s}$, $L$为内摆线: $x^{\frac{2}{3}}+y^{\frac{2}{3}}=a^{\frac{2}{3}}$.
		\item $\displaystyle{\int_{L}xyz\mathrm{d}s}$, $L$为螺线: $x=a\mathrm{cos}t, y=a\mathrm{sin}t, z=bt\ (0\le t \le 2\pi)(0<a<b)$.
	\end{enumerate}
\item 计算第一型曲线积分:
\begin{enumerate}
	\item $\displaystyle{\int_{L}(xy+yz+zx)\mathrm{d}s}$, $L$为球面 $x^2+y^2+z^2=a^2$与平面$x+y+z=0$之交线;
		\item $\displaystyle{\int_{L}xyz\mathrm{d}s}$, $L$同上.
\end{enumerate}
\item 
\begin{enumerate}
	\item 求第一型曲线积分:$$
	I = \displaystyle{\int\limits_{x^2+y^2=R^2}\mathrm{ln}\frac{1}{\sqrt{(x-h)^2+(y-b)^2}}\mathrm{d}s\ \ (h\ne R)};$$
	\item 写出圆周的单层位势:
	$$ U(a,b) = \displaystyle{\int\limits_{x^2+y^2=R^2}\mathrm{ln}\frac{1}{\sqrt{(x-a)^2+(y-b)^2}}\mathrm{d}s},$$
	其中$a^2+b^2\ne R^2$.
\end{enumerate}
\item 设$f(x,y)$在$L$上连续, $L$是一封闭的逐段光滑. 证明:
$$ u(x,y) = \oint_{L} f(\xi,\eta)\mathrm{ln}\frac{1}{\sqrt{(\xi-x)^2+(\eta-y)^2}\mathrm{d}s}$$
当$x^2+y^2\rightarrow +\infty$时趋于零的充要条件是$\oint_{L}f(\xi,\eta)\mathrm{d}s=0$.
\item 设$u(x,y)$在$\bm{R}^2$上连续, 对任意$r>0$. 证明: 等式
$$ u(x,y) = \frac{1}{\pi r^2}\displaystyle{\iint\limits_{(\xi-x)^2+(y-\eta)^2\le r^2}u(\xi,\eta)\mathrm{d}\xi\mathrm{d}\eta}$$
成立的充要条件是等式
$$ u(x,y) = \frac{1}{2\pi r}\displaystyle{\int\limits_{(\xi-x)^2+(\eta-y)^2=r^2}u(\xi,\eta)\mathrm{d}s=\frac{1}{2\pi}\int_{0}^{2\pi}u(x+r\mathrm{cos}\theta,y+r\mathrm{sin}\theta)\mathrm{d}\theta}\ \ (\forall r>0)$$
成立.
\item 求下列第二型曲线积分:
\begin{enumerate}
	\item  $\displaystyle{\int_{\widehat{AB}}(x-2xy^2)\mathrm{d}x+(y-2x^2y)\mathrm{d}y}$, 其中$A(0,0), B(2,4), \widehat{AB}:y=x^2$;
	\item $\displaystyle{\int_{\widehat{AB}}(x+y)\mathrm{d}x+xy\mathrm{d}y}$, 其中$A(0,0), B(2,0), \widehat{AB}: y=1-|1-x|$;
	\item $\displaystyle{\int_{\widehat{AB}(x-y)\mathrm{d}x+(y-z)\mathrm{d}y+(z-x)\mathrm{d}z}}$, 其中$A(0,0,0), B(1,1,1), \widehat{AB}:x=t,y=t^2,z=t^3$;
	\item $\displaystyle{\int_{\widehat{AB}}y^2\mathrm{d}x+z^2\mathrm{d}y+x^2\mathrm{d}z}$, 其中$A(\alpha,0,0),B(\alpha,0,2\pi\gamma),\widehat{AB}: x=a\mathrm{cos}t, y=\beta\mathrm{sin}t,z=\gamma t$ $(\alpha,\beta,\gamma\text{为正数})$.
\end{enumerate}
\item 求第二型曲线积分
$$ \displaystyle{\int_{L}(y^2-z^2)\mathrm{d}x+(z^2-x^2)\mathrm{d}y+(x^2-y^2)\mathrm{d}z}.$$
\begin{enumerate}
	\item $L$为球面三角形$x^2+y^2+z^2=1, x\ge 0,y\ge 0,z\ge	0$的边界线, 从球的外侧看去, $L$的方向为逆时针方向;
	\item $L$是球面$x^2+y^2+z^2=a^2$和柱面$x^2+y^2=ax(a>0)$的交线位于$xy$平面上方部分, 从$x$轴上$(b,0,0)\ (b>a)$点看去, $L$的方向是顺时针方向.
\end{enumerate}
\item 求第二型曲线积分
$$ \displaystyle{\oint_{L}\frac{x\mathrm{d}x+y\mathrm{d}y}{x^2+y^2}}.$$
\begin{enumerate}
	\item $L$为圆周$x^2+y^2=a^2$, 逆时针方向;
	\item $L$为正方形$|x|\le 1,|y|\le 1$的边界, 逆时针方向.
\end{enumerate}
\item 计算第二型曲线积分
$$ \displaystyle{\oint_{L}(x^2+y^2)\mathrm{d}x+(x+y)^2\mathrm{d}y},$$
$L$为圆周$x^2+y^2=ax$, 逆时针方向.
\item 设$P, Q, R$为$L$上的连续函数, $L$为光滑弧段, 弧长为$l$. 证明:
$$ |\displaystyle{\oint_{L}P\mathrm{d}x+Q\mathrm{d}y+R\mathrm{d}z}| \le M\cdot l,$$
其中$M=\underset{(x,y,z)\in L}{\mathrm{max}\{\sqrt{P^2+Q^2+R^2}\}}$.
\item 计算下列积分:
\begin{enumerate}
	\item $\displaystyle{\oint{\partial D}xy^2\mathrm{d}y-yx^2\mathrm{d}x}, D: \frac{x^2}{a^2}+\frac{y^2}{b^2}\le 1$;
	\item $\displaystyle{\oint_{\partial D}(x^2+y^3)\mathrm{d}x-(x^3-y^2)\mathrm{d}y}, D: x^2+y^2\le 1$;
	\item $\displaystyle{\oint_{\partial D}e^y\mathrm{sin}x\mathrm{d}x+\mathrm{e}^{-x}\mathrm{sin}y\mathrm{d}y}$, $D: 0\le x\le b,c\le y\le d$.
\end{enumerate}
\item 计算下列积分:
\begin{enumerate}
	\item $\displaystyle{\int_{\widehat{AO}(x^2+y^2)\mathrm{d}x+(x+y)^2\mathrm{d}y, A(a,0),O(0,0),\widehat{AO}:x^2+y^2=ax\ (y\ge 0)}}$;
		\item $\displaystyle{\int_{\widehat{OA}\mathrm{e}^x[1-\mathrm{cos}y\mathrm{d}x-(y-\mathrm{sin}y\mathrm{d}y)]},A(\pi,0),O(0,0),\widehat{AO}:y=\mathrm{sin}x}$;
		\item $\displaystyle{\oint_{\widehat{OA}}\mathrm{e}^{-(x^2-y^2)}[x(1-x^2-y^2)\mathrm{d}x+y(1+x^2+y^2)\mathrm{d}y],A(1,1),O(0,0),\widehat{OA}:y=x^2}$.
\end{enumerate}
\item 设$C$为光滑的简单闭曲线, 求下列积分:
\begin{enumerate}
	\item $\displaystyle{\oint_{c}\mathrm{cos}<\bm{l},\bm{n}>\mathrm{d}s,\bm{l}}$为给定的方向, $\bm{n}$为C的外法线方向;
	\item $\displaystyle{\oint_{C}\mathrm{cos}<\bm{r},\bm{n}>\mathrm{d}s,\bm{r}=x\bm{i}+y\bm{j},\bm{n}}$为$C$的外法线方向.
\end{enumerate}
\item 
\begin{enumerate}
	\item 设$f(x,y)=\mathrm{ln}\frac{1}{\sqrt{(x-a)^2+(y-b)^2}(a^2+b^2\ne R^2)}$, 试证函数在圆$x^2+y^2=R^2$上每点沿外法线方向$n$的方向导数为
	$\frac{\partial f}{\partial \bm{n}}=-\frac{(x-a)\mathrm{cos}<\bm{\tau},\bm{j}>-(y-b)\mathrm{cos}<\bm{\tau,\bm{i}}}{(x-a)^2+(y-b)^2}$,
	其中$\tau$为圆的切向量, $\bm{i,j}$分别为$x$轴, $y$轴上的单位向量;
	\item 求圆周$x^2+y^2=R^2$的双层位势
	$$ u(a,b) = \displaystyle{\int\limits_{x^2+y^2=R^2}\frac{(x-a)\mathrm{d}y-(y-b)\mathrm{d}x}{(x-a)^2+(y-b)^2}}\ (a^2+b^2\ne R^2)$$
\end{enumerate}
\item 
\begin{enumerate}
	\item 求积分$I=\displaystyle{\int_{\partial D\mathrm{e}^{-(R^2-y^2)}(\mathrm{cos}2xy\mathrm{d}x+\mathrm{sin}2xy\mathrm{d}y)}, D: |x|\le R,0\le y\le b}$;
		\item 证明: $\displaystyle{\lim\limits_{R\rightarrow +\infty}\mathrm{e}^{-(R^2-y^2)}\mathrm{sin}2Ry\mathrm{d}y=0}$;
		\item 证明: $\displaystyle{\int_{-\infty}^{+\infty}\mathrm{e}^{-x^2}\mathrm{cos}2bx\mathrm{d}x=\sqrt{\pi}\mathrm{e}^{-b^2}}$.
\end{enumerate}
\item 设$A>0, C>0, AC-B^2>0$, 求证:
$$ \displaystyle{\oint_{L}\frac{x\mathrm{d}y-y\mathrm{d}x}{Ax^2+2Bxy+Cy^2}=\frac{2\pi}{AC-B^2}},\  L: x^2 + y^2 = R^2.$$
\item 设$f(x,y)$在上半平面$y>0$上连续可微. 证明: 对上平面上的任一光滑闭曲线$C$, 等式
$$ \displaystyle{\oint_{C}f(x,y)(x\mathrm{d}y-y\mathrm{d}x)=0}$$
成立的充要条件是: $f(x,y)$为2次齐次函数.
\item 计算线积分
	$$ I = \displaystyle{\oint_{L}\frac{\mathrm{e}^x}{x^2+y^2}[(x\mathrm{cos}y+y\mathrm{sin}y)\mathrm{d}y+(x\mathrm{sin}y-y\mathrm{cos}y)\mathrm{d}x]},$$
	其中$L$是包含原点在其内部的光滑简单闭曲线.
	\item 设$C$是逐段光滑简单闭曲线, 它围成的区域记作$D$, 函数$u(x,y),v(x,y)\in C^2(\overline{D})$.证明$$
	\displaystyle{\oint_{C}v\frac{\partial u}{\partial n}\mathrm{d}s=\iint\limits_{D}v[\frac{\partial^2 u}{\partial x^2}+\frac{\partial^2u}{\partial y^2}]\mathrm{d}x\mathrm{d}y+\iint\limits_{D}[\frac{\partial u}{\partial x}\frac{\partial v}{\partial x}+\frac{\partial u}{\partial y}\frac{\partial v}{\partial y}]\mathrm{d}x\mathrm{d}y}.$$
	\item 设$C$和$D$的条件同上题, $u(x,y)$是$D$上调和函数. 证明: 若$u(x,y)|c=0$, 则$u(x,y)\equiv 0,(x,y)\in D$. 
\end{enumerate}
\section{曲面积分}
\centering{\textbf{练习题}}
\begin{enumerate}
	\item 球环面$x=(b+a\mathrm{cos}\phi)\mathrm{cos}\theta, y=(b+a\mathrm{cos}\phi)\mathrm{sin}\theta,z=a\mathrm{sin}\phi(0<a<b)$被两条经线$\theta=\theta_1,\theta=\theta_2$和两条纬线$\phi=\phi_1,\phi=\phi_2$所围成的那部分面积,并求出整个环面面积.
	\item 求螺旋面$x=r\mathrm{cos}\phi, y=r\mathrm{sin}\phi,z=h\phi,(0<r<a,0<\phi<2\pi)$的面积. 
	\item 求球面$x^2+y^2+z^2=a^2$包含在柱体$y^2+z^2\le 1(a>1)$中那部分的面积.
	\item 求曲面$z=\sqrt{2xy}$被平面$x+y=1,x=1$及$y=1$所截下的那部分面积.
	\item 求曲面$x^2+y^2=\frac{1}{3}z^2,\sqrt{2}x+z=2a\ (a>0)$围成的立体的表面积
	\item 平面上一椭圆绕其长轴旋转得一旋转椭球$\Omega$, 求$\Omega$之表面积.
	\item 求下列第一型曲面积分:
	\begin{enumerate}
		\item $\displaystyle{\iint_{S}(x^2+y^2)\mathrm{d}S}$, $S$为立体$\sqrt{x^2+y^2}\le x\le 2$的边界面;
		\item $\displaystyle{\iint_{S}|xyz|\mathrm{d}S}$, $S$为曲面$z=x^2+y^2$被平面$z=1$割下的部分;
		\item $\displaystyle{\iint_{S}z^2\mathrm{d}S}$, $S$为螺旋面: $x=u\mathrm{cos}v,y=\mathrm{sin}v,z=v\ (0\le u\le a, 0\le v\le 2\pi)$;
		\item $\displaystyle{\iint_{S}(x^2+y^2)\mathrm{d}S}$, $S:x^2+y^2+z^2=R^2$.
	\end{enumerate}
\item 设$f(x)$为一元连续函数. 证明: 普阿松公式$$ \displaystyle{\iint\limits_{S}f(ax+by+cz)\mathrm{d}S=2\pi\iint_{-1}^{1}f(\sqrt{a^2+b^2+c^2}\xi)\mathrm{d}\xi},$$
其中$S$为球面: $x^2+y^2+z^2=1$.
\item 计算$F(t)=\displaystyle{\iint_{S}f(x,y,z)\mathrm{d}S}$, 其中$S$是一平面$x+y+z=t$, 而$$
f(x,y,z)=\begin{cases}
1-x^2-y^2-z^2,\ &\qquad x^2+y^2+z^2\le 1\\
0,\ &\qquad x^2+y^2+z^2>1,
\end{cases}$$

并做出$F(t)$的图形.
\item 求下列第二型曲面积分:
\begin{enumerate}
	\item $\displaystyle{\iint\limits_{S}x^2\mathrm{d}y\mathrm{d}z+y^3\mathrm{d}z\mathrm{d}x+z^3\mathrm{d}x\mathrm{d}y}$, $S$为球面$x^2+y^2+z^2=R^2$的外侧.
	\item $\displaystyle{\iint\limits_{S}x^2\mathrm{d}y\mathrm{d}z+y^2\mathrm{d}z\mathrm{d}x+z^2\mathrm{d}x\mathrm{d}y}$, $S$是立体$\Omega$的边界线的外侧, $\Omega$的表达式为$$
	\Omega = \{(x,y,z)|0\le x\le a,0\le y\le b,0\le z\le c\};$$
	\item $\displaystyle{\iint\limits_{S}x\mathrm{d}y\mathrm{d}z+y\mathrm{d}z\mathrm{d}x+z\mathrm{d}x\mathrm{d}y}$, $S$为球面
	$$ (x-a)^2+(y-b)^2+(z-c)^2=R^2$$的上半部分的上侧;
	\item $\displaystyle{\iint\limits_{S}\left(\frac{\mathrm{d}y\mathrm{d}z}{x}+\frac{\mathrm{d}z\mathrm{d}x}{y}+\frac{\mathrm{d}x\mathrm{d}y}{z}\right)}$, $S$为椭球面$\frac{x^2}{a^2}+\frac{y^2}{b^2}+\frac{z^2}{c^2}=1$的外侧.
\end{enumerate}
\end{enumerate}

\section{奥氏积分、斯托克斯公式、线积分与路径无关}
\centering{\textbf{练习题}}
\begin{enumerate}
	\item 利用奥斯公式求下列积分:
	\begin{enumerate}
		\item $\displaystyle{\iint\limits_{S}x\mathrm{d}y\mathrm{d}z}+y\mathrm{d}z\mathrm{d}x+z\mathrm{d}x\mathrm{d}y$,~$S:~(x-a)^2+(y-b)^2+(z-c)^2=R^2$, 外侧;
		\item $\displaystyle{\iint\limits_{S}x^2\mathrm{d}y\mathrm{d}z+y^2\mathrm{d}z\mathrm{d}x+z^2\mathrm{d}x\mathrm{d}y}$, $S:~x^2+y^2\le z\le h$的边界线, 外侧;
		\item $\displaystyle{\iint\limits_{s}(x-y+z)\mathrm{d}y\mathrm{d}z+(y-z+x)\mathrm{d}z\mathrm{d}x+(z-x+y)\mathrm{d}x\mathrm{d}y}$, $S$为曲面.
	$$ |x-y-z|+|y-z+x|+|z-x+y|=1$$
	的外侧.
	\end{enumerate}
\item 计算下列曲面积分 
\begin{enumerate}
\item $\displaystyle{\iint\limits_{S}(x^2-y^2)\mathrm{d}y\mathrm{d}z+(y^2-z^2)\mathrm{d}z\mathrm{d}x+(z^2-x^2)\mathrm{d}x\mathrm{d}y}$, $S$是$\frac{x^2}{a^2}+\frac{x^2}{a^2}+\frac{y^2}{b^2}+\frac{z^2}{c^2}=1\ (z\ge 0)$的上侧;
\item $\displaystyle{(x+\mathrm{cos}y)\mathrm{d}y\mathrm{d}z+(y+\mathrm{cos}z)\mathrm{d}z\mathrm{d}x+(z+\mathrm{cos}x)\mathrm{d}x\mathrm{d}y}$, 其中$S$为$x+y+z=\pi$在第一卦限部分, 上侧.	
\end{enumerate}

	\item 求 $$ f(x,y,z) = \frac{1}{\sqrt{(x-a)^2+(y-b)^2+(z-c)^2} }\ (a^2+b^2+c^2\ne R^2)$$ 沿球面$x^2+y^2+z^2=R^2$上各点外法线方向的方向导数;
	\item 求球面的双层位势$$u(a,b,c)=\displaystyle{\iint\limits_{x^2+y^2+z^2=R^2}\frac{(x-a)\mathrm{d}x\mathrm{d}z+(y-b)\mathrm{d}z\mathrm{d}x+(z-c)\mathrm{d}x\mathrm{d}y}{[(x-a)^2+(y-b)^2+(z-c)^2]^{\frac{3}{2}}}}\ (a^2+b^2+c^2\ne R^2)$$.
	\item 设$V$为可测闭区域, $\partial V=S$为光滑闭曲面, 函数$u(x,y,z),v(x,y,z)\in C^2(V)$. 证明: 
	$$ \displaystyle{\iint\limits_{S}v\frac{\partial u}{\partial \bm{n}}\mathrm{d}S}=\iiint\limits_{V}v
	(\frac{\partial^2 u}{\partial x^2}+\frac{\partial^2 u}{\partial y^2}+\frac{\partial^2 u}{\partial z^2})
	\mathrm{d}x\mathrm{d}y\mathrm{d}z+\iiint\limits_{V}\left[\frac{\partial u}{\partial x}
	\frac{\partial v}{\partial x}+\frac{\partial u}{\partial y}\frac{\partial v}{\partial y}+\frac{\partial u}{\partial z}
	\frac{\partial v}{\partial z}\right]\mathrm{d}x\mathrm{d}y\mathrm{d}z$$
	其中$\bm{n}$为$S$的外法线方向.
	\item 设$V,S$条件同上题, $u(x,y,z)$为调和函数: $\frac{\partial^2u}{\partial x^2}+\frac{\partial^2 u}{\partial y^2}+\frac{\partial^2 u}{\partial z^2}=0$, 且$u(x,y,z)|s=0$\ (即函数$u$在边界$S$上取值为零). 证明:
	$$ u(x,y,z) \equiv 0\ \ (x,y,z)\in V.$$
	\item 设$V,S$条件同上, $u$为调和函数, $v(x,y,z)|s=0$. 证明:
	$$ \displaystyle{\iiint\limits_{V}\left[\frac{\partial u}{\partial x}\frac{\partial v}{\partial x}+
	\frac{\partial u}{\partial y}\frac{\partial v}{\partial y}+\frac{\partial u}{\partial z}\frac{\partial v}{\partial z}\right]\mathrm{d}x\mathrm{d}y\mathrm{d}z}=0$$
	\item 设$V,S$条件同上, $u, w\in C^2(V)$, $u$是调和函数, 且$$ [w(x,y,z)-u(x,y,z)]|s=0.$$
	证明:
 $$\displaystyle{
		\iiint\limits_{S}\left[\left(\frac{\partial u}{\partial x}\right)^2+\left(\frac{\partial u}{\partial y}\right)^2+\left(\frac{\partial u}{\partial z}\right)^3\right]\mathrm{d}x\mathrm{d}y\mathrm{d}z \le \iiint\limits_{V}\left[\left(\frac{\partial w}{\partial x}\right)^2+\left(\frac{\partial w}{\partial y}\right)+
	\left(\frac{\partial w}{\partial z}\right)^2\right]\mathrm{d}x\mathrm{d}y\mathrm{d}z.}$$.
	\item 求下列曲线积分:
	\begin{enumerate}
	\item $\displaystyle{\oint_{L}(y-z)\mathrm{d}x+(z-x)\mathrm{d}y+(x-y)\mathrm{d}z}$, 式中$L$为椭圆, 即$x^2+y^2=R^2$与$\frac{x}{a}+\frac{z}{h}=1\ (a>0,h>0)$
	的交线, 若从$Ox$轴正向看去, 此椭圆是以反时针方向进行的;
	\item $\displaystyle{\oint_{L}(y-z)\mathrm{d}x+(z-x)\mathrm{d}y+(x-y)\mathrm{d}z}$, 式中$L$为圆周, 即$x^2+y^2+z^2=a^2$与$y=x\mathrm{tan}\alpha(0<\alpha<\pi,\alpha\ne \frac{\pi}{2})$的交线, 若从$Ox$轴的正向看去, 圆周是依反时针方向进行的;
	\item $\displaystyle{\oint_{L}y^2\mathrm{d}x+z^2\mathrm{d}y+x^2\mathrm{d}z}$, 式中$L$为维维安尼曲线: $x^2+y^2+z^2=a^2, x^2+y^2=ax\ (z\ge 0,a>0)$, 若从$Ox$轴正向看去, 曲线是依反时针方向进行的;
	\item $\displaystyle{\oint_{L}(y^2+z^2)\mathrm{d}x+(x^2+z^2)\mathrm{d}y+(x^2+y^2)\mathrm{d}z}$, 式中$L$是曲线: $x^2+y^2+z^2=2Rx,x^2+y^2=2rx\ (0<r<R, z>0)$, 此曲线是如下进行的: 由它所包围的在球$x^2+y^2+z^2=2Rx$外表面上的最小区域保持在左方.
	\end{enumerate}
	\item 下列被积表达式是否是恰当, 并求线积分:
	\begin{enumerate}
		\item $w=(x^2-2yz)\mathrm{d}x+(y^2-2xz)\mathrm{d}y+(z^2-2xy)\mathrm{d}z$, 求$\displaystyle{\int_{(0,0,0)}^{(1,1,1)}w}$;
		\item $w=(yz\mathrm{e}^{xyz}+2x)\mathrm{d}x+(zx\mathrm{e}^{xyz}+3y^2)\mathrm{d}y+(xy\mathrm{e}^{xyz}+4z^3)\mathrm{d}z$, 求$\displaystyle{\int_{(0,0,0)}^{(x,y,z)}w}$;
		\item $w=[2x\mathrm{sin}(x+y+z)+x^2\mathrm{cos}(x+y+z)]\mathrm{d}x+x^2\mathrm{cos}(x+y+z)(\mathrm{d}y+\mathrm{d}z)$, 求$\displaystyle{\int_{(1,2,3)}^{(x,y,z)}w}$.
		\end{enumerate}
	\item 设$\Omega$为包含原点的单连通区域, 线积分$\displaystyle{\int_{\widehat{AB}}P\mathrm{d}x+Q\mathrm{d}y+R\mathrm{d}z}$在$R$上与路径无关, 若$P,Q,R$皆为$n$次齐次函数, 证明: 线积分$$
		\displaystyle{\int_{\widehat{AB}x\mathrm{d}P+y\mathrm{d}Q+z\mathrm{d}R}}$$
		也在$\Omega$上与路径无关.
		\item 设$\Omega$是包含原点的凸区域, $P,Q, R\in C^1(\Omega)$.证明下面四个命题等价:
		\begin{enumerate}
			\item $w=P\mathrm{d}y\mathrm{d}z+Q\mathrm{d}z\mathrm{d}x+R\mathrm{d}x\mathrm{d}y$的曲面积分与曲面无关, 即$S_1,S_2$为定向光滑曲面, $\partial S_1=\partial S_2$, 由$S_1,S_2$的定向决定的边界正定向相同, 则有
			$$\displaystyle{\iint_{S_1}w=\iint_{S_2}w};$$
			\item 设$S$为$\Omega$内一光滑闭曲面, 则有$\displaystyle{\iint_{S}w=0}$;
				\item 被积表达式$w$是封闭的, 即外微分$\mathrm{d}w=0$,或
				$$ \frac{\partial P}{\partial x}+\frac{\partial Q}{\partial y}+\frac{\partial R}{\partial z}\equiv 0$$
				\item $w$是恰当的, 即存在\\
				
				$\qquad \eta = \displaystyle{\int_{0}^{1}t[zQ(tx,ty,tz)-yR(tx,ty,tz)]\mathrm{d}t\mathrm{d}x}$\\ \qquad 
				
				$\qquad\qquad+\displaystyle{\int_{0}^{1}t[xR(tx,ty,tz)-zP(tx,ty,tz)]\mathrm{d}t\mathrm{d}y}$\\ \qquad
				$
				\qquad +\displaystyle{\int_{0}^{1}t[yP(tx,ty,tz)-xQ(tx,ty,tz)]\mathrm{d}t\mathrm{d}z},$\\
				
			使$\mathrm{d}\eta=w$.
	\end{enumerate}
\end{enumerate}


\section{场论}
\centering{\textbf{练习题}}
\begin{enumerate}
	\item 设$u(x,y,z)\in C^2, f(t)\in C^2$. 求
	\begin{table}[H]
		\begin{tabular}{ll}
		  \qquad	(1)\ $\mathrm{grad}f(u)$;\qquad \qquad \qquad \quad & (2)\ $\mathrm{div\ grad}f(u)$.
		\end{tabular}
	\end{table}
\item $c$为常向量, $r=\sqrt{x^2+y^2+z^2}$, $f(r)$可微. 求
	\begin{table}[H]
		\begin{tabular}{ll}
			(1)\ $\mathrm{div}$[$\mathrm{c}\times f(r)\bm{r}$];\qquad \qquad \qquad \qquad &(2)\ $\mathrm{rot}[\bm{c}\times f(r)\bm{r}]$.
		\end{tabular}
	\end{table}
\item 证明: \qquad  (1)\ $\mathrm{rot}(\mathrm{grad}u)=0$;\qquad \qquad \qquad (2)\ $\mathrm{div}(\mathrm{rot}\bm{F})=0$.
\item 设$u=u(x,y,z)$, 作柱坐标变换: $x=r\mathrm{cos}\theta,y=r\mathrm{sin}\theta,z=z$.令$\bm{e}_r,\bm{e}_\theta,\bm{e}_z=\bm{k}$为两两正交的单位向量. 证明
$$\mathrm{grad}u = \frac{\partial u}{\partial r}\bm{e}_r +\frac{1}{r}\frac{\partial u}{\partial \theta}\bm{e}_\theta$$
\item 设$u=u(x,y,z)$, 作球坐标变换: $x=r\mathrm{cos}\theta\mathrm{sin}\phi,y=r\mathrm{sin}\theta\mathrm{sin}\phi,z=r\mathrm{cos}\phi$. 令$\bm{e}_r,\bm{e}_\phi,\bm{e}_\theta$为两两正交的单位向量. 证明
$$ \mathrm{grad}u=\frac{\partial u}{\partial r}\bm{e}_r+\frac{1}{r}\frac{\partial u}{\partial \phi}\bm{e}_\phi+\frac{1}{r\mathrm{sin}\phi}\frac{\partial u}{\partial \theta}\bm{e}_\theta$$
\item 设物体$\Omega$以一定角速度$w$绕轴$\bm{l}=(\mathrm{cos}\alpha,\mathrm{cos}\beta,\mathrm{cos}\gamma)$旋转.
\begin{enumerate}
	\item 求物体$\Omega$上各点的速度, 即求速度场$\bm{v}$;
	\item 求$\mathrm{rot}\bm{v}$.
\end{enumerate}
\item 证明: 场$\bm{F}=yz(2x+y+z)\bm{i}+xz(x+2y+z)\bm{j}+xy(x+y+2z)\bm{k}$是保守场, 并求势函数.
	\item 设$f(x,y,z)$是一次齐次函数, $\bm{F}=\frac{1}{4}f(x,y,z)\bm{r}$.试证:
	$$\mathrm{div}\bm{F}=f(x,y,z)$$
\item 设$\bm{R}^3$空间有一变换$T: x_i=x_i(p_1,p_2,p_3)\ (i=1,2,3)$, 或记作$\bm{x}=T(\bm{p})$.又设向量$\frac{\partial \bm{x}}{\partial p_1},\frac{\partial \bm{x}}{\partial p_2},\frac{\partial \bm{x}}{\partial p_3}$两两相互正交, 记$H_i=|\frac{\partial \bm{x}}{\partial p_i}|\ (i=1,2,3)$单位向量$\bm{e}_i=\frac{1}{H_i}\frac{\partial \bm{x}}{\partial p_i}\ (i=1,2,3)$.又$\bm{F}=F_1\bm{e}_1+F_2\bm{e}_2+F_3\bm{e}_3$.则有$$
\mathrm{div}\bm{F}=\frac{1}{H_1H_2H_3}\sum\limits_{i=1}^{3}\frac{\partial }{\partial p_i}(F_i\frac{H_1H_2H_3}{H_i}).$$
试利用此公式求下列各式:
\begin{enumerate}
	\item 在空间柱坐标系, 求$\mathrm{div}\ \mathrm{grad}u(r,\theta,z)$;
	\item 在空间极坐标系, 求$\mathrm{div}\ \mathrm{grad}u(r,\theta)$
	\item 在空间球坐标系, 求$\mathrm{div}\ \mathrm{grad}u(r,\alpha,\theta)$.
\end{enumerate} 
\end{enumerate}