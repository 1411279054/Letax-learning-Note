\chapter{一元函数积分学}
\section{不定积分和可积函数类}
\centering{\textbf{练习题}}
\begin{enumerate}
\item 求下列不定积分:
\begin{table}[H]
\begin{tabular}{llll}
(1)\ $\int{\frac{e^{3x}+1}{e^x+1}}\mathrm{d}x$;\qquad \qquad &(2)\ $\int\frac{\mathrm{d}x}{x^2(1+x^2)}$;\qquad \qquad&(3)\ $\int \sqrt{x\sqrt{x}}\mathrm{d}x$;\qquad \qquad&(4)\ $\int [\sqrt{\frac{1+x}{1-x}}+\sqrt{\frac{1-x}{1+x}}]\mathrm{d}x$;\\
(5)\ $\int \mathrm{tan}^2x\mathrm{d}x$;\qquad \qquad &(6)\ $\int \frac{1+\mathrm{sin}^2x}{\mathrm{cos}^2x}\mathrm{d}x$;\qquad \qquad &(7)\ $\int \frac{\mathrm{dx}}{\mathrm{cos}^2x\mathrm{sin}^2x}$;\qquad \qquad &(8)\ $\int \frac{1+\mathrm{sin}^2x}{\mathrm{cos}^2x}\mathrm{d}x$;\\
(9)\ $\int \frac{\mathrm{d}x}{3x^2}$;\qquad \qquad &(10)\ $\int \frac{\mathrm{d}x}{2-3x^2}$;\qquad \qquad &(11)\ $\int \sqrt[3]{1-3x}\mathrm{d}x$;\qquad \qquad &(12)\ $\int x\cdot \sqrt[3]{1-3x}\mathrm{d}x$. 

\end{tabular}
\end{table}

\item 求下列不定积分$I=\int \frac{1}{1+x^4}\mathrm{d}x, J=\int \frac{x^2}{1+x^4}\mathrm{d}x$.
\item 求下列不定积分:
\begin{table}[H]
	\begin{tabular}{llll}
(1)\ $\int \frac{\mathrm{d}x}{\sqrt{3x^2-2}}$;\qquad  & (2)\ $\int \frac{\mathrm{d}x}{x\sqrt{x^2+1}}$;\qquad  &(3)\ $\int \sqrt{\frac{a+x}{a-x}}\mathrm{d}x(a>0)$; \qquad &(4)\ $\int \sqrt{\frac{x-a}{x+a}}(a\ge 0)$;\\
(5)\ $\int \frac{\mathrm{d}x}{\sqrt{1+x+x^2}}$; \qquad &(6)\ $\int \frac{x+3}{\sqrt{4x^2+4x+3}}\mathrm{d}x$;
\qquad &(7)\ $\int \frac{\mathrm{d}x}{\sqrt{(x+a)(x+b)}}\mathrm{d}x(a<b)$;\qquad &(8)\ $\int \sqrt{\frac{x}{1-x\sqrt{x}}}\mathrm{d}x$.

\end{tabular}
\end{table}
\item 求下列不定积分:
\begin{table}[H]
	\begin{tabular}{lll}
		(1)\ $\int \frac{\mathrm{d}x}{x^2\sqrt{x^2+1}}$;\qquad \qquad \qquad &(2)\ $\int \sqrt{\frac{x}{1-x\sqrt{x}}}\mathrm{d}x$;\qquad \qquad \qquad&(3)\ $\int \frac{\sqrt{a^2-x^2}}{x}\mathrm{d}x$;\\
		(4)\ $\int \frac{\sqrt{x^2-a^2}}{x}\mathrm{d}x$;\qquad \qquad \qquad&(5)\ $\int x^2\sqrt{4-x^2}\mathrm{d}x$;\qquad \qquad &(6)\ $\int \frac{x}{1+\sqrt{x}}\mathrm{d}x$.
	\end{tabular}
\end{table}
\item 求下列不定积分:
\begin{table}[H]
	\begin{tabular}{ll}
		(1)\ $\int \mathrm{ln}(1+x^2)\mathrm{d}x$;\qquad \qquad \qquad \qquad &(2)\ $\int x^\alpha\mathrm{ln}x\mathrm{d}x$;\\
		(3)\ $\int \sqrt{x}\mathrm{ln}^2x\mathrm{d}x$;\qquad \qquad \qquad \qquad &(4)\ $\int x^2e^{-2x}\mathrm{d}x$;\\
		(5)\ $\int x\mathrm{cos}\beta x\mathrm{d}x$;\qquad \qquad \qquad \qquad &(6)\ $\int x^2\mathrm{sin}2x\mathrm{d}x$;\\
		(7)\ $\int x\mathrm{arctan}x \mathrm{d}x$;\qquad \qquad \qquad \qquad &(8)\ $\int \frac{\mathrm{arcsin}x}{x^2}\mathrm{d}x$;\\
		(9)\ $\int \frac{x}{\mathrm{cos}^2x}\mathrm{d}x$;\qquad \qquad \qquad \qquad &(10)\ $\int \frac{x^2e^x}{(x+2)^2}\mathrm{d}x$.
	\end{tabular}
\end{table}
\item 求下列不定积分:
\begin{table}[H]
	\begin{tabular}{ll}
(1)\ $\int \frac{1+x+x^2}{\sqrt{1+x^2}}\mathrm{e}^x\mathrm{d}x$;\qquad \qquad \qquad \qquad &(2)\ $\int \frac{1+\mathrm{tan}x}{\mathrm{cos}x}\mathrm{e}^x\mathrm{d}x$.\\
(3)\ $\int (\mathrm{cos}x-\mathrm{sin}x)\mathrm{e}^{-x}\mathrm{d}x$;\qquad \qquad \qquad \qquad &(4)\ $\int x(2-x)\mathrm{e}^{-x}\mathrm{d}x$.
	\end{tabular}
\end{table}
\item 求下列不定积分:
\begin{table}[H]
	\begin{tabular}{ll}
		(1)\ $\int \sqrt{a^2-x^2}\mathrm{d}x$;\qquad \qquad \qquad \qquad &(2)\ $\int \sqrt{x^2-a^2}$;\\
		(3)\ $\int \mathrm{arctan}\sqrt{\frac{x}{1+x}}\mathrm{d}x$;\qquad \qquad \qquad \qquad &(4)\ $\int \frac{\mathrm{e}^{\mathrm{arctan}x}}{(1+x^2)^{\frac{3}{2}}}\mathrm{d}x$\\
		(5)\ $\int x\mathrm{arctan}x\mathrm{ln}(1+x^2)\mathrm{d}x$;\qquad \qquad \qquad \qquad &
		(6)\ $\int \frac{x^3\mathrm{arccos}x}{\sqrt{1-x^2}}\mathrm{d}x$;
	\end{tabular}
\end{table}
\item 求下列不定积分:
\begin{table}[H]
	\begin{tabular}{ll}
		(1)\ $\int \mathrm{sin}(\mathrm{ln}x)\mathrm{d}x$; \qquad \qquad \qquad \qquad & (2)\ $\int \mathrm{cos}(\mathrm{ln}x)\mathrm{d}x$;\\
		(3)\ $\int x\mathrm{e}^x\mathrm{cos}x\mathrm{d}x$;\qquad \qquad \qquad \qquad &(4)\ $\int x\mathrm{e}^x\mathrm{sin}x\mathrm{d}x$;
	\end{tabular}
\end{table}
\item 求下列不定积分的递推公式:\\
\begin{table}[H]
	\begin{tabular}{ll}
	(1)\ $\int x^n\mathrm{e}^x\mathrm{d}x$; \qquad \qquad \qquad \qquad &(2)\ $\int x^n(\mathrm{ln}x)^\mathrm{d}x$;\\
		(3)\ $\int \mathrm{sin}^nx\mathrm{d}x$; \qquad \qquad \qquad \qquad &(4)\ $\int \frac{\mathrm{d}x}{\mathrm{sin}^nx}(n\ge2)$.
	\end{tabular}
\end{table}
\item 	求下列不定积分:
\begin{table}[H]
	\begin{tabular}{ll}
	(1)\ $\int \frac{2x+3}{(x-2)(x+5)}\mathrm{d}x$;\qquad \qquad \qquad \qquad & (2)\ $ \int \frac{\mathrm{d}x}{8-2x-x^2}$;\\
	(3)\ $\int \frac{\mathrm{d}x}{(x+1)^2(x-1)}$;\qquad \qquad \qquad \qquad &(4)\ $\int \frac{2x-3}{x^2+2x+1}\mathrm{d}x$;\\
	(5)\ $\int \frac{\mathrm{d}x}{(x+1)(x^2+1)}$;\qquad \qquad \qquad \qquad &(6)\ $\int \frac{x^4}{x^4+5x^2+4}\mathrm{d}x$.
	\end{tabular}
\end{table}
\item 求下列不定积分:
\begin{table}[H]
	\begin{tabular}{llll}
		(1)\ $\int \mathrm{cos}x\mathrm{sin}^2x\mathrm{d}x$; \qquad \qquad  &(2)\ $\int \frac{\mathrm{cos}x}{1+\mathrm{sin}x}\mathrm{d}x$; \qquad \qquad  &(3)\ $\int \mathrm{tan}x\mathrm{sin}^2x\mathrm{d}x$;\qquad  \qquad  &(4)\ $\int \mathrm{tan}^3x\mathrm{d}x$;\\
		(5)\ $\int\mathrm{cos}^4\mathrm{sin}^3x\mathrm{d}x $; \qquad \qquad &(6)\ $\int \frac{\mathrm{sin}^3x}{1+\mathrm{cos}^2x}\mathrm{d}x$; \qquad \qquad &(7)\ $\int \frac{\mathrm{d}x}{\mathrm{sin}^2\mathrm{cos}x}$; \qquad \qquad &(8)\ $\int \frac{\mathrm{sin}2x}{2+\mathrm{tan}^2x}\mathrm{d}x$.
	\end{tabular}
\end{table}
\item 求下列不定积分:
\begin{table}[H]
\begin{tabular}{ll}
	(1)\ $\int \mathrm{sec}^3x\mathrm{d}x$;\qquad \qquad \qquad \qquad &(2)\ $\int \frac{\mathrm{sin}^2x}{1+\mathrm{cos}^2x}$;\\
	(3)\ $\int \frac{\mathrm{sin}2x}{\mathrm{cos}^4x+\mathrm{sin}^4x}\mathrm{d}x$;\qquad \qquad \qquad \qquad &(4)\ $\int \frac{\mathrm{d}x}{2\mathrm{cos}^x+\mathrm{sin}x\mathrm{cos}x+\mathrm{sin}^2x}$.
\end{tabular}
\end{table}
\item 求下列不定积分:
\begin{table}[H]
	\begin{tabular}{ll}
		(1)\ $\int \frac{\mathrm{d}x}{(1+\mathrm{cos}x)^2}$;\qquad \qquad \qquad \qquad &(2)\ $\int \frac{\mathrm{d}\theta}{1+r^2-2r\mathrm{cos}\theta}$\\
		(3)\ $\int \frac{\sqrt{x}}{1+\sqrt[4]{x^3}}\mathrm{d}x$;\qquad \qquad \qquad \qquad &(4)\ $\int \frac{\mathrm{d}x}{1+\sqrt{x}+\sqrt{x+1}}$;\\
		(5)\ $\int \frac{x}{x+\sqrt{x^2-1}}\mathrm{d}x$;\qquad \qquad \qquad \qquad &(6)\ $\int \frac{\mathrm{d}x}{1+\sqrt{1-2x-x^2}}$;
	\end{tabular}
\end{table}
\item 求下列不定积分:
\begin{table}[H]
	\begin{tabular}{ll}
		(1)\ $\int \frac{1}{x}\sqrt{\frac{x+1}{x-1}}\mathrm{d}x$;\qquad \qquad \qquad \qquad &(2)\ $\int \frac{\mathrm{d}x}{\sqrt{x+1}+\sqrt[3]{x+1}}$;\\
		(3)\ $\int \sqrt{x^2+\frac{1}{x^2}}\mathrm{d}x$;\qquad \qquad \qquad \qquad &(4)\ $\int \frac{\mathrm{d}x}{x+\sqrt{x^2-x+1}}$.
	\end{tabular}
\end{table}
\item 问下列积分是否可积(即原函数是否为初等函数):
\begin{table}[H]
	\begin{tabular}{ll}
		(1)\ $\int \frac{x\mathrm{d}x}{\sqrt{1+\sqrt[3]{x^2}}}$;\qquad \qquad \qquad \qquad &(2)\ $\int \sqrt{\mathrm{cos}x}\mathrm{d}x$.
	\end{tabular}
\end{table}
\end{enumerate}
\section{定积分概念、可积条件与定积分性质}
\begin{enumerate}
	\item 设$f(x)\in R[a,b]$, 且$f(a)\ge a>0.$求证:\\
	(1)\  $\frac{1}{f(x)}\in R[a,b]$; \qquad \qquad \qquad (2)\ $\mathrm{ln}f(x)\in R[a,b]$.
	\item 求证: $\displaystyle{\lim\limits_{n\rightarrow \infty}\int_{0}^{1}\frac{x^n}{\sqrt{1+x^4}}\mathrm{d}x=0}$.
	\item 设$f(x)\in R[0,1]$, 且$f(x)\ge a>0$.求证: $\displaystyle{\int_{0}^{1}\frac{1}{f(x)}\mathrm{d}x}\ge \frac{1}{\displaystyle{\int_{0}^{1}f(x)\mathrm{d}x}}$.
	\item 求证: $\displaystyle{\lim\limits_{n\rightarrow \infty}\int_{0}^{1}(1-x^2)^n\mathrm{d}x=0}$.
	\item 设$a,b>0,f(x)\ge 0$, 且$f(x)\in R[a,b]$,又$\displaystyle{\int_{-a}^{b}xf(x)\mathrm{d}x}=0$.求证:
	$$\displaystyle{\int_{-a}^{b}x^2f(x)\mathrm{d}x\le ab\int_{-a}^{b}f(x)\mathrm{d}x}.$$
	\item 设$f(x)\ge 0,f''(x)\le 0(\forall x\in [a,b])$.求证:
	$$\underset{a\le x \le b}{\mathrm{max}}f(x)\le \displaystyle{\frac{2}{b-a}\int_{a}^{b}f(x)\mathrm{d}x}.$$
	\item 设$f(x)$在$[a,b]$上可导, 且$f'(x)\in R[a,b]$.求证:
	$$\underset{a\le x\le b}{\mathrm{max}}|f(x)|\le \displaystyle{|\frac{1}{b-a}\int_{a}^{b}f(x)\mathrm{d}x|+\int_{a}^{b}|f'(x)|\mathrm{d}x}.$$
	\item 设$f(x)\in R[a,b]$, 且$a\le f(x)\le b$, 又$\varphi(x)$是$[a,b]$上的凹函数.求证:
	\begin{enumerate}
		\item $\varphi(f(x))\ge \varphi(t)+\varphi'(t)(f(x)-t)(\forall t\in(a,b))$;
		\item $\displaystyle{\int_{0}^{1}\varphi(f(x))\mathrm{d}x\ge \varphi(\int_{0}^{1}f(x)\mathrm{d}x)}$.
		\item $\displaystyle {\int_{0}^{1}\mathrm{e}^{f(x)}\ge \mathrm{e}^{\int_{0}^{1}f(x)\mathrm{d}x}}$.
		
	\end{enumerate}
\item 求证: 极限$\displaystyle{\lim\limits_{b\rightarrow 1}\int_{0}^{b}\frac{\mathrm{sin}x}{\sqrt{1-x^2}}\mathrm{d}x}(0<b<1)$存在, 并且其极限值不超过1.
\item 求证: $\displaystyle{\int_{0}^{\frac{\pi}{2}}\mathrm{sin}(\mathrm{sin}x)\mathrm{d}x\le \int_{0}^{\frac{\pi}{2}}\mathrm{cos}(\mathrm{cos}x)}\mathrm{d}x$.
\end{enumerate}

\section{变限定积分、微积分基本原理、定积分的换元法}
\begin{enumerate}
	\item 设$f(x)=2x\mathrm{sin}\frac{1}{x}-\mathrm{cos}\frac{1}{x}(x\ne 0);f(0)=0$
	\begin{enumerate}
		\item 问$f(x)$是$[-1,1]$上可积?
		\item 问变上限积分$\displaystyle{\int_{-1}^{x}f(t)\mathrm{d}t}$在点$x=0$处是否可导?
	\end{enumerate}
	\item 求下列定积分:
	\begin{table}[H]
		\begin{tabular}{ll}
			(1)\ $\displaystyle{\int_{0}^{1}\frac{x}{(1+x)^\alpha}\mathrm{d}x}$;\qquad \qquad \qquad&(2)\ $\displaystyle{\int_{0}^{1}\mathrm{ln}(1+\sqrt{x})\mathrm{d}x}$;\\
			(3)\ $\displaystyle{\int_{0}^{\frac{a}{\sqrt{2}}}\frac{\mathrm{d}x}{(a^2-x^2)^{\frac{3}{2}}}}$;\qquad \qquad \qquad&(4)\ $\displaystyle{\int_{1}^{\sqrt{3}}\frac{\sqrt{1+x^2}}{x}\mathrm{d}x}$;\\
			(5)\ $\displaystyle{\int_{0}^{4}\frac{\sqrt{x}}{1+x}\mathrm{d}x}$;\qquad \qquad \qquad &(6)\
			 $\displaystyle{\int_{0}^{1}\mathrm{arctan}\sqrt{\frac{x}{1+x}}\mathrm{d}x}$;
		\end{tabular}
	\end{table}
	\item 求下列定积分:
		\begin{table}[H]
		\begin{tabular}{ll}
			(1)\ $\displaystyle{\int_{\frac{1}{2}}^{1}\mathrm{e}^{\sqrt{2x-1}}\mathrm{d}x}$;\qquad \qquad \qquad &(2)\ $\displaystyle{\int_{0}^{\mathrm{ln}2}\sqrt{1-e^{-2x}}\mathrm{d}x}$;
		\end{tabular}
	\end{table}
\end{enumerate}

\section{定积分的应用}

\section{广义积分}