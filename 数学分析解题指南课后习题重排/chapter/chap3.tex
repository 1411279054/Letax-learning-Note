\chapter{一元函数积分学}
\section{不定积分和可积函数类}
\centering{\textbf{练习题}}
\begin{enumerate}
\item 求下列不定积分:
\begin{table}[H]
\begin{tabular}{llll}
(1)\ $\int{\frac{e^{3x}+1}{e^x+1}}\mathrm{d}x$;\qquad \qquad &(2)\ $\int\frac{\mathrm{d}x}{x^2(1+x^2)}$;\qquad \qquad&(3)\ $\int \sqrt{x\sqrt{x}}\mathrm{d}x$;\qquad \qquad&(4)\ $\int [\sqrt{\frac{1+x}{1-x}}+\sqrt{\frac{1-x}{1+x}}]\mathrm{d}x$;\\
(5)\ $\int \mathrm{tan}^2x\mathrm{d}x$;\qquad \qquad &(6)\ $\int \frac{1+\mathrm{sin}^2x}{\mathrm{cos}^2x}\mathrm{d}x$;\qquad \qquad &(7)\ $\int \frac{\mathrm{dx}}{\mathrm{cos}^2x\mathrm{sin}^2x}$;\qquad \qquad &(8)\ $\int \frac{1+\mathrm{sin}^2x}{\mathrm{cos}^2x}\mathrm{d}x$;\\
(9)\ $\int \frac{\mathrm{d}x}{3x^2}$;\qquad \qquad &(10)\ $\int \frac{\mathrm{d}x}{2-3x^2}$;\qquad \qquad &(11)\ $\int \sqrt[3]{1-3x}\mathrm{d}x$;\qquad \qquad &(12)\ $\int x\cdot \sqrt[3]{1-3x}\mathrm{d}x$. 

\end{tabular}
\end{table}

\item 求下列不定积分$I=\int \frac{1}{1+x^4}\mathrm{d}x, J=\int \frac{x^2}{1+x^4}\mathrm{d}x$.
\item 求下列不定积分:
\begin{table}[H]
	\begin{tabular}{llll}
(1)\ $\int \frac{\mathrm{d}x}{\sqrt{3x^2-2}}$;\qquad \qquad & (2)\ $\int \frac{\mathrm{d}x}{x\sqrt{x^2+1}}$;\qquad \qquad &(3)\ $\int \sqrt{\frac{a+x}{a-x}}\mathrm{d}x(a>0)$;\qquad \qquad &(4)\ $\int \sqrt{\frac{x-a}{x+a}}(a\ge 0)$;\\
(5)\ $\int \frac{\mathrm{d}x}{\sqrt{1+x+x^2}}$;\qquad \qquad &(6)\ $\int \frac{x+3}{\sqrt{4x^2+4x+3}}\mathrm{d}x$;
\qquad \qquad &(7)\ $\int \frac{\mathrm{d}x}{\sqrt{(x+a)(x+b)}}\mathrm{d}x(a<b)$;\qquad \qquad &(8)\ $\int \sqrt{\frac{x}{1-x\sqrt{x}}}\mathrm{d}x$.

\end{tabular}
\end{table}
\item 求下列不定积分:
\begin{table}[H]
	\begin{tabular}{lll}
		(1)\ $\int \frac{\mathrm{d}x}{x^2\sqrt{x^2+1}}$;\qquad \qquad \qquad &(2)\ $\int \sqrt{\frac{x}{1-x\sqrt{x}}}\mathrm{d}x$;\qquad \qquad \qquad&(3)\ $\int \frac{\sqrt{a^2-x^2}}{x}\mathrm{d}x$;\\
		(4)\ $\int \frac{\sqrt{x^2-a^2}}{x}\mathrm{d}x$;\qquad \qquad \qquad&(5)\ $\int x^2\sqrt{4-x^2}\mathrm{d}x$;\qquad \qquad &(6)\ $\int \frac{x}{1+\sqrt{x}}\mathrm{d}x$.
	\end{tabular}
\end{table}
\item 求下列不定积分:
\begin{table}[H]
	\begin{tabular}{ll}
		(1)\ $\int \mathrm{ln}(1+x^2)\mathrm{d}x$;\qquad \qquad \qquad \qquad &(2)\ $\int x^\alpha\mathrm{ln}x\mathrm{d}x$;\\
		(3)\ $\int \sqrt{x}\mathrm{ln}^2x\mathrm{d}x$;\qquad \qquad \qquad \qquad &(4)\ $\int x^2e^{-2x}\mathrm{d}x$;\\
		(5)\ $\int x\mathrm{cos}\beta x\mathrm{d}x$;\qquad \qquad \qquad \qquad &(6)\ $\int x^2\mathrm{sin}2x\mathrm{d}x$;\\
		(7)\ $\int x\mathrm{arctan}x \mathrm{d}x$;\qquad \qquad \qquad \qquad &(8)\ $\int \frac{\mathrm{arcsin}x}{x^2}\mathrm{d}x$;\\
		(9)\ $\int \frac{x}{\mathrm{cos}^2x}$;\qquad \qquad \qquad \qquad &(10)\ $\int \frac{x^2e^x}{(x+2)^2}\mathrm{d}x$.
	\end{tabular}
\end{table}
\item 求下列不定积分:
\begin{table}[H]
	\begin{tabular}{ll}
(1)\ $\int \frac{1+x+x^2}{\sqrt{1+x^2}}\mathrm{e}^x\mathrm{d}x$;\qquad \qquad \qquad \qquad &(2)\ $\int \frac{1+\mathrm{tan}x}{\mathrm{cos}x}\mathrm{e}^x\mathrm{d}x$.\\
(3)\ $\int (\mathrm{cos}x-\mathrm{sin}x)\mathrm{e}^{-x}\mathrm{d}x$;\qquad \qquad \qquad \qquad &(4)\ $\int x(2-x)\mathrm{e}^{-x}\mathrm{d}x$.
	\end{tabular}
\end{table}
\item 求下列不定积分:
\begin{table}[H]
	\begin{tabular}{ll}
		(1)\ $\int \mathrm{sin}(\mathrm{ln}x)\mathrm{d}x$; \qquad \qquad \qquad \qquad & (2)\ $\int \mathrm{cos}(\mathrm{ln}x)$;\\
		(3)\ $\int x\mathrm{e}^x\mathrm{cos}x\mathrm{d}x$;\qquad \qquad \qquad \qquad &(4)\ $\int x\mathrm{e}^x\mathrm{sin}x\mathrm{d}x$;
	\end{tabular}
\end{table}
\item 求下列不定积分的递推公式:\\

\end{enumerate}
\section{定积分概念、可积条件与定积分性质}

\section{变限定积分、微积分基本原理、定积分的换元法}

\section{定积分的应用}

\section{广义积分}