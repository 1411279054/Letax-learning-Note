\chapter{级数}
\section{级数敛散判别法与性质、上极限与下极限}
\centering{\textbf{练习题}}
\begin{enumerate}
\item 求下列级数的和:
\begin{table}[H]
	\begin{tabular}{ll}
		(1)\ $\sum\limits_{n=1}^{\infty}\frac{1}{(3n-2)(3n+1)}$;\qquad \qquad \qquad\qquad	\qquad&(2)\ $\sum\limits_{n=1}^{\infty}\frac{(-1)^{n-1}}{n(n+2)}$.\\
	\end{tabular}
\end{table}
\item 判断下列级数的收敛性:
\begin{table}[H]
	\begin{tabular}{ll}
		(1)\ $\sum\limits_{n=1}^{\infty}\frac{n^2}{(1+\frac{1}{n})^{n^2}}$;\qquad \qquad \qquad \qquad \qquad&(2)\ $\sum\limits_{n=1}^{\infty}\frac{\mathrm{ln}n}{2^n}$;\\
		(3)\ $\sum\limits_{n=1}^{\infty}\frac{n^{n-1}}{(2n^2+1)^\frac{n}{2}}$;\qquad \qquad \qquad\qquad	\qquad&(4)\ $\sum\limits_{n=1}^{\infty}\frac{a^n}{n!}$;\\
		(5)\ $\sum\limits_{n=1}^{\infty}\frac{\sqrt{n!}}{n^\frac{n}{2}}$;\qquad \qquad \qquad\qquad	\qquad&(6)\ $\sum\limits_{n=1}^{\infty}\frac{\sqrt{n!}\cdot 2^n}{n^\frac{n}{2}}$;\\
	(7)\ $\frac{3}{1}+\frac{3\cdot5}{1\cdot 4}+\frac{3\cdot5\cdot 7}{1\cdot 4\cdot 7}+\frac{3\cdot5\cdot7\cdot9}{1\cdot 4\cdot7\cdot 10}+\cdots$.
	\end{tabular}
\end{table}
\item 判断下列级数的收敛性:
\begin{table}[H]
	\begin{tabular}{ll}
		(1)\ $\sum\limits_{n=1}^{\infty}\frac{1}{2^{\mathrm{ln}n}}$;\qquad \qquad \qquad\qquad	\qquad&(2)\ $\sum\limits_{n=1}^{\infty}\frac{n}{3^{\sqrt{n}}}$;\\
		(3)\ $\sum\limits_{n=1}^{\infty}\frac{n}{3^{\sqrt{n}}}$;\qquad \qquad \qquad\qquad	\qquad&(4)\ $\sum\limits_{n=2}^{\infty}\frac{\mathrm{ln}n}{n^p}(p>1)$.
\end{tabular}
\end{table}
\item 判断下列级数的收敛性:
\begin{table}[H]
	\begin{tabular}{ll}
		(1)\ $\sum\limits_{n=1}^{\infty}2^n\mathrm{sin}\frac{\pi}{3^n}$;\qquad \qquad \qquad\qquad	\qquad&(2)\ $\lim\limits_{n=1}^{\infty}\frac{1}{\sqrt{n^3+n+1}}$;\\(3)\ $\sum\limits_{n=1}^{\infty}[(n+\frac{1}{2})\mathrm{ln}(1+\frac{1}{n})-1]$\qquad \qquad \qquad\qquad	\qquad&
		(4)\ $\sum\limits_{n=2}^{\infty}\mathrm{ln}\sqrt[n]{\frac{n+1}{n-1}}$;\\
		(5)\ $\sum\limits_{n=1}^{\infty}(\sqrt{n+a}-\sqrt[4]{n^2+n})(a>0)$;\qquad \qquad \qquad\qquad	\qquad&(6)\ $\sum\limits_{n=1}^{\infty}(\sqrt[n]{n}-1)^p(p>0)$.
	\end{tabular}
\end{table}
\item 若级数$\sum\limits_{n=1}^{\infty}a_n(A)$及$\sum\limits_{n=1}^{\infty}b_n(B)$皆收敛, 且$a=n\le c_n\le b_n\ (n=1,2,3,\cdots)$.试证级数$\sum\limits_{n=1}^{\infty}c_n(C)$收敛; 若级数$(A),(B)$皆收敛, 问级数$(C)$的收敛性如何?
\item 若正向级数$\sum\limits_{n=1}^{\infty}a_n$收敛,${a_n}$单调递减.求证:
\begin{table}[H]
	\begin{tabular}{ll}
		(1)\ $\lim\limits_{n\rightarrow \infty}\sum\limits_{k=[\frac{n}{2}]+1}{n}a_k=0$;\qquad \qquad \qquad&(2)\ $\lim\limits_{n\rightarrow \infty}na_n=0$.
	\end{tabular}
\end{table}
\item 若正向级数$\sum\limits_{n=1}^{\infty}a_n$的项$a_n$单调递减, 且$\sum\lim\limits_{n=1}^{\infty}a_{2n}$收敛, 求证:$\sum\limits_{n=1}^{\infty}a_n$收敛.
\item 设$0<p_1<p_2<\cdots<p_n<\cdots$.求证:级数$\sum\limits_{n=1}^{\infty}\frac{1}{p_n}$收敛的充分必要条件为下面的级数收敛:
$$\sum\limits_{n=1}^{\infty}\frac{n}{p_1+p_2+\cdots+p_n}.$$
\item 判断下列级数的收敛性:
\begin{table}[H]
	\begin{tabular}{ll}
		(1)\ $\sum\limits_{n=1}^{\infty}(-1)^{n-1}\frac{\mathrm{ln}^2n}{n}$;\qquad \qquad \qquad \qquad&(2)\ $\sum\limits_{n=1}^{\infty}(-1)^{n-1}\frac{\sqrt{n}}{n+100}$;\\
		(3)\ $\sum\limits_{n=1}^{\infty}(-1)^{n-1}\frac{1+\frac{1}{2}+\cdots+\frac{1}{n}}{n}$;\qquad \qquad\qquad \qquad&(4)\ $\sum\limits_{n=1}^{\infty}\mathrm{sin}(\pi\sqrt{n^2+1})$.
			\end{tabular}
	\end{table}
\item 判断下列级数的收敛性:
\begin{table}[H]
	\begin{tabular}{ll}
(1)\ $\sum\limits_{n=1}^{\infty}(-1)^n\frac{\mathrm{sin}^n}{n}$;\qquad \qquad \qquad \qquad &(2)\ $\sum\limits_{n=1}^{\infty}\frac{(-1)^{n-1}}{n}\cdot \frac{a^n}{1+a^n}(a>0)$;\\
(3)\ $\sum\limits_{n=1}^{\infty}\frac{\mathrm{sin}n\cdot\mathrm{sin}n^2}{n}$;\qquad \qquad \qquad \qquad &(4)\ $\sum\limits_{n=1}^{\infty}\frac{\mathrm{sin}(n+\frac{1}{n})}{n}$.
	\end{tabular}
\end{table}
\item 讨论下列级数的收敛性:
\begin{table}[H]
	\begin{tabular}{ll}
		(1)\ $\sum\limits_{n=1}^{\infty}\frac{x^n}{1+x^{2n}}$;\qquad \qquad \qquad \qquad &(2)\ $\sum\limits_{n=1}^{\infty}\frac{(-1)^{n-1}}{n^2}\cdot e^{-nx}$.
	\end{tabular}
\end{table}
\item 讨论下列级数的收敛性于绝对收敛性$(p>0)$:
\begin{table}[H]
	\begin{tabular}{ll}
		(1)\ $\sum\limits_{n=1}^{\infty}\frac{(-1)^{n-1}}{n^p+(-1)^{n-1}}$;\qquad \qquad \qquad \qquad &(2)\ $\sum\limits_{n=1}^{\infty}\frac{(-1)^{n-1}}{(n+(-1)^{n-1})^p}$.
	\end{tabular}
\end{table}
\item 求证:若级数$\sum\limits_{n=1}^{\infty}a_n^2$及$\sum\limits_{n=1}^{\infty}b_n^2$收敛, 则下列级数:
$$ \sum\limits_{n=1}^{\infty}a_n \cdot b_n, \sum\limits_{n=1}^{\infty}(a_n+b_n)^2,\sum\limits_{n=1}^{\infty}\frac{a_n}{n}$$
皆收敛.
\item 设$\sum\limits_{n=1}^{\infty}a_n$收敛, 且$\lim\limits_{n\rightarrow \infty}na_n=0$.求证: $\sum\limits_{n=1}^{\infty}n(a_n-a_{n+1})$收敛, 并且$$
\sum\limits_{n=1}^{\infty}n(a_n-a_{n+1})=\sum\limits_{n=1}^{\infty}a_n$$
\item 
\begin{enumerate}
	\item 设正项数列$\{x_n\}$单调上升. 求证:当${x_n}$有界时,级数$$
	\sum\limits_{n=1}^{\infty}(1-\frac{x_n}{x_{n+1}})$$
	收敛, 当$\{x_n\}$无界时, 该级数发散.
	\item 设$\alpha\ge 1,a_1=1,a_{n+1}=\frac{n}{n+\alpha}(n=1,2,3,\cdots)$.求证:$(n^\alpha a_n)$是收敛数列.
\end{enumerate}
\item 求证: 级数$\sum\limits_{n=1}^{\infty}\frac{(-1)^{n-1}}{\sqrt{n}}$的平方(指柯西乘积)是发散的.
\item 求证: 级数$\sum\limits_{n=1}^{\infty}\frac{(-1)^{n-1}}{n}$的平方(指柯西乘积)是收敛的.
\item 用级数方法证序列$x_n=1+\frac{1}{\sqrt{2}+\cdots+\frac{1}{\sqrt{n}}-2\sqrt{n}}$的极限存在$(n\rightarrow +\infty)$.
\item 设$p_n>0(n=1,2,\cdots)$, 若级数$\sum\limits_{n=1}^{\infty}\frac{1}{p_n}$收敛, 求证级数
$$ \sum\limits_{n=1}^{\infty}\frac{n}{p_1+p_2+\cdots+p_n}\textbf{收敛}$$
\item 设${a_n},{b_n}$满足关系式$a_{n+1}=b_n-qa_n(0<q<1)$, 且$\lim\limits_{n\rightarrow \infty}b_n=b$存在, 证明$\lim\limits_{n\rightarrow \infty}a_n$存在.
\item 设$x_1>0,x_{n+1}=1+\frac{1}{x_n}(n=1,2,\cdots)$.求证:
\begin{enumerate}
	\item $1\le \underset{n\rightarrow \infty}{\underline{\lim}}x_n\le \overline{\lim\limits_{n\rightarrow \infty}}x_n\le 2$
	\item $\lim\limits_{n\rightarrow \infty}x_n$存在, 并求其极限值.
\end{enumerate}
\item 设序列$\{a_n\}$有界, 并满足$\lim\limits_{n\rightarrow \infty}(a_{2n}+2a_n)=0$, 求证: $\lim\limits_{n\rightarrow \infty}a_n=0$.
\item 求证$\overline{\lim\limits_{n\rightarrow \infty}}\le 1(S_n\ge 0)$的充要条件为: 对任一大于1的数为$l$, 有$\lim\limits_{n\rightarrow \infty}\frac{S_n}{l^n}=0$.
\item 设$0\le x_{n+m}\le x_n \cdot x_m\ (x,m\in N)$.求证:序列$\{\sqrt[n]{x_n}\}$极限存在.
\item 
\begin{enumerate}
	\item 设$0<q<l$, 求证:$\exists r\in (q,1)$, 使$n$充分打时, 有$$
	1+\frac{q}{n}<(1+\frac{1}{n})^r\ (n>N);
	$$
	\item 设$a_n>0$, 求证:$\overline{\lim\limits_{n\rightarrow \infty}}n(\frac{1+a_{n+1}}{a_n})-1\ge 1$.
\end{enumerate}
\end{enumerate}

\section{函数级数}
\centering{\textbf{练习题}}
\begin{enumerate}
	\item 讨论下列函数序列在指定区间上的一致收敛性:
	\begin{enumerate}
		\item $f_n(x)=\frac{x^n}{1+x^n}, i)\ 0\le x\le b<1;ii)\ 0\le x\le 1;iii)\ 1<a\le x<+\infty$.
		\item $f_n(x)=\frac{1}{n}\mathrm{ln}(1+e^{-nx}),\ i)\ x\le 0;\ ii)\ x<0$.
	\end{enumerate}
\item 设$f(x)$在$(A,B)$内有连续导数$f'(x)$, 且$$
f_n(x)\overset{\text{记为}}{=}n[f(x+\frac{1}{n}-f(x))].$$
求证: 当$n\rightarrow \infty$时, $f_n(x)$在闭区间$[a,b]\subset(A,B)$上一致收敛于$f'(x)$.
\item 求证下列级数在所示区间上的一致收敛性:
\begin{enumerate}
	\item $\sum\limits_{n=2}^{\infty}\frac{(-1)^n}{n+\mathrm{sin}x}\ (|x|<+\infty)$;
	\item $\sum\limits_{n=1}^{\infty}(-1)^n\frac{\mathrm{sin}nx}{n}\ \ (-\pi+\delta\le x\le \pi-\delta,\delta>0)$;
	\item $\sum\limits_{n=0}^{\infty}\frac{(-1)^n}{a+n}x^{n+a}\ \ (0<a<1,0\le x\le 1)$;
	\item $\sum\limits_{n=1}^{\infty}\frac{(-1)^\frac{n(n-1)}{2}}{\sqrt{x^2+n^2}}\ \ (|x|<+\infty)$.
\end{enumerate}
\item 设$u_n(x)$在$[a,b]$上连续而且非负, $\lim\limits_{n=1}^{\infty}u(x)$收敛, 且和函数$S(x)=\sum\limits_{n=1}^{\infty}u_n(x)$在$[a.b]$上连续, 求证: $\sum\limits_{n=1}^{\infty}u_n(x)$在$[a,b]$上一致收敛.
\item 求证: 级数$\lim\limits_{n=0}^{\infty}x^N\mathrm{sin}^2\pi x$在$[0,1]$上一致收敛.
\item 给定序列$f_n(x)=nxe^{-nx^2} (n=1,2,\cdots)$. 求证:
\begin{enumerate}
	\item $\displaystyle{\int_{0}^{1}[\lim\limits_{n\rightarrow \infty}]\mathrm{d}x\ne \lim\limits_{n\rightarrow \infty}\int_{0}^{1}f_n(x)\mathrm{d}x}$;
	\item 序列$f_n(x)$在$[0,1]$上不一致收敛.
\end{enumerate}
\item 求证下列级数在$[0,1]$上不一致收敛:
\begin{table}[H]
	\begin{tabular}{ll}
		(1)\ $\sum\limits_{n=0}^{\infty}x^n\mathrm{ln}x$;\qquad \qquad \qquad \qquad \qquad &(2)\ $\sum\limits_{n=1}^{\infty}\frac{x^2}{(1+x^2)^n}$.
	\end{tabular}
\end{table}
\item 求证: 级数$\int\limits_{n=1}^{\infty}\frac{\mathrm{cos}nx}{n}$在$(0,2\pi)$上不一致收敛.
\item 设$f_n(x)\ (n=1,2,\cdots)$在$(-\infty,+\infty)$上一致收敛, 且$$
f_n(x)\xrightarrow[\text{一致}]{R}f(x)\ \ (n\rightarrow \infty),$$
求证: $f(x)$在$(-\infty,+\infty)$上一致收敛.
\item 设$f_n(x)\ (n=1,2,\cdots)$在$[a,b]$上连续, 且$n\rightarrow \infty$时,$$
f_n(x)\xrightarrow[\text{一致}]{[a,b]}f(x).$$
又设$f(x)$在$[a,b]$上无零点, 求证:
\begin{enumerate}
	\item 当$n$充分大时, $f_n(x)$在$[a,b]$上也无零点;
	\item $\frac{1}{f_n(x)}\xrightarrow[\text{一致}]{[a,b]}\frac{1}{f(x)}\ \ (n\rightarrow +\infty)$
\end{enumerate}
	\item 设$f_n(x)\in C[a,b]\ (n=1,2,\cdots),f_n(x)\xrightarrow[\text{一致}]{[a.b]}f(x)$. 求证:
	\begin{enumerate}
		\item $\exists M$, 使$|f_n(x)|\le M$, $|f(x)|\le M\ (a\le x\le b,n=1,2,\cdots)$;
		\item 若$g(x)$在$(-\infty,+\infty)$上连续, 则$g(f_n(x))\xrightarrow[\text{一致}]{[a,b]}g(f(x))$.
	\end{enumerate}
\item 设$f(x) = \sum\limits_{n=1}^{\infty}\frac{(-1)^{n-1}}{n}e^{-nx}$, 求证:
\begin{table}[H]
	\begin{tabular}{ll}
		(1)\ $f(x)$在$x\ge 0$上连续; \qquad \qquad \qquad \qquad&(2)\ $f(x)$在$x>0$上连续可微.
	\end{tabular}
\end{table}
\item 求证:
\begin{table}[H]
	\begin{tabular}{ll}
		(1)\ $\sum\limits_{n=0}^{\infty}x^n\mathrm{ln}^2x$在$[0,1]$上一致连续;\qquad \qquad \qquad \qquad \qquad &(2)\ $\displaystyle{\int_{0}^{1}\frac{\mathrm{ln}^2x}{1-x}\mathrm{d}x=\sum_{n=1}^{\infty}\frac{2}{n^3}}$.
		\end{tabular}
\end{table}
\item 求证:
\begin{table}[H]
	\begin{tabular}{ll}
		(1)\ $\displaystyle{\sum\limits_{n=0}^{\infty}\int_{0}^{x}t^n\mathrm{ln}t\mathrm{d}t}$在$[0,1]$上一致收敛;\qquad \qquad \qquad \qquad \qquad & (2)\ $\displaystyle{\int_{0}^{1}\frac{\mathrm{ln}x}{1-x}\mathrm{d}x=-\sum\limits_{n=1}^{\infty}\frac{1}{n^2}}$.
	\end{tabular}
\end{table}
\item 设函数$f(x)$在$(-a,a)$上无穷多次可微, 且序列$f^n(x)$在$(-a,a)$上一致收敛到函数$\varphi(x)=Ce^x\ \ (C\text{为常数})$.
\item 求证: 函数$$
f(x)=\sum\limits_{n=1}^{\infty}\frac{|x-\frac{1}{n}|}{2^n}$$
在$(0,1)$上连续, 除点$x_k=\frac{1}{k}\ (k=2,3,\cdots)$处不可微.
\item 设$x_n$是$(0,1)$内一个序列, 即$0<x_n<1$且$x_i\ne x_j\ (i\ne j)$. 求证: 函数$f(x)=\sum\limits_{n=1}^{\infty}\frac{\mathrm{sgn}(x-x_n)}{2^n}$在$(0,1)$中除点$x_n(n=1,2,\cdots)$处不连续外皆连续.
\end{enumerate}

\section{幂函数}
\centering{\text{练习题}}
\begin{enumerate}
	\item 求下列幂函数的收敛半径, 并讨论收敛区间端点的收敛性:
	\begin{table}[H]
		\begin{tabular}{ll}
			(1)\ $\sum\limits_{n=1}^{\infty}\frac{1+\frac{1}{2}+\cdots+\frac{1}{n}}{n}x^n$;\qquad \qquad \qquad \qquad \qquad &(2)\ $\sum\limits_{n=1}^{\infty}\frac{(2n)!!}{(2n+1)!!}x^n$;\\
			(3)\ $\sum\limits_{n=1}^{\infty}(1+\frac{1}{n}^{n^2})x^{2n}$;\qquad \qquad \qquad \qquad \qquad&(4)\ $\sum\limits_{n=1}{\infty}\frac{2^n+3^n}{n}x^n$;\\
			(5)\ $\sum\limits_{n=0}^{\infty}(1+2\mathrm{cos}\frac{n\pi}{4})^nx^n$.
		\end{tabular}
	\end{table}
\item 求下列级数的收敛性:
\begin{table}[H]
	\begin{tabular}{ll}
		(1)\ $\sum\limits_{n=1}^{\infty}\frac{(-1)^{n-1}}{2n-1}(\frac{1-x}{1+x})^n$;\qquad \qquad \qquad \qquad \qquad &(2)\ $\sum\limits_{n=0}^{\infty}[x(1+x)]^{3^n}$.
	\end{tabular}
\end{table}
\item 给定零阶贝塞尔函数:
$$ y = J_0(x) = 1+\sum\limits_{n=1}^{\infty}(-1)^n\frac{x^{2n}}{(n!)^22^{2n}},$$
求证: 它在实轴上满足方程:$$
xy''+y'+xy=0.$$
\item 求下列级数的和:
\begin{table}[H]
	\begin{tabular}{ll}
	(1)\ $\sum\limits_{n=1}^{\infty}\frac{n+1}{n!2^n}x^n$;\qquad \qquad \qquad \qquad \qquad &(2)\ $\sum\limits_{n=0}^{\infty}\frac{x^{4n+1}}{4n+1}$;\\
	(3)$\sum\limits_{n=1}^{\infty}n^2x^{n-1}$.
	\end{tabular}
\end{table}
\item 求下列级数的和:
	\begin{table}[H]
		\begin{tabular}{ll}
			(1)\ $\sum\limits_{n=1}^{\infty}\frac{2n-1}{2^n}$;\qquad \qquad \qquad \qquad \qquad &(2)\ $\sum\limits_{n=1}^{\infty}\frac{1}{n(2n+1)}$;\\
			(3)\ $\sum\limits_{n=1}^{\infty}\frac{(-1)^{n-1}}{n(2n+1)}$.
	\end{tabular}
	\end{table}
\item 设$0<a<1$, 求证:
\begin{enumerate}
\item $\displaystyle{\int_{0}^{b}\frac{x^{\alpha-1}}{1+x}\mathrm{d}x=\sum\limits_{n=0}^{\infty}\frac{(-1)^n}{n+a}b^{n+a}}\ (0\le b<1)$;
\item 级数$\sum\limits_{n=0}^{\infty}\frac{(-1)^n}{n+a}b^{n+a}$对在$[0,1]$上一致收敛;
\item $\displaystyle{\int_{0}^{1}\frac{x^{a-1}}{1+x}=\sum\limits_{n=0}^{\infty}\frac{(-1)^n}{n+a}}$.
\end{enumerate}
\item 已知零阶贝塞尔函数:
	$$ J_0(x) \overset{\text{定义}}{=}\frac{2}{\pi}\int_{0}^{\frac{\pi}{2}}\mathrm{cos}(x\mathrm{sin}\theta)\mathrm{d}\theta,$$
	求证: $J_0(x)=\int\limits_{n=0}^{\infty}(-1)^n\frac{x^{2n}}{(n!)^22^{2n}}$.
	\item 设对$\forall k \in N, |f^{(k)}(x)|\le M^k\ (|x|<a)$, 其中$M$为与$k$和$x$都无关的常数.求证:
	\begin{enumerate}
		\item $f(x)$可以在$(-a,a)$上展开成幂级数;
		\item $f(x)$可以开拓到$(-\infty,+\infty)$, 且在$(-\infty,+\infty)$上无穷多次可微.
	\end{enumerate}
\item 把下列函数在$x=0$点展开成幂级数:
\begin{table}[H]
	\begin{tabular}{ll}
		(1)\ $\frac{x}{(1-x)(1-x^2)}$;\qquad \qquad \qquad \qquad \qquad &(2)\ $\frac{x}{\sqrt{1-x}}$;\\
		(3)\ $\mathrm{cos}^2x$;\qquad \qquad \qquad \qquad \qquad &(4)\ $\mathrm{ln}(1+x+x^2)$;\\
		(5)\ $\mathrm{ln}(1+x+x^2+x^3)$\qquad \qquad \qquad \qquad \qquad &(5)\ $\mathrm{\ln}\frac{1+x}{1-x}$.
	\end{tabular}
\end{table}
\item 求证下列展开式成立:
\begin{enumerate}
	\item $\mathrm{ln}(x+\sqrt{1+x^2})=x+\sum\limits_{n=1}^{\infty}(-1)^n\frac{(2n-1)!!}{(2n)!!}\cdot\frac{x^{2n+1}}{2n+1}\ (|x|\le 1)$;
	\item $\mathrm{artan}\frac{2x}{2-x^2}=\sum\limits_{n=0}^{\infty}(-1)^{[\frac{n}{2}]}\frac{x^{2n+1}}{2^n(2n+1)}\ (|x|\le \sqrt{2})$.
\end{enumerate}
\item 
\begin{enumerate}
	\item 将$(\mathrm{arctan}x)^2$在$x=0$点展开为幂级数;
	\item 求级数$\sum\limits_{n=0}^{\infty}\frac{(-1)^n}{n+1}(1+\frac{1}{3}+\cdots+\frac{1}{2n+1})$的和.
\end{enumerate}
\item 求下列幂级数的收敛半径, 并讨论收敛区间端点的收敛性;
\begin{table}[H]
	\begin{tabular}{ll}
		(1)\ $\sum\limits_{n=0}^{\infty}\frac{x^n}{\sqrt[n]{n!}}$;\qquad \qquad \qquad \qquad \qquad &(2)\ $\sum\limits_{n=1}^{\infty}\frac{n^nx^n}{n!}$.
	\end{tabular}
\end{table}
\item 
\begin{enumerate}
	\item 求证: 函数$y=\mathrm{arcsin}x/\sqrt{1-x^2}$满足方程$$
	(1-x^2)y'-xy=1,$$
	并由此求出$y^{(n)}(0)/n!$;
	\item 求证:$\frac{\mathrm{sin}^{-1}x}{\sqrt{1-x^2}}=\sum\limits_{n=0}^{\infty}\frac{(2n)!!}{(2n+1)!!}x^{2n+1}\ \ (|x|<1)$;
	\item 求证: $(\mathrm{arctan}x)^2=\sum\limits_{n=0}^{\infty}\frac{(2n)!!}{(2n+1)!!}\cdot\frac{x^{2n+2}}{n+1}\ (|x|\le 1)$.
\end{enumerate}
\item 设$0<\theta<2\pi$, 利用幂级数的乘法求证:
\begin{enumerate}
	\item $\frac{\mathrm{cos}\theta-x}{1-2x\mathrm{cos}\theta+x^2}=\sum\limits_{n=1}^{\infty}\mathrm{cos}n\theta x^{n-1}\ \ (|x|<1)$.
\end{enumerate}
\item 求下列函数的幂级数展开式:
\begin{table}[H]
	\begin{tabular}{ll}
		(1)\ $\mathrm{arctan}\frac{x\mathrm{sin}\theta}{1-x\mathrm{cos}\theta}$;\qquad \qquad \qquad \qquad \qquad&(2)\ $-\frac{1}{2}\mathrm{ln}(1-2x\mathrm{cos}\theta+x^2)$. 	
	\end{tabular}
\end{table}
\item 设$0<\theta<2\pi$, 求证:
\begin{table}[H]
	\begin{tabular}{ll}
		(1)\ $\sum\limits_{n=1}^{\infty}\frac{\mathrm{sin}n\theta}{n}=\frac{\pi - \theta}{2}$;\qquad \qquad \qquad \qquad \qquad &(2)\ $\sum\limits_{n=1}^{\infty}\frac{\mathrm{cos}n\theta}{n}=-\mathrm{ln}2\mathrm{sin}\frac{\theta}{2}$;\\
		(3)\ 级数$\sum\limits_{n=1}^{\infty}\frac{\mathrm{sin}n\theta}{n}$在$(0,2\pi)$上不一致收敛.
	\end{tabular}
\end{table}
\item 设$f(x)=\sum\limits_{n=0}^{\infty}a_nx^n$收敛半径为1, 求
$$
F(x) \overset{\text{定义}}{=}\frac{f(x)}{1-x}$$
的幂级数展开式, 并求出它的收敛半径.
\item 设$A=\sum\limits_{n=0}^{\infty}a_n,B=\sum\limits_{n=0}^{\infty}b_n$, 又已知这两个级数的柯西乘积产生的级数
$$
\sum\limits_{n=0}^{\infty}(a_0b_n+a_1b_n+\cdots+a_nb_0)$$
收敛. 求证: 乘积级数的积等于$A\cdot B$.
\end{enumerate}

\section{傅氏级数的收敛性、平均收敛与一致收敛}
\centering{\text{练习题}}
\begin{enumerate}
	\item 
	\begin{enumerate}
		\item $\{\mathrm{cos}nx\}_{n=0}^{\infty}$是$[0,\pi]$上的正交系;
		\item $\{\mathrm{sin}nx\}_{n=1}^{\infty}$是$[0,\pi]$上的正交系;
		\item$l,\mathrm{cos}\frac{\pi x}{l},\mathrm{sin}\frac{\pi x}{l},\cdots,\mathrm{cos}\frac{n\pi x}{l},\mathrm{sin}\frac{n\pi x}{l},\mathrm{sin}\frac{n\pi x}{l},\cdots$是$[-l,l]$上的正交系.
	\end{enumerate}
\item 将下列函数展开成傅氏级数:
\begin{enumerate}
	\item $f(x)=\mathrm{sin}^4x\ \ (-\pi<le x\le \pi)$;
	\item $f(x)=\mathrm{sec}x\ \ (-\pi\le <x \le \pi)$;
	\item $f(x)=\mathrm{sin}\frac{x}{2}\ \ (-\pi<x<\pi)$;
	\item $f(x)=|\mathrm{sin}x|\ \ (-\pi\le x\le \pi)$.
\end{enumerate}
\item 将$f(x)=|x|(-\pi \le x\le \pi)$展开成傅氏级数, 并求下列级数的和:
\begin{table}[H]
	\begin{tabular}{lll}
		\qquad (1)\ $\sum\limits_{n=1}^{\infty}\frac{1}{(2n-1)^2}$;\qquad \qquad &(2)\ $\sum\limits_{n=1}^{\infty}\frac{1}{n^2}$;\qquad \qquad &(3)\ $\sum\limits_{n=1}^{\infty	}\frac{(-1)^{n-1}}{n^2}$.
	\end{tabular}
\end{table}
\item 将$f(x)=e^x(-\pi\le x\le \pi)$展开成傅氏级数, 并求级数$\sum\limits_{n=1}^{\infty}\frac{1}{1+n^2}$的和.
\item 将$f(x)=\begin{cases}
	1,\quad& 0<a<1,\\
	2,\quad &h\le x\le \pi
	\end{cases}$\\
	(1)\ 按余弦展开;	\qquad \qquad \qquad \qquad(2)\ 按正弦展开.
\item 求证:
\begin{enumerate}
	\item $\frac{\pi}{\mathrm{tan}a\pi}=\frac{1}{a}+\sum\limits_{n=1}^{\infty}\ (0<a<1)$
	\item $\frac{1}{\mathrm{tan}x}=\frac{1}{x}+\sum\limits_{n=1}^{\infty}\frac{2x}{x^2-n^2\pi^2}\ (0<x<\pi)$;
	\item $\frac{1}{\mathrm{sin}^2}=\frac{1}{x^2}+\sum\limits_{n=1}^{\infty}[\frac{1}{(x-n\pi)^2}+\frac{1}{(x+n\pi)^2}]\ (0<x<\pi)$.
\end{enumerate}
\item 求证:
\begin{enumerate}
	\item $\displaystyle{\int_{-\pi}^{\pi}|\mathrm{cos}nx|}\mathrm{d}x=4\ (n\in N)$;
	\item 若$\forall a\in N$, 设$T_n(x)$是任意的$n$阶三角矩阵, 其中$\mathrm{cos}nx$的系数为1, 则$$
	\underset{|x|\le \pi}{\mathrm{max}|T_n(x)|}\ge \frac{\pi}{4}.$$
\end{enumerate}
\item 求证:
\begin{enumerate}
	\item $\displaystyle{\lim\limits_{n\rightarrow \infty}\int_{0}^{\pi}(\frac{1}{2\mathrm{sin}\frac{x}{2}-\frac{1}{x}})\mathrm{sin}(n+\frac{1}{2})x\mathrm{d}x}$.
	\item $\displaystyle{\int_{0}^{\infty}\frac{\mathrm{sin}x}{x}\mathrm{d}x}$.
\end{enumerate}
\item 设$f(x)\in C[0,T], g(x)$是周期为$T$的连续周期函数, 求证:
$$\displaystyle{\lim\limits_{n\rightarrow \infty}\int_{0}^{T}f(x)g(nx)\mathrm{d}x=\frac{1}{T}f(x)\mathrm{d}x\cdot \int_{0}^{T}g(x)\mathrm{d}x}$$.
\item 设$0<a<1$, 求证:
\begin{table}[H]
	\begin{tabular}{ll}
	(1)\ $\displaystyle{\lim\limits_{b\rightarrow 1}\int_{0}^{b}\frac{x^{a-1}}{1+x}\mathrm{d}x=\sum\limits_{n=0}^{\infty}\frac{(-1)^n}{a+n}}$;\qquad \qquad \qquad \qquad \qquad &(2)\ $\displaystyle{\lim\limits_{b\rightarrow 1}\int_{0}^{b}\frac{x^{-a}}{1+x}}\mathrm{d}x=\sum\limits_{n=1}^{\infty}\frac{(-1)^n}{a-n}$;\\
(3)\ $\displaystyle{\int\limits_{0}^{1}\frac{x^{a-1}+x^{-a}}{1+x}\mathrm{d}x=\frac{\pi}{\mathrm{sin}a\pi}}$;\qquad \qquad \qquad \qquad \qquad &(4)\ $\displaystyle{\int_{0}^{\infty}\frac{x^{a-1}}{1+x}\mathrm{d}x=\frac{\pi}{\mathrm{sin}a\pi}}$
	\end{tabular}
\end{table}
\item 设$f(x),g(x)\in LR^2[-\pi,\pi]$, 求证:
$$\displaystyle{\int_{-\pi}^{\pi}[f(x)+g(x)]^2\mathrm{d}x+\int_{-\pi}^{\pi}[f(x)-g(x)]^2\mathrm{d}x=2[\int_{-\pi}^{\pi}f^2(x)\mathrm{d}x+\int_{-\pi}^{\pi}g^2(x)\mathrm{d}x]}$$.
\item 设$f(x),g(x)\in LR^2[-\pi,\pi]$, 它们的傅氏级数分别记为$a_n,b_n;\ \alpha_n,\beta_n$.求证:
$$\displaystyle{\frac{1}{\pi}\int_{-\pi}^{\pi}f(x)g(x)\mathrm{d}x=\frac{a_0\alpha_0}{2}+\sum\limits_{n=1}^{\infty}(a_n\alpha_n+b_n\beta_n)}$$.
\item 利用上题结果, 求证: 如果$f(x)\in LR^2[-\pi,\pi]$, 那么$f(x)$的傅氏级数可逐项积分, 即$$
\displaystyle{\int_{0}^{x}f(t)\mathrm{d}t=\int_{0}^{x}\frac{a_0}{2}\mathrm{d}t+\sum\limits_{n=1}^{\infty}\int_{0}^{x}(a_n\mathrm{cos}nt+b_n\mathrm{sin}nt)\mathrm{d}t}.$$
\item 利用逐项积分定理, 将$f(x)=x^4\ (-\pi\le x\le \pi)$展开为傅氏级数, 并用下列级数的和:
\begin{table}[H]
	\begin{tabular}{ll}
		(1)\ $\sum\limits_{n=1}^{\infty}\frac{(-1)^{n-1}}{n^4}$;\qquad \qquad \qquad \qquad \qquad&(2)\ $\sum\limits_{n=1}^{\infty}\frac{1}{n^8}$.
	\end{tabular}
\end{table}
\item 将如下定义的函数$f(x)$展开为傅氏级数:
$$f(x)=\begin{cases}
1-|x|/2h,\quad& 2\le |x| \le 2h,\\
0,\quad & 2h\le |x|\le \pi.
\end{cases}$$
并求下列级数的和:
\begin{table}[H]
	\begin{tabular}{lll}
		(1)\ $\sum\limits_{n=1}^{\infty}\frac{\mathrm{sin}^2nh}{n^2}$;\qquad \qquad\qquad&(2)\ $\sum\limits_{n=1}^{\infty}\frac{\mathrm{cos}^2nh}{n^2}$;\qquad \qquad \qquad &(3)\ $\sum\limits_{n=1}^{\infty}\frac{\mathrm{sin}^4nh}{n^4}$.
	\end{tabular}
\end{table}
\item 求证: 收敛级数$$
\sum\limits_{n=1}^{\infty}\frac{\mathrm{sin}nx}{\sqrt{n}}\ \ (0<x<2\pi)$$
不可能是某个黎曼可积函数的傅氏级数.
\item 将函数$f(x)=x\ (0\le x\le 1)$展开为傅氏级数.
\item 将如下定义的函数$f(x)$展开为傅氏级数:
$$f(x)=\begin{cases}
A,\quad &0<x<l,\\
0,\quad &l\le x\le 2l.
\end{cases}
$$
\item 设$f(x)\in C[-\pi,\pi],f(-\pi)=f(\pi)$, 且$f(x)$是奇函数, 它的傅氏级数为$$
f(x)\sim\sum\limits_{n=1}^{\infty}b_n\mathrm{sin}nx.$$
求证: 对$\forall h>0$, 函数$$
F(x)\overset{定义}{=}\frac{1}{2h}\int_{x-k}^{x+k}f(t)\mathrm{d}t \ \ (|x|\le \pi)$$
也是奇函数, 并求它的傅氏级数.
\end{enumerate}