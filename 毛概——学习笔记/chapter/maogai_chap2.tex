\chapter{新民主主义革命}
\section{新民主主义革命理论的形成}
\subsection{近代中国国情和中国革命的时代特征}
\begin{itemize}
	\item 认清国情是认清和解决革命问题的基本依据。\\
	近代中国,逐步沦为一个半殖民地半封建性质的社会,这是最基本的国情
	\item 近代中国半殖民地半封建的社会性质,决定了社会主要矛盾是帝国主义和中华民族的矛盾、封建主义和人民太众的矛盾。而帝国主义和中华民族的矛盾,又是各种矛盾中最主要的矛盾。
	\item 近代中国社会的性质和主要矛盾,决定了近代中国革命的根本任务是推融帝国主义义封建主义和宣僚资本主义的统治。
	\item 近代中国社会的性质和主要矛盾,决定了中国革命是资产阶级民主革命。
	\begin{itemize}
		\item 	从鸦片战争到辛亥革命期间,中国人民在不同时期和不同程度上进行的反帝反封建的斗争,属于旧民主主义革命的范畴。
		\item 十月革命使中国的资产阶级民主主义革命,从原来属于旧的世界资产阶级民主主义革命的范畴,属于旧的世界餐产阶级民主义革命的一部分,转变为属于新的资产阶级民主主义革命的范畴,属于世界无产阶级社会主义革命的一部分。
		\item 近代中国革命五四运动为开端,进入新民主主义革命阶段。\\
		1919年五四运动之后,中国无产阶级开始以独立的政治力量登上历史舞台,由自在阶级转变为自为的阶级。
	\end{itemize}
\end{itemize}
\subsection{中国新民主主义革命理论的实践基础}
新民主主义革命理论的形成,一是基于旧民主主义革命没有为中国民族找到出路;二是新民主主义革命的实践需要以及对中国革命经验教训的概括和总结。
\begin{itemize}
	\item 旧民主主义革命的失败呼唤着新民主理论的产生。
	\item 新民主主义革命的实践探索坚定了革命理论的基础。\\
	新民主主义革命实践,是新民主主义理论得以形成的实践基础和智慧源泉所在。
\end{itemize}
\section{新民主主义革命的总路线和基本纲领}
\subsection{新民主主义革命的总路线}
总路线是相对于具体路线而言的根本指导路线1939年,毛泽东在《中国革命和中国共产党》一文中,第一次提出了“新民主主义革命”的科学概念。\\ 
1948年,毛泽东在《在晋绥干部会议上的讲话》中完整地表述了总路线的内容,即无产阶级领导的,人民太众的,反对帝国主义、封建主义和官僚资本主义的革金。
\subsubsection{新民主主义革命的对象}
近代中国社会的性质和主要矛盾,决定了中国革命的主要敌人就是帝国主义、封建主义和官僚资本主义。
\begin{itemize}
	\item 帝国主义是中国革命的首要对象。
	\item 封建地主阶级是帝国主义统治中国和封建军阀实行专制统治的社会基础。\\
	民族革命和民主革命两个基本任务,既相互区别,又相互统一。
	\item 官僚资本主义,封建性质国家买办的国家垄断资本主义,严重束缚了中国社会生产力的发展。
\end{itemize}
但不同历史阶段,随着社会主要矛盾的变化,集中反对的主要敌人有所不同。
\begin{itemize}
	\item 在国共合作的大革命时期,革命的主要对象是帝国主义支持下的北洋军阀。
	\item 在十地革命战争时期,革命的主要对象是国民党新军阀。
	\item 在抗日战争时期,革命的主要对象是日本帝国主义。
	\item 在解放战争时期,革命的主要对象是美帝国主义支持下的国民党反动派。
\end{itemize}
\subsubsection{新民主主义革命的动力}
新民主主义革命的动力包括工人阶级、农民阶级、城市小资产阶级和民族资产阶级。
\begin{itemize}
	\item 无产阶级是中国革命最基本的动力。中国无产阶级是新的社会生产力的代表,是近代中国最进步的阶级,是中国革命的领导力量。
	\item 农民是中国革命的主力军\\
	农民问题是中国革命的基本问题,新民主主义革命实质上就是党领导下农民革命,中国革命战争实质上就是党领导下的农民战争。
	\item 城市小资产是无产阶级的可靠同盟者。
	\\城市小资产阶级,包括广大的知识分子、小商人、手工业者和自由职业者,同样受帝国主义、封建主义和官僚资本主义的压迫。因此,城市小资产阶级同样是中国革命的动力。
	\item 民族资产阶级也是中国革命的动力之一。(两面性)
\end{itemize}
\subsubsection{新民主主义革命的领导}
无产阶级的领导权是中国革命的中心问题,也是新民王主义革命理论的核心问题。区别新旧两种不同范畴的民主主义革命,根本的标志是革命的领导权是掌握在无产阶级手中还是掌握在资产阶级手中。\\
无产阶级及其政党的领导,是中国革命取得胜利的根本保证。
\subsubsection{新民主主义革命的性质和前途}
近代中国半殖民地半封建社会的性质和中国革命的历史任务,决定了中国革命的性质不是无产阶级社会主义革命,而是资产阶级民主主义革命。
\begin{itemize}
	\item 新民主主义革命与社会主义革命性质不同。新民主主义革命仍然属于资产阶级民主主义的革命范畴,它推翻帝国主义、封建主义和官僚资本主义的反动统治,在政治上争取和联合民族资产阶级去反对共同的敌人;在经济上保护民族工商业,容许有利于国计民生的私人资本主义发展。它要建立的是无产阶级领导的各革命阶级的联合专政,而不是无产阶级专政。
	\item 新民主主义革命旧民主主义革命比有新的内容和特点,集中表现在:
	\begin{itemize}
	\item 中国革命处于世界无产阶级社会主义革命的时代,是世界无产阶级社会主义革命的一部分;
		\item 革命的领导力量是中国无产阶级及其先锋队中国共产党;
		\item 革命的指导思想是马克思列宁主义;
		\item 革命的前途是社会主义而不是资本主义。
	\end{itemize}
	\item 新民主主义革命与社会主义革命与社会主义革命又相互联系、紧密连接,中间不容横插一个资产阶级专政。新民主主义革命是社会主义革命的必要准备,社会主义革命是民主主义革命的必然趋势。
	\item 革命前途问题上的错误表现:
	\begin{itemize}
		\item 一次革命论:在党的历史上,“左”倾教条主义的“一次革命论”的错误在于,只看到了民主革命与社会主义革命的联系,而混淆了民主革命和社会主义革命的区别,主张把社会主义革命阶段的任务放在民主革命阶段来完成,在反帝反封建的同时,也反对民族资产阶级,在政治上和经济上实行“左”的政策,使中国革命蒙受了重大损失。
		\item 二次革命论:右的“二次革命论”的错误在于,只看到了民主革命和社会主义革命的区别,而没有看到两个革命阶段的联系,主张在民主革命胜利后,建立一个资产阶级专政的资本主义国家,将来再进行社会主义革命。放在党对民主革命的领导权,同样使中国革命遭受了严重损失。
		 
	\end{itemize}

\end{itemize}
\subsection{新民主主义的基本纲领}
\begin{itemize}
	\item 新民主主义的政治纲领\\
	\text{\qquad}新民主主义的政治纲领是:推翻帝国主义和封建主义的统治,建立一个无产阶级领导的、以工农联盟为基础的、各革命阶级联合专政的新民主主义。\\
	新民主主义国家的国体是无产阶级领导的以工农联盟为基础,包括小资产阶级、民族资产阶级和其他反帝反封建的人们在内的各革命阶级的联合专政。与新民主主义国体相适应的政体是实行民主集中制的人民代表大会制度。总之,国体——各革金阶级联合专政,政体——民主集中制的人民代表大会制度,这就是新民主主义政治。
	\item 新民主主义的经济纲领\\
	\text{\qquad}新民主主义的经济纲领是、没收封建地主阶级的土地归农民,所没收官僚资产阶级的垄断资本归新民主主的国家所有,保护民族工商业。
	\begin{itemize}
		\item 没收封建地主阶级的土地归农民所有,是新民主主义革命的主要内容。
		\item 没收官僚资本归新民主主义国家所有,是新民主主义革命的题中应有之义。
		\item 保护民族工商业,是新民主主义经济纲领中极具特色的一项内容。
	\end{itemize}
\item 新民主主义的文化纲领\\
\text{\qquad}新民主主义文化,就是无产阶级领导的人民太众的反帝反封建的文化,即民族的科学的太众的文化。在新民主主义文化中居于指导地位的是共产主义思想。
\end{itemize}