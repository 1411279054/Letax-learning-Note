\chapter{新民主主义革命}
\section{新民主主义革命理论的形成}
\subsection{近代中国国情和中国革命的时代特征}
\begin{itemize}
	\item 认清国情是认清和解决革命问题的基本依据。\\
	近代中国,逐步沦为一个半殖民地半封建性质的社会,这是最基本的国情
	\item 近代中国半殖民地半封建的社会性质,决定了社会主要矛盾是帝国主义和中华民族的矛盾、封建主义和人民太众的矛盾。而帝国主义和中华民族的矛盾,又是各种矛盾中最主要的矛盾。
	\item 近代中国社会的性质和主要矛盾,决定了近代中国革命的根本任务是推融帝国主义义封建主义和宣僚资本主义的统治。
	\item 近代中国社会的性质和主要矛盾,决定了中国革命是资产阶级民主革命。
	\begin{itemize}
		\item 	从鸦片战争到辛亥革命期间,中国人民在不同时期和不同程度上进行的反帝反封建的斗争,属于旧民主主义革命的范畴。
		\item 十月革命使中国的资产阶级民主主义革命,从原来属于旧的世界资产阶级民主主义革命的范畴,属于旧的世界餐产阶级民主义革命的一部分,转变为属于新的资产阶级民主主义革命的范畴,属于世界无产阶级社会主义革命的一部分。
		\item 近代中国革命五四运动为开端,进入新民主主义革命阶段。\\
		1919年五四运动之后,中国无产阶级开始以独立的政治力量登上历史舞台,由自在阶级转变为自为的阶级。
	\end{itemize}
\end{itemize}
\subsection{中国新民主主义革命理论的实践基础}
新民主主义革命理论的形成,一是基于旧民主主义革命没有为中国民族找到出路;二是新民主主义革命的实践需要以及对中国革命经验教训的概括和总结。
\begin{itemize}
	\item 旧民主主义革命的失败呼唤着新民主理论的产生。
	\item 新民主主义革命的实践探索坚定了革命理论的基础。\\
	新民主主义革命实践,是新民主主义理论得以形成的实践基础和智慧源泉所在。
\end{itemize}