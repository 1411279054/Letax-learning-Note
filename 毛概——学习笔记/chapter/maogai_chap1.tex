\chapter{马克思主义中国化的两大理论成果}
\section{马克思主义中国化及其发展}
\subsection{马克思主义中国化的提出}

中国共产党领导中国革命、建设和改革的过程,就是把马克思主义的基本原理同中国的具体实际结合起来,实现马克思主义中国化的过程。
\begin{itemize}
	\item 党对这个问题的认识,经历了一个过程。
	\begin{itemize}
		\item 在1935年遵义会议之前,党对马克思同中国实际结合还不够自觉,使中国革命遭受严重挫折。
		\item 	毛泽东思想是党推进马克思主义中国化过程中第一个重大理论成果.
		\item 1938年,毛泽东在党的六届三中全会上作了《论新阶段》的政治报告。这是党首次明确提出“马克思主义中国化”的重大命题。
		\item 经历延安整风,推进马克思主义中国化,更好地指导中国革命,称为全党的共识。
		\item 在党的七大上(1945年),七大正式将毛择东思想确立为党的指导思想并写入党章。
		
	\end{itemize}
	\item 马克思主义中国化的必要性
	\begin{itemize}
		\item 第一,实现马克思主义中国化,是解决中国实际问题的客观需要。
		\item 第二,实现马克思主义中国化,是马克思主义理论发展的内在要求。
	\end{itemize}
	
\end{itemize}
\subsection{马克思主义中国化的科学内涵}
\textbf{马克思主义中国化}就是将马克思主义基本原理同中国具体实际相结合,不断形成具有中国特色的马克思主义理论成果的过程。具体地说,就是把马克思主义基本原理同中国革命、建设和改革的实践结合起来,既坚持马克思主义,又发展马克思主义。
\subsection{马克思主义中国化两大理论成果的关系}
\begin{itemize}
	\item 首先,毛泽东思想是中国特色社会主义理论体系的重要思想渊源。
	\begin{itemize}
		\item 毛泽东思想所蕴含的马克思主义的立场、观点和方法,为中国特色社会主义理论体系提供了基本遵循。\textbf{实事求是、群众路线、独立自主}是毛泽东思想的灵魂,是马克思主义根本立场、观点、方法的集中体现,也贯穿于中国特色社会主义理论体系之中。
		\item 毛泽东思想关于社会建设的理论,为开创和发展中国特色社会主义作了重要的理论准备。
		\item 其次,中国特色社会主义理论体系在新的历史条件下进一步丰富和发展了毛泽东思想。\\
		中国特色社会理论体系,在认真\textbf{总结中国特色社会主义建设历史经验和最新经验}的基础上,创造性地提出了一系列新思想、新观点、新论断,进一步丰富和发展了马克思列宁主义、毛泽东思想、中国特色社会主义理论体系同毛泽东思想是一脉相承而又与时俱进的,\textbf{这个“承”和“进”,主要体现在以上这些新思想、新观点、新论断上}。
		\item 最后,毛泽东思想和中国特色社会主义理论体系都是马克思主义在中国的运用和发展。
		\begin{itemize}
			\item 毛泽东思想和中国特色社会主义理论体系有着共同的“根”,这个“根”就是马克思列宁主义。
			\item 这两大理论成果虽以独创性成果丰富和发展了马克思的理论宝库,但仍属于马克思的科学体系,而不是取代马克思主义的另一个什么主义。
		\end{itemize}
	\end{itemize}
\end{itemize} 
\section{毛泽东思想}
\subsection{毛泽东思想的形成和发展}

\begin{itemize}
	\item 毛泽东思想形成和发展的时代背景和实践基础\\
	\begin{itemize}
		\item 毛泽东思想是马克思主义中国化的第一个重大理论成果,是中国共产党集体的结晶。
		\item 中国共产党领导的革命和建设的实践,是毛泽东思想形成的实践基础。
	\end{itemize}
	\item 毛泽东思想形成和发展的历史过程
	\begin{itemize}
		\item 《中国社会各阶段分析》(1925年),标志着毛泽东思想初步萌芽。
		\item 农村包围城市、武装夺取政权理论的提出,标志着毛泽东思想开始形成。
		\item 新民主主义革命理论的系统阐述,这标志着毛泽东思想走向成熟。1945年党的七大,毛泽东思想被确立为党的指导思想。
		\item 党在社会主义建设道路的初步探索中取得的重要理论成果,是毛泽思想的重要组成部分。
	\end{itemize}
\end{itemize}
\subsection{毛泽东思想的主要内容和活的灵魂}
\begin{itemize}
	\item 毛泽东思想在以下几个方面,以独创性的理论丰富和发展了马克思列宁主义
	\begin{itemize}
	\item 新民主主义理论\\
	新民主主义理论,是毛泽东思想达到成熟的主要标志。
	\item 社会主义革命和社会主义建设理论 
	\item 革命军队建设和军事战略的理论
	\item 政策和策略的理论
	\item 思想政治工作和文化工作的理论
	\item 党的建设理论
	\end{itemize}
	\item 毛泽东思想的活的灵魂,三个基本方面:实事求是、群众路线、独立自主
	\begin{itemize}
		\item 实事求是是党的根本思想路线
		\item 群众路线是党的根本工作路线
		\item 独立自主是党的根本政治路线
	\end{itemize}
	\item 毛泽东思想的历史地位
	\begin{itemize}
		\item 是马克思主义中国化的第一个历史性飞跃的理论成果
		\item 是中国革命和建设的科学指南
		\item 是党和人民的宝贵精神财富
		
	\end{itemize}
		\item 科学评价毛泽东和毛泽东思想\\
		正确区别毛泽东思想和毛泽东的思想
\end{itemize}



