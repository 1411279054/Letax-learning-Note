\chapter{马克思主义中国化的两大理论成果}
\section{马克思主义中国化及其发展}
\subsection{马克思主义中国化的提出}

中国共产党领导中国革命、建设和改革的过程,就是把马克思主义的基本原理同中国的具体实际结合起来,实现马克思主义中国化的过程。
\begin{itemize}
	\item 党对这个问题的认识,经历了一个过程。
	\begin{itemize}
		\item 在1935年遵义会议之前,党对马克思同中国实际结合还不够自觉,使中国革命遭受严重挫折。
		\item 	毛泽东思想是党推进马克思主义中国化过程中第一个重大理论成果.
		\item 1938年,毛泽东在党的六届三中全会上作了《论新阶段》的政治报告。这是党首次明确提出“马克思主义中国化”的重大命题。
		\item 经历延安整风,推进马克思主义中国化,更好地指导中国革命,称为全党的共识。
		\item 在党的七大上(1945年),七大正式将毛择东思想确立为党的指导思想并写入党章。
		
	\end{itemize}
	\item 马克思主义中国化的必要性
	\begin{itemize}
		\item 第一,实现马克思主义中国化,是解决中国实际问题的客观需要。
		\item 第二,实现马克思主义中国化,是马克思主义理论发展的内在要求。
	\end{itemize}
	
\end{itemize}
\subsection{马克思主义中国化的科学内涵}
\textbf{马克思主义中国化}就是将马克思主义基本原理同中国具体实际相结合,不断形成具有中国特色的马克思主义理论成果的过程。具体地说,就是把马克思主义基本原理同中国革命、建设和改革的实践结合起来,既坚持马克思主义,又发展马克思主义。
\subsection{马克思主义中国化两大理论成果的关系}
\begin{itemize}
	\item 首先,毛泽东思想是中国特色社会主义理论体系的重要思想渊源。
	\begin{itemize}
		\item 毛泽东思想所蕴含的马克思主义的立场、观点和方法,为中国特色社会主义理论体系提供了基本遵循。\textbf{实事求是、群众路线、独立自主}是毛泽东思想的灵魂,是马克思主义根本立场、观点、方法的集中体现,也贯穿于中国特色社会主义理论体系之中。
		\item 毛泽东思想关于社会建设的理论,为开创和发展中国特色社会主义作了重要的理论准备。
		\item 其次,中国特色社会主义理论体系在新的历史条件下进一步丰富和发展了毛泽东思想。\\
		中国特色社会理论体系,在认真\textbf{总结中国特色社会主义建设历史经验和最新经验}的基础上,创造性地提出了一系列新思想、新观点、新论断,进一步丰富和发展了马克思列宁主义、毛泽东思想、中国特色社会主义理论体系同毛泽东思想是一脉相承而又与时俱进的,\textbf{这个“承”和“进”,主要体现在以上这些新思想、新观点、新论断上}。
		\item 最后,毛泽东思想和中国特色社会主义理论体系都是马克思主义在中国的运用和发展。
		\begin{itemize}
			\item 毛泽东思想和中国特色社会主义理论体系有着共同的“根”,这个“根”就是马克思列宁主义。
			\item 这两大理论成果虽以独创性成果丰富和发展了马克思的理论宝库,但仍属于马克思的科学体系,而不是取代马克思主义的另一个什么主义。
		\end{itemize}
	\end{itemize}
\end{itemize} 
\section{毛泽东思想}
\subsection{毛泽东思想的形成和发展}

\begin{itemize}
	\item 毛泽东思想形成和发展的时代背景和实践基础\\
	\begin{itemize}
		\item 毛泽东思想是马克思主义中国化的第一个重大理论成果,是中国共产党集体的结晶。
		\item 中国共产党领导的革命和建设的实践,是毛泽东思想形成的实践基础。
	\end{itemize}
	\item 毛泽东思想形成和发展的历史过程
	\begin{itemize}
		\item 《中国社会各阶段分析》(1925年),标志着毛泽东思想初步萌芽。
		\item 农村包围城市、武装夺取政权理论的提出,标志着毛泽东思想开始形成。
		\item 新民主主义革命理论的系统阐述,这标志着毛泽东思想走向成熟。1945年党的七大,毛泽东思想被确立为党的指导思想。
		\item 党在社会主义建设道路的初步探索中取得的重要理论成果,是毛泽思想的重要组成部分。
	\end{itemize}
\end{itemize}
\subsection{毛泽东思想的主要内容和活的灵魂}
\begin{itemize}
	\item 毛泽东思想在以下几个方面,以独创性的理论丰富和发展了马克思列宁主义
	\begin{itemize}
	\item 新民主主义理论\\
	新民主主义理论,是毛泽东思想达到成熟的主要标志。
	\item 社会主义革命和社会主义建设理论 
	\item 革命军队建设和军事战略的理论
	\item 政策和策略的理论
	\item 思想政治工作和文化工作的理论
	\item 党的建设理论
	\end{itemize}
	\item 毛泽东思想的活的灵魂,三个基本方面:实事求是、群众路线、独立自主
	\begin{itemize}
		\item 实事求是是党的根本思想路线
		\item 群众路线是党的根本工作路线
		\item 独立自主是党的根本政治路线
	\end{itemize}
	\item 毛泽东思想的历史地位
	\begin{itemize}
		\item 是马克思主义中国化的第一个历史性飞跃的理论成果
		\item 是中国革命和建设的科学指南
		\item 是党和人民的宝贵精神财富
		
	\end{itemize}
		\item 科学评价毛泽东和毛泽东思想\\
		正确区别毛泽东思想和毛泽东的思想
\end{itemize}
\section{中国特色社会主义理论体系}
\subsection{中国特色社会主义理论体系的形成和发展}
中国特色社会主义理论体系是对马克思列宁主义、毛泽东思想的继承和发展,是被实践证明了的关于在中国建设、巩固和发展社会主义的正确的理论原则和经验总结,是中国共产党集体智慧的结晶。
\begin{itemize}
\item 中国特色社会主义理论体系的形成\\
中国特色社会主义理论体系,是在和平与发展成为时代主题的历史条件下,在我国改革开放和社会主义现代化建设的伟大实践中,在总结我国社会主义建设正反两方面历史经验和改革开放以来新鲜经验,并汲取其他社会主义国家兴衰成败经验教训的基础上,逐步形成和发展起来的。
\item 中国特色社会主义理论体系的发展
\begin{itemize}
	\item 1978年,召开党的十一届三中全会,重新确立了实事求是的思想路线。1997年,十五大正式使用“邓小平理论”概念并作为党的指导路线写入党章。
	\item 党的十六大将“三个代表”重要思想写入党章,实现了党的指导思想的又一次与时俱进。
	\item 党的十八大将科学发展观确立为党必须长期坚持的指导思想并写入党章。
\end{itemize}
\item 党的十八大以来,习近平总书记提出了“四个全面”战略布局,丰富和发展了中国特色社会主义理论体系,进一步深化了我们党对中国特色社会主义规律和马克思主义执政党建设规律的认识。
\end{itemize}
\subsection{中国特色社会主义理论体系的根本特征}
邓小平理论、“三个代表”重要思想、科学发
展观之所以被统称为中国特色社会主义理论体系
是因为:
\begin{itemize}
\item 它们都坚持以马克思列宁主义、毛泽东思想为指导,都坚持以中国特色社会主义为主题,都坚持实事求是的思想路线。
\item 它们既相互贯通又层层递进,体现了改革开放以来党的理论创新成果阶段性和系统性的内在统一.
\end{itemize}
\subsection{中国特色社会主义理论体系的主要内容}
\begin{itemize}
	\item 中国特色社会主义的思想路线
	\item 建设中国特色社会主义总依据理论
	\item 社会主义本质和建设中国特色社会主义总任
	务理论\\
	解放和发展社会生产力是社会主义的根本任务。实现社会主义现代化和中华民族伟大复兴,是建设中国特色社会主义的总任务。中国梦是实现中华民族伟大复兴的形象表达,其基本内涵是国家富强、民族振兴、人民幸福。
	\item 社会主义改革开放理论
	\item 建设中国特色社会主义总布局理论
	\item 实现祖国完全统一的理论
	\item 中国特色社会主义外交和国际战略理论
	\item 中国特色社会主义建设的根本母的和依靠力量理论
	\item 国防和军队现代化建设理论
	\item 中国特色社会主义建设的领导核心理论
	
 \end{itemize}
\subsection{中国特色社会主义理论体系的最新成果}
\begin{itemize}
	\item 中华民族伟大复兴的“中国梦”的提出\\
	\text{\qquad}“中国梦”的核心目标概括为“两个一百年目标”,也就是:到中华共产党成立100年时全面建成小康社会的目标一定能实现,到新中国成立100年时中华民族伟大复兴的梦想一定能实现。\\
	\text{\qquad}中华民族的伟大复兴,具体表现是国家富强、民族振兴、人民幸福。
	\item “四个全面”的战略布局
	\begin{itemize}
	\item “四个全面布局的提出”\\
	全面建成小康社会、全面深化改革、全面推进依法治国、全面从严治党
	\item “四个全面”战略布局提出的依据
	\item “四个全面”战略布局提出的关系\\
	战略目标是全面建成小康社会,战略举措是全面深化改革、全面从严治党。
	\end{itemize}
	\item “四个全面”战略布局的意义
\end{itemize}
\subsection{中国特色社会主义理论体系的历史地位}
\begin{itemize}
\item 马克思主义中国化第二次历史性飞跃的理论成果。
\item 新时期全党全国各族人民团结奋斗的共回思想基础。
\item 实现中华民族伟大 复兴中国梦的根本指针。
\end{itemize}
\subsection{四个自信}
道路自信、理论自信、制度自信、文化自信
\section{思想路线与理论精髓}
\subsection{实事求是思想路线的形成和发展}
\begin{itemize}
	\item 实事求是思想路线的形成\\
	1938年,毛泽东在党的六届六中全会上提出“马克思主义中国化”任务的同时,借用我国传统文化中“实事求是”命题,来提倡马克思主义同中国实际相结合的科学态度。
	\item 实事求是思想路线的确立
\end{itemize}
\subsection{实事求是思想路线的科学内涵}
《 中国共产章章程》确规定:“党的思想路线是一切从实际出发,理论联系实际实事求是,在实践中检验真理和发展真理。
\begin{itemize}
	\item 一切从实际出发是实事求是的前提
	\item 理论联系实际是实事求是的根本途径和方法
	\item 在实践中检验真理和发展真理是实事求是的印证条件和目的
\end{itemize}
\subsection{实事求是马克思主义中国化理论成果的精髓}
实事求是既是党的思想路线的核心,也是马克思主义中国化两大理论成果的精髓。

