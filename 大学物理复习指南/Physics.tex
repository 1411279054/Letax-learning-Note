\documentclass[cn,11pt,fancy,cyan]{elegantbook}
\title{大学物理复习题}
\subtitle{物理复习题}
\author{CharlesLC}
\institute{JXUST}
\date{\today}
\version{(期中版本)1.00}
\extrainfo{一个人真正的强大在于他的内心,在任何时候,都能撇开浮躁和喧哗,保持内心的清净。}
\logo{heart2.pdf}
\cover{cover2.jpg}

\usepackage{float}
\usepackage{amsmath}
\usepackage{wasysym}
\usepackage{enumitem}

\linespread{1.3}%1.3倍行距
\everymath{\displaystyle}
\newcommand{\dd}{\mathrm{d}}
\newcommand{\nl}{\underline{\makebox[4em]{}}}
\newcommand{\anl}[1]{\underline{\makebox[7em]{~#1}}}
\newcommand{\bnl}[1]{\underline{\makebox[4em]{~#1}}}
\newcommand{\spaces}{(\hspace*{1pc})}
\newcommand{\fourch}[4]{\\\begin{tabular}{*{4}{@{}p{3.25cm}}}(A)~#1 & (B)~#2 & (C)~#3 & (D)~#4\end{tabular}} % 一行
\newcommand{\twoch}[4]{\\\begin{tabular}{*{2}{@{}p{6cm}}}(A)~#1 & (B)~#2\end{tabular}\\\begin{tabular}{*{2}{@{}p{6cm}}}(C)~#3 & (D)~#4\end{tabular}}  %两行
\newcommand{\onech}[4]{\\(A)~#1 \\ (B)~#2 \\ (C)~#3 \\ (D)~#4}  % 四行
\newcommand{\insertfig}[3]{\begin{figure}[H]
    \centering
    \includegraphics[width=#1\textwidth]{#2}
    \caption{如图所示}\label{#3}
\end{figure}}
\usepackage{wallpaper}
\renewcommand{\CenterWallPaper}[2]{%
	\AddToShipoutPicture{\put(\LenToUnit{\wpXoffset},\LenToUnit{\wpYoffset}){%
			\parbox[b][\paperheight]{\paperwidth}{%
				\vfill
				\centering
				\tikz[opacity=0.075] \node[inner sep=0pt] {\includegraphics[angle=0,width=#1\paperwidth,height=#1\paperheight,keepaspectratio]{#2}};%
				\vfill
		}}
	}
} 
\CenterWallPaper{1}{picture/ctanlion.pdf}
\begin{document}
	\maketitle
    \tableofcontents
    \chapter{声明}
本产品不用与任何商业用途,最新版下载地址为:
\href{https://github.com}{Github}(点击即可下载),不保证题目和答案的正确性(因为本人能力有限),
但如有错误可通过QQ(见图\ref{fig:1}) \footnote{1411279054}或者邮箱\footnote{1411279054@qq.com}联系我。
\begin{figure}[htbp]
	\centering
	\includegraphics[width=0.3\textheight]{2weima.jpg}
	\caption{二维码}\label{fig:1}
\end{figure}

点击\href{https://github.com}{Github}后,找到$\mathrm{main.ptf}$后点击,点击$\mathrm{download}$即可。
    \chapter{分析基础}
\section{实数共理、确界、不等式}
\centering{\textbf{练习题}}
\begin{enumerate}
	\item 设$\mathrm{max}\{a+b\text{ },|a-b|\}<\tfrac{1}{2}$, 求证:\,$|\mathrm{a}|<\tfrac{1}{2},  |b|<\tfrac{1}{2}.$
	\begin{solution}
		$2|a|=|a+b+a-b|\le |a+b|+|a-b|\le 2\mathrm{max}\{a+b\text{ },|a-b|\}<1$ $\therefore|a|<\tfrac{1}{2}$\\
		$2|b|=|a+b-(a-b)|\le |a+b|+|a-b|\le 2\mathrm{max}\{a+b\text{ },|a-b|\}<1$ $\therefore|b|<\tfrac{1}{2}$                         
	\end{solution}
	\item 求证: 对$\mathbf{\forall}a,b\in \boldsymbol{R}$,有$\mathrm{max}\{ |a+b| ,|a-b| , |1-b| \} \ge \tfrac{1}{2}$.
	\begin{solution}
		$2 = |a+b-(a-b)+2(1-b)|\le |a+b|+|a-b|+2|1-b|\le 4\mathrm{max}\{|a+b|,|a-b|,|1-b|\} $ \\
		$\therefore \mathrm{max}\{|a+b|,|a-b|,|1-b|\}\ge \tfrac{1}{2} $
	\end{solution}
	\item 求证: 对$\forall a,b\in \boldsymbol{R}$,有\\
	$\mathrm{max} \{a,b\} = \tfrac{a\,+b}{2}+\tfrac{|a - b|}{2}$,\ $\mathrm{min} \{a,b\} = \tfrac{a+b}{2}-\tfrac{|a-b|}{2}$ ;\\
	并解释其几何意义.
	\begin{solution}
	易知,$\mathrm{max}\{a,b\}+\mathrm{min}\{a,b\}=a+b$ \circled{1}\quad $\mathrm{max}\{a,b\}-\mathrm{min}\{a,b\}=|a-b|$\circled{2}\\
	由\circled{1}、\circled{2}得$\mathrm{max} \{a,b\} = \tfrac{a\,+b}{2}+\tfrac{|a - b|}{2}$\quad
	$\mathrm{min} \{a,b\} = \tfrac{a+b}{2}-\tfrac{|a-b|}{2}$\\
	几何意义:$\mathrm{max}\{a,b\}$指的是$a,b$中较大的那个, $\mathrm{min}\{a,b\}$指的是$a,b$中较小的那个。
	\end{solution}
	\item 设$f\left( x \right) $在集合$X$上有界,求证:
	$$
	|f\left( x \right)-f\left( y\right)  | \le \underset{x\in X}{\mathrm{sup}}f\left( x\right)  -\underset{x\in X}{\mathrm{inf}}f\left( x \right)\quad (\forall\,x, y \in X)
	$$ 
	\begin{solution}
			$f(x)-f(y) \le \underset{x \in X}{\mathrm{sup}}f(x) - \underset{x \in X}{\mathrm{inf}}f(x)$ $\therefore |f(x)-f(y)| \le |\underset{x \in X}{\mathrm{sup}}f(x) - \underset{x \in X}{\mathrm{inf}}f(x)|=\underset{x \in X}{\mathrm{sup}}f(x) - \underset{x \in X}{\mathrm{inf}}f(x)$
	\end{solution}
	\item 设 $f(x)$,$g(x)$在集合$X$上有界, 求证:\\
	
	$$\circled{1}
	\underset{x\in X}{\mathrm{inf}}\{f(x)\}\,+\,\underset{x\in X}{\mathrm{inf}}\{g(x)\} \le
	\underset{x\in X}{\mathrm{inf}}\{f(x)+g(x)\} \le 	\underset{x\in X}{\mathrm{inf}}\{f(x)\}\,+\,\underset{x\in X}{\mathrm{sup}}\{g(x)\}
	$$ 
	$$\circled{2}
		\underset{x\in X}{\mathrm{sup}}\{f(x)\}\,+\,\underset{x\in X}{\mathrm{inf}}\{g(x)\} \le 	\underset{x\in X}{\mathrm{sup}}\{f(x)+g(x)\} \le \underset{x\in X}{\mathrm{sup}}\{f(x)\}\,+\,	\underset{x\in X}{\mathrm{sup}}\{g(x)\}
	$$
	\begin{solution}
		\textcolor{green}{\circled{1}} 易知, $\underset{x\in X}{\mathrm{sup}}\{f(x)\}\,+\,\underset{x\in X}{\mathrm{inf}}\{g(x)\}\le f(x)+g(x)\ (\forall x \in X)$, $\therefore \underset{x\in X}{\mathrm{inf}}\{f(x)\}\,+\,\underset{x\in X}{\mathrm{inf}}\{g(x)\} \le
		\underset{x\in X}{\mathrm{inf}}\{f(x)+g(x)\}$, 又$\because \underset{x\in X}{\mathrm{inf}}\{f(x)+g(x)\} \le f(x)+g(x) \le f(x)\,+\,\underset{x\in X}{\mathrm{sup}}\{g(x)\} $, 即$\underset{x\in X}{\mathrm{inf}}\{f(x)+g(x)\} - \underset{x\in X}{sup}g(x)\le f(x), (\forall x \in X)$, $\therefore \underset{x\in X}{\mathrm{inf}}\{f(x)+g(x)\} \le 	\underset{x\in X}{\mathrm{inf}}\{f(x)\}\,+\,\underset{x\in X}{\mathrm{sup}}\{g(x)\}$, 所以,
		$
		\underset{x\in X}{\mathrm{inf}}\{f(x)\}\,+\,\underset{x\in X}{\mathrm{inf}}\{g(x)\} \le
		\underset{x\in X}{\mathrm{inf}}\{f(x)+g(x)\} \le 	\underset{x\in X}{\mathrm{inf}}\{f(x)\}\,+\,\underset{x\in X}{\mathrm{sup}}\{g(x)\}
		$
		\textcolor{green}{\circled{2}} 类似上面做法.

		
	\end{solution}
\end{enumerate}
\section{函数}
\centering{\textbf{练习题}}
\begin{enumerate}
\item 设$f(x)=|1+x|-|1-x|$.
	\begin{enumerate}
		\item 求证:$f(x)$是奇函数;
		\item 求证:$|f(x)|\le 2$.
		\item 求$\underset{n\text{次}}{\underbrace{(f\circ f \circ \cdots \circ f)}}\left(	x \right) .$
	\end{enumerate}
	\begin{solution}
		\begin{enumerate}[(1)]
			\item $f(x)=f(-x)$,$\therefore f(x)$是奇函数.
			\item $f(x)=|1+x|-|1-x|\le|1+x+1-x|=2$ 
			\item 易知,$f(x)$是一个分段函数,
			$f(x)=\left\{\begin{array}{lcc}
					-2 &  & x < -1\\
					2x & &-1 \le  x\le 1\\
					2 & &x \ge 1
				
			\end{array}\right.
			$, 下面当$-1 \le  x\le 1$时,$f(x) = 2x$  $\therefore (f\circ f)(x) =\left\{\begin{array}{lcc}
			-2 &  & x < \tfrac{-1}{2}\\
			4x & & \tfrac{-1}{2} \le  x\le \tfrac{1}{2}\\
			2 & &x \ge \tfrac{1}{2}
			
		\end{array}\right. $ \quad $\therefore$ 可得,$(f\circ f \circ \cdots \circ f)(x)=\left\{\begin{array}{lcc}
		-2 &  & x < \tfrac{-1}{2^(n-1)}\\
		2^(n-1)x & & \tfrac{-1}{2^(n-1)} \le  x\le \tfrac{1}{2^(n-1)}\\
		2 & &x \ge \tfrac{1}{2^(n-1)}
		
	\end{array}\right.$
		\end{enumerate}
	\end{solution}
\item 设$f(x)$在(0,+$\infty$)上定义, $a>0,b>0$.求证:
	\begin{enumerate}
		\item 若$\frac{f(x)}{x}$单调下降, 则$f(a+b)\le f(a)+f(b)$;
		\item 若$\frac{f(x)}{x}$单调上升, 则$f(a+b)\ge f(a)+f(b)$
	\end{enumerate}
	\begin{solution}
		\begin{enumerate}[(1)]
			\item 由已知得, $\frac{f(x)}{x}$单调下降\quad$\therefore \frac{f(a+b)}{a+b} \le \tfrac{f(a)}{a},\ \frac{f(a+b)}{a+b} \le \frac{f(b)}{b}$,$\therefore af(a+b) \le (a+b)f(a),\ bf(a+b)\le (a+b)f(b)$, 可得$f(a+b)\le f(a)+f(b)$.
			\item 与第一小题类似.
		\end{enumerate}
	\end{solution}
\item 利用上题证明: 当$a>0,b>0$时,有
	\begin{enumerate}
		\item	当$p>1$时, $\left( a+b\right) ^p \ge a^p+b^p$;
		\item 当$0<p<1$时,$(a+b)^p\le a^p+b^p$.
	\end{enumerate}
	\begin{solution}
		\begin{enumerate}[(1)]
			\item 令$f(x)=x^p,\ \frac{f(x)}{x}=x^{p-1}$, $ \because p>1 , p-1>0\ \therefore x^{p-1} $单调递增,由第二题可得$f(a+b)\ge f(a)+f(b)$\ $\therefore \left( a+b\right) ^p \ge a^p+b^p$
			\item 与第一小题类似
		\end{enumerate}
	\end{solution}
\item 设$f(x)$在\textbf{R}上定义, 且$f(f(x))\equiv x$.
	\begin{enumerate}
		\item 问这种函数有几个?
		\item 若$f(x)$为单调增加函数, 问这种函数有几个?
	\end{enumerate}
	\begin{solution}
		\begin{enumerate}[(1)]
			\item 令$y = f(x),\ x = f^{-1}(y)\ \because f(f(x))\equiv x \therefore f(y) \equiv f^{-1}(y)$, 说明其原函数等于反函数,说明函数图像关于直线$y=x$对称,其这样的函数有无数多个.
			\item 一个, $f(x) \equiv x$
		\end{enumerate}
	\end{solution}
\item 求证:若$y=f(x)(x\in (-\infty, +\infty))$是奇函数, 并且它的图像关于直线$x=b(b>0)$对称, 则函数$f(x)$是周期函数并求其周期.
\begin{solution}
	$\because f(x)$是奇函数, $\therefore f(x) = -f(-x)$,又$\therefore f(x)$关于直线$x=b(b>0)$对称, $f(b+x)=f(b-x)$, 即$f(b+b+x)=f(-x)=-f(x)$, $f(x+2b)=-f(x)=-f(x+2b-2b)=f(x-2b)$, $\therefore f(x+4b)=f(x)$, 因此$f(x)$是周期函数, 其周期是$4b$.
\end{solution}
\item 设$f: X\rightarrow Y$时满射, $g: Y\rightarrow Z$.求证:$g\circ f: X\circ Z$.有反函数的充分必要条件为$f$和$g$都有反函数存在, 且$(g\circ f)^{-1}=f^{-1} \circ g^{-1}$.
\begin{solution}
	$g\circ f:X\circ Z$有反函数,说明$g\circ f$一一对应,即$f$和$g$都一一对应, 所以, $f$和$g$存在反函数, 令$(g\circ f)$的反函数为$H$, 假设$H(a)=b$,有$(g\circ f)(b) = a$,左乘$g^{-1}$, 即$f(b) = g^{-1}(a)$, 再左乘$f^{-1}$, 即$b = (f^{-1}\circ g^{-1})(x)$\ $\therefore H=f^{-1}\circ g^{-1}, (g\circ f)^{-1} = f^{-1}\circ g^{-1}$.
\end{solution}
\end{enumerate}
\section{序列极限}
\centering{\textbf{练习题}}
\begin{enumerate}
	\item 设$x_n>0, \lim\limits_{n\rightarrow \infty}x_n=a$.
	\begin{enumerate}
		\item 当$a\ne 0$时, 求证:$\lim\limits_{n \rightarrow \infty}\frac{x_{n+1}}{x_{n}}=1$
		\item 举例说明当$a=0$时,$\lim\limits_{n \rightarrow \infty}\frac{x_{n+1}}{x_{n}} \ne 1$ 可能成立;
		\item 举例说明当$a=1$时, $\lim\limits_{n \rightarrow \infty}(x_n)^n\ne 1$可能成立.
	\end{enumerate}
	\item 设$0<x_1<1, x_{n+1}=1-\sqrt{1-x_n}$, 求$\lim\limits_{n \rightarrow \infty}x_n$和$\lim\limits_{n\rightarrow \infty}\frac{x_{n+1}}{x_1}$. 
	\item 设$c>1$, 求序列$\sqrt{c}, \sqrt{c\sqrt{c}},\sqrt{c\sqrt{c\sqrt{c}}},\cdots$的极限.
	\item 设$A>0,x_1>0,x_{n+1}=\frac{1}{2}(x_n+\frac{A}{x_n})\,(n=1,2,\cdots)$
	\begin{enumerate}
		\item 求证: $x_n$单调下降且有界;
		\item 求$\lim\limits_{n\rightarrow \infty}x_n$.
	\end{enumerate}
	\item 设$F_0=F_1=1,F_{n+1}=F_n+F_{n-1}$, 求证:$\lim\limits_{n\rightarrow \infty}\tfrac{F_{n-1}}{F_n}=\frac{\sqrt{5}-1}{2}$.
	\item 求证:
	\begin{enumerate}
		\item $\frac{1}{2\sqrt{n+1}}<\sqrt{n+1}-\sqrt{n}<\frac{1}{2\sqrt{n}}$;
		\item 序列$x_n=1+\frac{1}{\sqrt{2}}+\cdots+\frac{1}{n}-2\sqrt{n}$的极限存在.
	\end{enumerate}
	\item 设$0<a_1<b_1$, 令
	$$
		a_{n+1} = \sqrt{a_n\cdot b_n},\ b_{n+1}=\tfrac{a_n+b_n}{2}\ (n=1,2,\cdots)
	$$求证: 序列${a_n},{b_n}的极限存在$.
	\item 求证: 如下序列的极限存在.
	$$
	\lim\limits_{n\rightarrow \infty}(1+\frac{1}{2^x}(1+\frac{1}{3^2})\cdots(1+\frac{1}{n^2})).
	$$
	\item 求证: 如下序列的极限存在:
	$$
	\lim\limits_{n\rightarrow \infty} \left[\frac{(2n)!!}{(2n-1)!!}\right]^2\frac{1}{2n+1}.
	$$
	\item 设$c>0$,求序列
	$$
	\sqrt{c}, \sqrt{c+\sqrt{c}},\sqrt{c+\sqrt{c+\sqrt{c}}}, \cdots
	$$
	的极限.
	\item 设$x_n=a_1+a_2+\cdots+a_n$,求证: 若$\tilde{x}=|a_1|+|a_2|+\cdots+|a_n|$极限存在,则${x_n}$的极限也存在.
	\item 设$x_n=a_1+a_2+\cdots+a_n,y_n=b_1+b_2+\cdots+b_n,z_n=c_1+c_2+\cdots+c_n$, 且$c_n\le a_n\le b_n(n=1,2,\cdots)$; 又设${y_n},{z_n}$极限存在.求证: ${x_n}$极限也存在.
	\item 设序列${x_n}$满足$|x_{n+1}-x_n|\le q|x_n-x_{n-1}|(n=1,2,\cdots)$, 其中$0<q<1$.求证:
	序列${x_n}$的极限存在.
	\item 设$f(x)$在$(-\infty,+\infty)$上满足条件:
	$$|f(x)-f(y)|\le q|x-y|\quad (\forall x,y \in (-\infty,+\infty))$$
	其中$0<q<1$.对$\forall x_1\in (-\infty,+\infty)$, 令$x_{n+1}=f(x_n)(n=1,2,\cdots)$.求证: 序列${x_n}$的极限存在, 且极限值是$f(x)$的不动点.
	\item 设$x_0=a,x_1=b(b>a)$, 用如下公式定义序列的项: $$
	x_{2n}=\frac{x_{2n-1}+2x_{2n-2}}{3},\ x_{2n+1}=\frac{2x_{2n}+x_{2n-1}}{3}\quad (n = 1,2,\cdots)$$
	求证: 序列${x_n}$极限存在.
\end{enumerate}
\section{函数极限与连续概念}
\centering{\textbf{练习题}}
\begin{enumerate}
	\item 设在正实轴上, $h(x)\le f(x)\le g(x)$, 且广义极限$$
	\lim\limits_{x\rightarrow \infty}h(x) = A = \lim\limits_{x\rightarrow \infty}g(x)$$存在.求证: $\lim\limits_{x\rightarrow \infty}f(x)=A$(分别讨论$A=+\infty,-\infty,有限数三种情形$).
	\item 设$\lim\limits_{x\rightarrow a}f(x)=+\infty,\lim\limits_{x\rightarrow a}g(x)=A(>0)$, 求证:$$
	\lim\limits_{x\rightarrow a}f(x)g(x) = +\infty$$
	\item 设$0<x_n<+\infty$, 且满足$x_n+\frac{4}{x^2}<3$, 求证: 极限$\lim\limits_{x_n}存在, 并求出此极限值$.
	\item 设$f(x)$是$(-\infty,+\infty)$上的周期函数, 又$$
	\lim\limits_{x\rightarrow +\infty}f(x)=0$$ 求证:$f(x)\equiv 0$.
	\item 设$f(x),g(x)$在$(a,+\infty)$上定义,$g(x)$单调上升, 且$$
	\lim\limits_{x\rightarrow +\infty}g(f(x))=+\infty.$$
	求证: $\lim\limits_{x\rightarrow +\infty}f(x)=+\infty$, $\lim\limits_{x\rightarrow +\infty}g(x)=+\infty$.
	\item 设$x_n=\tfrac{1}{1 \cdot n}+\tfrac{1}{2 \cdot (n-1)}+\cdots+\tfrac{1}{(n-1) \cdot 2}+\tfrac{1}{n \cdot 1}$, 求$\lim\limits_{n\rightarrow \infty}x_n$.
	\item 设$\lim\limits_{n\rightarrow \infty}\tfrac{a_1+a_2+\cdots+a_n}{n}=a$, 求证: $\lim\limits_{n\rightarrow \infty}\tfrac{a_n}{n}=0$
	\item 设${x_n}$满足$\lim\limits_{n\rightarrow \infty}(x_n-x_{n-2})=0$, 求证: $\lim\limits_{n\rightarrow \infty}\tfrac{x_n}{n}=0$.
	\item 适当定义$f(0)$, 使函数$f(x)=(1-2x)^\frac{1}{x}$在点$x=0$处连续.
	\item 设$ f(x),g(x)\in C[a,b]$,求证:
	\begin{enumerate}
		\item $|f(x)|\in C[a,b];$\
		\item $max\{f(x),g(x)\}\in C[a,b]$;
		\item $min\{f(x),g(x)\}\in C[a,b]$.
	\end{enumerate}
	\item 设$f(x)\in C[a,b]$单调上升, 且$a<f(x)<b\ (\forall x\in [a,b])$.对$\forall x_1 \in [a,b]$, 由递推公式$x_{n+1}=f(x_n)(n=1,2,\cdots)$产生序列$\{x_n\}$.求证: 极限$\lim\limits_{n\rightarrow \infty}x_n$存在, 且其极限值$c$满足$c=f(c)$.
	\item 设序列$\{x_n\}$由如下迭代产生:
	$$
	x_1 = \tfrac{1}{2}, x_{n+1} = x_{n}^2 + x_n \quad (n = 1,2,\cdots)
	$$求证:$\lim\limits_{n\rightarrow \infty}(\frac{1}{1+x_1}+\frac{1}{1+x_2}+\cdots+\frac{1}{1+x_n})=2$
	\item 求出函数$f(x)=\frac{1}{1+\frac{1}{x}}$的间断点, 并判断间断点的类型.
\end{enumerate}
\section{闭区间上连续函数的性质}
    \chapter{一元函数微分学}
\section{导数和微分}
\centering{\textbf{练习题}}
\begin{enumerate}
    \item 用定义求$f'(0)$, 这里$f(x)=\begin{cases}
    x^2sin\frac{1}{x},&x\ne 0\\
    0,&x=0
    \end{cases}$
    \item 设$f'(x_0)$存在.求证:对数导数也存在并等于$f'(x_0)$, 即\\
    $$\lim\limits_{h\rightarrow 0}\frac{f(x_0+h)-f(x_0-h)}{2h}=f'(x_0)$$.
    \item 设$f(x)$在点$x_0$处可导,$\alpha_n,\beta_n$为趋于零的正数序列, 求证: 
    $$\lim\limits_{n\rightarrow \infty}\frac{f(x_0+\alpha_n-f(x_0+\beta_n))}{\alpha_n-\beta_n}
    = f'(x_0)$$
    \item 设$P(x)$是最高次项系数为1的多项式, $M$是它的最大实数.求证:$P'(M)\ge 0$.
    \item 给定曲线$y=x^2+5x+4$.
    \begin{enumerate}
    	\item 求曲线在点$(0,4)$处的切线.
    	\item 确定$b$使得直线$y=3x+b$为曲线的切线;
    	\item 求过点$(0,3)$的曲线的切线.
    \end{enumerate}
	\item 确定常数$a,b$使得函数$f(x)=\begin{cases}
	ax+b,&x>1\\
	x^2,&x\le 1
	\end{cases}$
	有连续导数.
	\item 设曲线由隐式方程$\sqrt[3]{x^2}+\sqrt[3]{y^2}=\sqrt[3]{a^2}(a>0)$给出.
	\begin{enumerate}
		\item 求证: 曲线的切线被坐标轴所截的长度为一常数;
		\item 写出曲线的参数式, 利用参数式求导给出上一小题的另一证法.
	\end{enumerate}
		\item 已知曳物线的参数方程为$$
		x=a[\mathrm{ln}(\mathrm{tan}\frac{t}{2})+\mathrm{cos}t],\quad y=a\mathrm{sin}t\quad
		(a>0,0<t<\pi).$$
		求证:在曳物线的任意切线上, 自切点至该切线与$x$轴交点之间的切线段为一定长.
		\item 试确定$\lambda$,使得曲线$\frac{x^2}{a^2}+\frac{y^2}{b^2}$与$xy=\lambda$相切, 并求出切线方程.
		\item 试确定$m$, 使直线$y=mx$为曲线$y=lnx$的切线.
		\item 设$y=\frac{\mathrm{arcsin}x}{\sqrt{1-x^2}}$.
		$(1)\text{求证}: (1-x^2)y'-xy=1;\qquad \quad(2)\text{求}y^{(n)}(0)$.
		\item 求$f(x)=\frac{x}{1-x^2}$的$n$阶导数.
		\item 设$y=x^(n-1)\mathrm{ln}x$.求证:$y^{(n)}=\frac{(n-1)!}{x}$.
		\item 求证:双曲线$r^2=a^2\text{cos}2\theta$的向径与切线的夹角等于极角的两倍加$\frac{\pi}{2}$
		\item 设曲线既可用参数式$x=x(t),y=y(t)$表示, 又可用极坐标$r=r(\theta)$表示.求证:$\frac{1}{2}r^2\mathrm{d}\theta=\frac{1}{2}(x\mathrm{d}y-y\mathrm{d}x)$.
		
	\end{enumerate}
\section{微分中值定理}
\centering{\textbf{练习题}}
\begin{enumerate}
	\item 设$f(x)$在$[a,b]$上连续, 在$(a,b)$内除仅有的一个点都可导.求证: $\exists c_1,c_2\in(a,b)$及$\theta \in (0,1)$, 使得
	$$f(b)-f(a)=(b-a)[\theta f'(c_1)+(1-\theta)f'(c_2)]
	$$
	\item 设函数$f(x)$在$[a,b]$上连续, 在$(a,b)$内可导, 且$$
	f(a)\cdot f(b) > 0,\ f(a)\cdot f(\frac{a+b}{2})<0.$$
	求证: 对$\forall k\in R, \exists \xi\in (a,b)$, 使得$f'(\xi)=kf(\xi)$.
	\item 设$f(x)$在$[a,b]$上连续, 在$(a,b)$内可导, 但非线性函数. 求证: 
	$\exists \xi,\eta\in(0,3), $使得$f'(\xi)=0$.
	\item 设$f(x)$在$[a,b]$上连续, 在$(a,b)$内可导, 但非线性函数.求证: $\exists \xi,\eta\in (a,b)$,使得$$
	f'(\xi)<\frac{f(b)-f(a)}{b-a}<f'(\eta).$$
	\item 设$f(x)$在$(a,b)$内二阶可导,且$x_0\in (a,b)$, 使得$f''(x_0)\ne 0$.求证:
	\begin{enumerate}
		\item 如果$f'(x_0)=0$,则存在$x_1,x_2\in (a,b)$,使得${f(x_1)-f(x_2)}=0$;
		\item 如果$f'(x_0)\ne 0$, 则存在$x_1,x_2\in (a,b)$, 使得$\frac{f(x_1)-f(x_2)}{x_1-x_2}=f'(x_0)$.
	\end{enumerate}
	\item 设$f(x)$在$[0,1]$上可导, $f(0)=0, f(x)\ne 0(\forall x\in(0,1))$.求证: 如果$f(x)$在$(0,1)$上不恒等于零, 则存在$\xi \in (0,1)$, 使得$f(\xi)\cdot f'(\xi)>0$.
	\item 设$f(x)$在$[0,1]$上可导,$f(0)=0,f(x)\ne 0(\forall x\in (0,1))$.求证:
	存在$\xi\in(a,b)$, 使得$f'(\xi)+f(\xi)=0$.
	\item 设$f(x)$在$[0,1]$上连续,在$(0,1)$内可导,$f(0)=0$.求证:如果$f(x)$在$(0,1)$上不恒等于零,则存在$\xi \in (0,1)$, 使得$f(\xi)\cdot f(\xi)>0$.
	\item 设函数$f(x)$在$[a,b]$上可导, 且$f'(a)=f'(b)$.求证: $\exists c\in (a,b), $使得$f(c)-f(a)=(c-a)f'(c)$ \\
	注:\quad 本题与本节例12比较,就是把条件$f'(a)=f'(b)=0$中的“=0”去掉了.
	\item 设$f(x)$在(0,1]上可导, 且存在有限极限$\lim\limits_{h\rightarrow 0+}\sqrt{x}f'(x)$.求证:$f(x)$在(0,1]上一致连续.
\end{enumerate}

\section{函数的升降、极值、最值问题}
\centering\textbf{练习题}
\begin{enumerate}
	\item 求证: 
	\begin{enumerate}
		\item 当$x\ge 0$时, $f(x)=\frac{x}{1+x}$单调增加;
		\item $\frac{|a+b|}{1+|a+b|}\le \frac{|a|}{1+|a|}+\frac{|b|}{1+|b|}$.
	\end{enumerate}
		\item 设$f(x)$在$[0,a]$上二次可导, 且$f(0)=0, f''(x)<0$.求证: $\frac{f(x)}{x}$在$(0,a]$上单调下降.
		\item 求证: 对任何$n(n>0)$次多项式$P(x),\exists x_0>0$,使得$P(x)$在$(-\infty,-x_0)$和在$(x_0,+\infty)$上都是严格单调的.
		\item 设$f(x)$在$[a,b]$上连续, 且在$(a,b)$内只有一个极大值点和一个极小值点. 求证: 极大值必大于极小值.
		\item 设$a,b>0,k\in R$.求证: 函数$f(x)=a^2e^{kx}+b^2e^{-kx}$存在与$k$无关的极小值.
		\item 
		\begin{enumerate}
			\item 设$f(x), g(x)$在$(a,b)$内可导, 且$f(x)\ne g(x),g(x)\ne 0$.求证: $\frac{f(x)}{g(x)}$在$(a,b)$内无极值的充分必要条件是$\frac{f(x)+g(x)}{f(x)-g(x)}$在$(a,b)$内无极值.
			\item 设$b>a>0$, 求证: $f(x)=\frac{(x-a)(x+b)}{(x-b)(x+a)}$无极值.
		\end{enumerate}
	\item 设函数$f(x)$在$(-\infty,+\infty)$内连续, 其导函数的图形如图2.4所示, 则$f(x)$有(\ ).
\end{enumerate}

\section{函数的凹凸性、拐点及函数作图}

\section{洛必达法则与泰勒公式}

\section{一元函数微分学的总合应用}
    \chapter{运动学牛顿定律习题课}
\section{作业习题}
\subsection*{一、填空题}
\begin{enumerate}
    \item 升降机内地板上放有物体$A$, 其上再放另一物体$B$, 二者的质量分别为$M_A$、$M_B$.当升降机以加速度$a$向下加速运动时$(a<g)$, 物体$A$对升降机地板的压力大小$N=\nl$.  
    \item 如图所示 \ref{fig:12} , 一个小物体$A$靠在一辆小车的竖直前壁上,$A$和车壁间静摩擦系数是$μ_s$,若要使物体$A$不致掉下来,小车的加速度的最小值应为$a=\nl$.
    \begin{figure}[h]
        \centering
        \includegraphics[width=0.15\textheight]{fig12}
            \caption{如图}\label{fig:12}
    \end{figure}
    \item 如图 \ref{fig:13} , 一圆锥摆摆长为$l$、摆锤质量为$m$,在水平面上作匀速圆周运动,摆线与铅直线夹角$\theta$,则摆线的张力$T=\nl$.摆锤的速率$v=\nl$.
    \begin{figure}[h]
        \centering
        \includegraphics[width=0.15\textheight]{fig13}
            \caption{如图}\label{fig:13}
    \end{figure}
\end{enumerate}
\subsection*{二、选择题}
\begin{enumerate}
    \item  如图 \ref{fig:14} , 一只质量为$m$的猴,原来抓住一根用绳吊在天花板上的质量为$M$的直杆,悬线突然断开,小猴则沿杆子竖直向上爬以保持它离地面的高度不变,此时直杆下落的加速度为(\hspace{1pc})
    \fourch{$g$ ;}{$\frac{m}{M}g$ ;}{$\frac{M+m}{M-m}g$ ;}{$\frac{M+m}{M}G$ .}
    \begin{figure}[ht]
        \centering
        \includegraphics[width=0.06\textheight]{fig14}
            \caption{如图}\label{fig:14}
    \end{figure}
    \item 一段路面水平的公路,转弯处轨道半径为$R$,汽车轮胎与路面间的摩擦系数为$\mu$,要使汽车不致于发生侧向打滑,汽车在该处的行驶速率(\hspace{1pc})
    \twoch{不得小于$\sqrt{\mu gR}$ ;}{不得大于$\sqrt{\mu gR}$ ;}{必须等于$\sqrt{2gR}$ ;}{还应由汽车的质量$M$决定 .}
    \item 质量分别为$m_1$和$m_2$的两滑块$A$和$B$通过一轻弹簧水平连结后置于水平桌面上,滑块与桌面间的摩擦系数均为$\mu$,系统在水平拉力$F$作用下匀速运动,如图所示\ref{fig:15}。如突然撤消拉力,则刚撤消后瞬间,二者的加速度$a_A$和$a_B$分别为(\hspace{1pc}) 
    \begin{figure}[ht]
        \begin{minipage}[ht]{0.4\linewidth}
           \begin{table}[H]
               \begin{tabular}{c}
                  \qquad   (A)\ $a_A=0, a_B=0$;\\
                  \qquad  (B)\ $a_A>0, a_B<0$;\\
                  \qquad  (C)\ $a_A<0, a_B>0$;\\
                  \qquad  (D)\ $a_A<0, a_B=0$.
               \end{tabular}
           \end{table}
        \end{minipage}
        \begin{minipage}[h]{0.5\linewidth}
            \includegraphics[width=0.25\textheight]{fig15}
            \caption{如图}\label{fig:15}
        \end{minipage}
    \end{figure}
\end{enumerate}
\subsection*{三、计算题}
\begin{enumerate}
    \item 质量为$m$的质点沿$x$轴正向运动: 设质点通过坐标点为$x$时的速度为$v=kx$($k$为常数), 求作用在质点的合外力及质点从$x=x_0$到$x=2x_0$处所需的时间$t$.
    \item 如图 \ref{fig:16}, 在光滑水平桌面上,有两个物体$A$和$B$紧靠在一起.它们的质量分别为$m_A=2kg$, $m_B=1kg$. 今用一水平力$F=3N$推物体$B$,求$B$推$A$的力$f$. 
    \begin{figure}[ht]
        \centering
        \includegraphics[width=0.15\textheight]{fig16}
            \caption{如图}\label{fig:16}
    \end{figure}
    \item 如图所示 \ref{fig:17}, 质量为$m$的物体用细绳水平拉住,静止在倾角为$\theta$的固定的光滑斜面上,求斜面给物体的支持力.
    \begin{figure}[h]
        \centering
        \includegraphics[width=0.15\textheight]{fig17}
            \caption{如图}\label{fig:17}
    \end{figure}

     \item 如图 \ref{fig:18} ,水平地面上放一物体$A$, 它与地面间的滑动摩擦系数为$\mu$.现加一恒力$\vec{F}$如图所示. 欲使物体$A$有最大加速度,求恒力$\vec{F}$与水平方向夹角$\theta$.
     \begin{figure}[H]
        \centering
        \includegraphics[width=0.15\textheight]{fig18}
            \caption{如图}\label{fig:18}
    \end{figure}
    \item 一质点从静止出发沿半径$R=1m$的圆周运动,其角加速度随时间$t$的变化规律是 (SI), 求质点从出发到$t$时刻走过的路程$S$.
        
\end{enumerate}

\section{习题参考答案}
\subsection*{一、填空题}
\begin{enumerate}
    \item 升降机内地板上放有物体$A$, 其上再放另一物体$B$, 二者的质量分别为$M_A$、$M_B$.当升降机以加速度$a$向下加速运动时$(a<g)$, 物体$A$对升降机地板的压力大小$N=\anl{$(M_A+M_B)(g-a)$}$.  
    \item 如图所示 \ref{Fig:12} , 一个小物体$A$靠在一辆小车的竖直前壁上,$A$和车壁间静摩擦系数是$μ_s$,若要使物体$A$不致掉下来,小车的加速度的最小值应为$a=\anl{$g/ \mu _s$}$.
    \begin{figure}[h]
        \centering
        \includegraphics[width=0.15\textheight]{fig12}
            \caption{如图}\label{Fig:12}
    \end{figure}
    \item 如图 \ref{Fig:13} , 一圆锥摆摆长为$l$、摆锤质量为$m$,在水平面上作匀速圆周运动,摆线与铅直线夹角$\theta$,则摆线的张力$T=\anl{$\frac{mg}{\mathrm{cos}\theta}$}$.摆锤的速率$v=\anl{$\sqrt{\frac{gl\mathrm{sin}^2\theta}{\mathrm{cos}\theta}}$}$.
    \begin{figure}[h]
        \centering
        \includegraphics[width=0.15\textheight]{fig13}
            \caption{如图}\label{Fig:13}
    \end{figure}
\end{enumerate}
\subsection*{二、选择题}
\begin{enumerate}
    \item  如图 \ref{Fig:14} , 一只质量为$m$的猴,原来抓住一根用绳吊在天花板上的质量为$M$的直杆,悬线突然断开,小猴则沿杆子竖直向上爬以保持它离地面的高度不变,此时直杆下落的加速度为( D )
    \fourch{$g$ ;}{$\frac{m}{M}g$ ;}{$\frac{M+m}{M-m}g$ ;}{$\frac{M+m}{M}G$ .}
    \begin{figure}[H]
        \centering
        \includegraphics[width=0.06\textheight]{fig14}
            \caption{如图}\label{Fig:14}
    \end{figure}

    \begin{note}
        \textcolor{red}{先受力分析, 再计算}
        \begin{figure}[H]
            \begin{minipage}[ht]{0.6\linewidth}
                \begin{table}[H]
                    \begin{tabular}{l}
                       \qquad 计算过程:\\
                       \qquad \qquad $F=mg$ \\
                        \qquad \qquad   $F+Mg = Ma$ \\
                        \qquad \qquad $a = \frac{M+m}{M}g$\\ 
                    \end{tabular}
                \end{table}  
            \end{minipage}
            \begin{minipage}[h]{0.3\linewidth}
                \includegraphics[width=0.06\textheight]{ans8}
            \end{minipage}
        \end{figure}
    \end{note}
    \item 一段路面水平的公路,转弯处轨道半径为$R$,汽车轮胎与路面间的摩擦系数为$\mu$,要使汽车不致于发生侧向打滑,汽车在该处的行驶速率( B )
    \twoch{不得小于$\sqrt{\mu gR}$ ;}{不得大于$\sqrt{\mu gR}$ ;}{必须等于$\sqrt{2gR}$ ;}{还应由汽车的质量$M$决定 .}
    \begin{note}
        \textcolor{red}{对车子受力分析即可:}
        \begin{figure}[H]
            \centering
            \includegraphics[width=0.30\textheight]{ans9}
        \end{figure}
    \end{note}
    \item 质量分别为$m_1$和$m_2$的两滑块$A$和$B$通过一轻弹簧水平连结后置于水平桌面上,滑块与桌面间的摩擦系数均为$\mu$, 系统在水平拉力$F$作用下匀速运动,如图所示\ref{Fig:15}。如突然撤消拉力,则刚撤消后瞬间,二者的加速度$a_A$和$a_B$分别为( D ) 
    \begin{figure}[ht]
        \begin{minipage}[ht]{0.4\linewidth}
           \begin{table}[H]
               \begin{tabular}{c}
                  \qquad   (A)\ $a_A=0, a_B=0$;\\
                  \qquad  (B)\ $a_A>0, a_B<0$;\\
                  \qquad  (C)\ $a_A<0, a_B>0$;\\
                  \qquad  (D)\ $a_A<0, a_B=0$.
               \end{tabular}
           \end{table}
        \end{minipage}
        \begin{minipage}[h]{0.5\linewidth}
            \includegraphics[width=0.25\textheight]{fig15}
            \caption{如图}\label{Fig:15}
        \end{minipage}
    \end{figure}
    \begin{note}
        \textcolor{red}{对物体受力分析: }
        \begin{figure}[H]
            \centering
            \includegraphics[width=0.30\textheight]{ans10}
        \end{figure}
    \end{note}
\end{enumerate}
\subsection*{三、计算题}
\begin{enumerate}
    \item 质量为$m$的质点沿$x$轴正向运动: 设质点通过坐标点为$x$时的速度为$v=kx$($k$为常数), 求作用在质点的合外力及质点从$x=x_0$到$x=2x_0$处所需的时间$t$.
    \begin{solution}
        $v = kx$, 加速度 $a = \frac{\mathrm{d}v}{\mathrm{d}t}=\frac{k\mathrm{d}x}{\mathrm{d}t}=kv=k^2x$\\
        $\therefore F = ma = mk^2x $\ \ $\therefore v = kx \Longrightarrow \frac{\mathrm{d}x}{\mathrm{d}t}=kx$\\ 
        $\therefore \displaystyle{\int_{x_0}^{2x_0}\frac{\mathrm{d}x}{x}=\int_0^t k\mathrm{d}t}$ \  $\therefore t = \frac{\mathrm{ln2}}{k}$.
    \end{solution}
    \item 如图 \ref{Fig:16}, 在光滑水平桌面上,有两个物体$A$和$B$紧靠在一起.它们的质量分别为$m_A=2kg$, $m_B=1kg$. 今用一水平力$F=3N$推物体$B$,求$B$推$A$的力$f$. 
    \begin{figure}[H]
        \centering
        \includegraphics[width=0.15\textheight]{fig16}
            \caption{如图}\label{Fig:16}
    \end{figure}
    \begin{solution}
        看图: 
        \begin{figure}[H]
            \centering
            \includegraphics[width=0.38\textheight]{ans12}
        \end{figure}
    \end{solution}
    \item 如图所示 \ref{Fig:17}, 质量为$m$的物体用细绳水平拉住,静止在倾角为$\theta$的固定的光滑斜面上,求斜面给物体的支持力.
    \begin{figure}[H]
        \centering
        \includegraphics[width=0.15\textheight]{fig17}
            \caption{如图}\label{Fig:17}
    \end{figure}
    \begin{solution}
        看图: 
        \begin{figure}[H]
            \centering
            \includegraphics[width=0.25\textheight]{ans11}
        \end{figure}
    \end{solution}

     \item 如图 \ref{Fig:18} ,水平地面上放一物体$A$, 它与地面间的滑动摩擦系数为$\mu$.现加一恒力$\vec{F}$如图所示. 欲使物体$A$有最大加速度,求恒力$\vec{F}$与水平方向夹角$\theta$.
     \begin{figure}[H]
        \centering
        \includegraphics[width=0.15\textheight]{fig18}
            \caption{如图}\label{Fig:18}
    \end{figure}
    \begin{solution}
        看图: 
        \begin{figure}[H]
            \centering
            \includegraphics[width=0.38\textheight]{ans13}
        \end{figure}
    \end{solution}
    \item 一质点从静止出发沿半径$R=1m$的圆周运动,其角加速度随时间$t$的变化规律是 $\beta = 12t^2-6t$ (SI), 求质点从出发到$t$时刻走过的路程$S$.
    \begin{solution}

        \begin{figure}[ht]
            \begin{minipage}[ht]{0.7\linewidth}
                \begin{table}[H]
                    \begin{tabular}{l}
                       \qquad $\beta = \frac{\mathrm{d}\omega}{\mathrm{d}t}\Longrightarrow \mathrm{d}w=\beta \mathrm{d}t$, $\displaystyle{\int_0^\omega \mathrm{d}w = \int_0^t(12t^2-6t)\mathrm{d}t}$\\
                        
                       \qquad 角速度 $w = 4t^3-3t^2$, 令$\omega = 0$, 得$t_1=\frac{3}{4}s$. \\
                       \qquad 又$\because \omega = \frac{\mathrm{d}\theta}{\mathrm{d}t}$\ \ $\displaystyle{\int_0^{\theta}\mathrm{d}\theta = \displaystyle{\int_0^t(4t^3-3t^2)\mathrm{d}t}}$, 
                $\theta = t^4-t^3$
                    \end{tabular}
                \end{table}
            \end{minipage}
            \begin{minipage}[h]{0.25\linewidth}
                \includegraphics[width=0.13\textheight]{ans14}
            \end{minipage}
        \end{figure}
        \begin{enumerate}
            \item[1)] 当$t<(3/4)s$时\\
            质点角坐标为负(顺时针转)对应路程: $S=|R\theta|=(t^3-t^4) m$
            \item[2)] 当$t>(3/4)s$时\\
            质点开始逆时针转, 路程由开始两部分组成 $S=R(\theta-\theta_1)+R|\theta_1|= (t^4-t^3+0.2) m$
        \end{enumerate}
    \end{solution}    
\end{enumerate}
    \chapter{级数}
\section{级数敛散判别法与性质、上极限与下极限}

\section{函数级数}

\section{幂函数}

\section{傅氏级数的收敛性、平均收敛与一致收敛}
    \chapter{动量能量习题课}
\section{作业习题}
\subsection*{一、填空题}

\begin{enumerate}
    \item 图中 \ref{fig:29} 沿着半径为$R$圆周运动的质点, 所受的几个力中有一个是恒力$\vec{F_0}$,方向始终沿$x$轴正向, 
    即, 当质点从$A$点沿逆时针方向走过$3/4$圆周到达$B$点时, 力$\vec{F_0}$所作的功为$A=\nl$.
    \begin{figure}[H]
        \centering
        \includegraphics[width=0.15\textheight]{fig29}
            \caption{如图}\label{fig:29}
    \end{figure}
    \item 一个力$F$作用在质量为$2.0 kg$的质点上, 使之沿$x$轴运动. 已知在此力作用下质点的运动学方程
    为$x=3t-4t^2+t^3$(SI). 在0到$4s$的时间间隔内, 力$F$的冲量大小$I=\nl$.
    \item  质量为$m$的物体, 置于电梯内, 电梯以$g/3$加速度, 匀加速下降$h$,在此过程中, 
    电梯对物体的作用力所做的功$A=\nl$.
\end{enumerate}
\subsection*{二、选择题}
\begin{enumerate}
    \item 一水平放置的轻弹簧, 劲度系数为$k$, 其一端固定, 另一端系一质量为$m$的滑块$A$, 
    $A$旁又有一质量相同的滑块$B$, 如图所示\ref{fig:30}.设两滑块与桌面间无摩擦, 若用外力将$A$、$B$一起推压使弹簧压缩量为$d$而静止, 然后撤消外力, 则$B$离开时的速度为(\hspace{1pc})
    \fourch{$0$;}{$d\sqrt{\frac{k}{2m}}$ ;}{$d\sqrt{\frac{k}{m}}$;}{$d\sqrt{\frac{2k}{m}}$.}
    \begin{figure}[H]
        \centering
        \includegraphics[width=0.15\textheight]{fig30}
            \caption{如图}\label{fig:30}
    \end{figure}
    \item 有一劲度系数为$k$的轻弹簧, 原长为$l_0$, 将它吊在天花板上. 当它下端挂一托盘平衡时, 其长度变为$l_1$. 然后在托盘中放一重物, 弹簧长度变为$l_2$, 则由$l_1$伸长至
    $l_2$的过程中, 弹性力所作的功为(\hspace{1pc})
    \fourch{$-\displaystyle{\int_{l_1}^{l_2}kx\mathrm{d}x}$;}{$\displaystyle{\int_{l_1}^{l_2}kx\mathrm{d}x}$;}
    {$\displaystyle{\int_{l_1-l_0}^{l_2-l_0}-kx\mathrm{d}x}$;}{$\displaystyle{\int_{l_1-l_0}^{l_2-l_0}kx\mathrm{d}x}$.}
    \item 质量分别为$m_1$、$m_2$的两个物体用一劲度系数为$k$的轻弹簧相联, 放在水平光滑桌面上. 当两物体相距$x$时, 系统由静止释放. 已知弹簧的自然长度为$x_0$, 则当物体相距$x_0$时, $m_1$的速度大小为(\hspace{1pc})
    \twoch{$\sqrt{\frac{k(x-x_0)^2}{m_1}}$;}{$\sqrt{\frac{k(x-x_0)^2}{m_2}}$;}{$\sqrt{\frac{k(x-x_0)^2}{m_1+m_2}}$;}{$\sqrt{\frac{k(x-x_0)^2}{m_1(m_1+m_2)}}$.}

\end{enumerate}
\subsection*{三、计算题}
\begin{enumerate}
    \item 如图 \ref{fig:31} ,用一弹簧把质量各为$m_1$和$m_2$的两木块连起来, 一起放在地面上, 弹簧的质量可不计, 而$m_2>m1$, 问:
    \begin{enumerate}
        \item[(1)] 对上面的木块必须施加多大的压力F, 以便在F突然撤去而上面的木块跳起来时, 恰能使下面的木块提离地面?
        \item[(2)] 如果$m_1$和$m_2$互换位置, 结果有无改变?
    \end{enumerate}
    \begin{figure}[H]
        \centering
        \includegraphics[width=0.15\textheight]{fig31}
            \caption{如图}\label{fig:31}
    \end{figure}
    \item 两个质量分别为$m_1$和$m_2$的木块$A$和$B$, 用一个质量忽略不计、劲度系数为$k$的弹簧联接起来, 放置在光滑水平面上, 使$A$紧靠墙壁, 如图所示\ref{fig:32}. 用力推木块$B$使弹簧压缩$x_0$,
    然后释放. 已知$m_1=m$, $m_2 = 3m$, 求:
    \begin{enumerate}
        \item[(1)] 释放后, 弹簧恢复到原长时$B$木块的速度为多大?
        \item[(2)] 释放后, $A$、$B$两木块速度相等时的瞬时速度的大小.
        \item[(3)] 释放后, 弹簧的最大伸长量.
    \end{enumerate}
    \begin{figure}[H]
        \centering
        \includegraphics[width=0.15\textheight]{fig32}
            \caption{如图}\label{fig:32}
    \end{figure}
    \item 如图 \ref{fig:33}, 质量$m=0.1kg$的小球, 拴在长度$L=0.5m$的轻绳的一端, 构成摆, 摆动时与竖直的最大夹角为$60^\circ$. 求: 在$\theta<60^\circ$的任一位置, 求小球速度$V$与$\theta$的关系式, 
    这时小球的加速度为何? 绳的张力为多大?
    \begin{figure}[H]
        \centering
        \includegraphics[width=0.13\textheight]{fig33}
            \caption{如图}\label{fig:33}
    \end{figure}
\end{enumerate}

\section{习题参考答案}
\subsection*{一、填空题}

\begin{enumerate}
    \item 图中 \ref{Fig:29} 沿着半径为$R$圆周运动的质点, 所受的几个力中有一个是恒力$\vec{F_0}$,方向始终沿$x$轴正向, 
    即, 当质点从$A$点沿逆时针方向走过$3/4$圆周到达$B$点时, 力$\vec{F_0}$所作的功为$A=\anl{$-F_0R$}$.
    \begin{figure}[H]
        \centering
        \includegraphics[width=0.15\textheight]{fig29}
            \caption{如图}\label{Fig:29}
    \end{figure}
    \begin{note}
        \textcolor{red}{$\mathrm{d}A=\vec{F}\cdot \mathrm{d}\vec{r}=F_0]mathrm{d}x$, $A = \displaystyle{\int_{0}^{-R}F_0\mathrm{d}x = -F_0R}$}
    \end{note}
    \item 一个力$F$作用在质量为$2.0 kg$的质点上, 使之沿$x$轴运动. 已知在此力作用下质点的运动学方程
    为$x=3t-4t^2+t^3$(SI). 在0到$4s$的时间间隔内, 力$F$的冲量大小$I=\anl{$32 N\cdot s$}$.
    \begin{note}
        \textcolor{red}{$V=\frac{\mathrm{d}x}{\mathrm{d}t=3-8t+3t^2}$ $\Longrightarrow$ $t=0$时, $V(0)=3$, $t=4$时, $V(4)=19$, $I=\Delta p = m\Delta V = 32 N\cdot s$}
    \end{note}
    \item  质量为$m$的物体, 置于电梯内, 电梯以$g/3$加速度, 匀加速下降$h$,在此过程中, 
    电梯对物体的作用力所做的功$A=\anl{$\frac{-2mgh}{3}$}$.
    \begin{note}
        看图, 懒得自己写了(其实自己写得不咋样):
    \end{note}
    \begin{figure}[H]
        \centering
        \includegraphics[width=0.25\textheight]{ans19}
    \end{figure}
\end{enumerate}

\subsection*{二、选择题}
\begin{enumerate}
    \item 一水平放置的轻弹簧, 劲度系数为$k$, 其一端固定, 另一端系一质量为$m$的滑块$A$, 
    $A$旁又有一质量相同的滑块$B$, 如图所示\ref{Fig:30}.设两滑块与桌面间无摩擦, 若用外力将$A$、$B$一起推压使弹簧压缩量为$d$而静止, 然后撤消外力, 则$B$离开时的速度为( B )
    \fourch{$0$;}{$d\sqrt{\frac{k}{2m}}$ ;}{$d\sqrt{\frac{k}{m}}$;}{$d\sqrt{\frac{2k}{m}}$.}
    \begin{figure}[h]
        \centering
        \includegraphics[width=0.15\textheight]{fig30}
            \caption{如图}\label{Fig:30}
    \end{figure}
    \begin{note}
        看图, 懒得自己写了(其实自己写得不咋样):
    \end{note}
    \begin{figure}[H]
        \centering
        \includegraphics[width=0.25\textheight]{ans20}
    \end{figure}
    \item 有一劲度系数为$k$的轻弹簧, 原长为$l_0$, 将它吊在天花板上. 当它下端挂一托盘平衡时, 其长度变为$l_1$. 然后在托盘中放一重物, 弹簧长度变为$l_2$, 则由$l_1$伸长至
    $l_2$的过程中, 弹性力所作的功为( C )
    \fourch{$-\displaystyle{\int_{l_1}^{l_2}kx\mathrm{d}x}$;}{$\displaystyle{\int_{l_1}^{l_2}kx\mathrm{d}x}$;}
    {$\displaystyle{\int_{l_1-l_0}^{l_2-l_0}-kx\mathrm{d}x}$;}{$\displaystyle{\int_{l_1-l_0}^{l_2-l_0}kx\mathrm{d}x}$.}
    \begin{note}
        看图, 懒得自己写了(其实自己写得不咋样):
    \end{note}
    \begin{figure}[H]
        \centering
        \includegraphics[width=0.25\textheight]{ans21}
    \end{figure}
    \item 质量分别为$m_1$、$m_2$的两个物体用一劲度系数为$k$的轻弹簧相联, 放在水平光滑桌面上. 当两物体相距$x$时, 系统由静止释放. 已知弹簧的自然长度为$x_0$, 则当物体相距$x_0$时, $m_1$的速度大小为( D )
    \twoch{$\sqrt{\frac{k(x-x_0)^2}{m_1}}$;}{$\sqrt{\frac{k(x-x_0)^2}{m_2}}$;}{$\sqrt{\frac{k(x-x_0)^2}{m_1+m_2}}$;}{$\sqrt{\frac{k(x-x_0)^2}{m_1(m_1+m_2)}}$.}
    \begin{note}
        \textcolor{red}{能量守恒: $\frac{1}{2}k(x-x_0)^2=\frac{1}{2}m_1V_1^2+\frac{1}{2}m_2V_2^2$}\\
        \textcolor{red}{动量守恒: $0=m_1V_1+m_2V_2$ 解出: $V_1=\sqrt{\frac{km_2(x-x_0)^2}{m_1(m_1+m_2)}}$}
    \end{note}
\end{enumerate}
\subsection*{三、计算题}
\begin{enumerate}
    \item 如图 \ref{Fig:31} ,用一弹簧把质量各为$m_1$和$m_2$的两木块连起来, 一起放在地面上, 弹簧的质量可不计, 而$m_2>m1$, 问:
    \begin{enumerate}
        \item[(1)] 对上面的木块必须施加多大的压力F, 以便在F突然撤去而上面的木块跳起来时, 恰能使下面的木块提离地面?
        \item[(2)] 如果$m_1$和$m_2$互换位置, 结果有无改变?
    \end{enumerate}
    \begin{figure}[H]
        \centering
        \includegraphics[width=0.15\textheight]{fig31}
            \caption{如图}\label{Fig:31}
    \end{figure}
    \begin{solution}
        看图: 
        \begin{figure}[H]
            \centering
            \includegraphics[width=0.48\textheight]{ans22}
        \end{figure}
    \end{solution}

    \item 两个质量分别为$m_1$和$m_2$的木块$A$和$B$, 用一个质量忽略不计、劲度系数为$k$的弹簧联接起来, 放置在光滑水平面上, 使$A$紧靠墙壁, 如图所示\ref{Fig:32}. 用力推木块$B$使弹簧压缩$x_0$,
    然后释放. 已知$m_1=m$, $m_2 = 3m$, 求:
    \begin{enumerate}
        \item[(1)] 释放后, 弹簧恢复到原长时$B$木块的速度为多大?
        \item[(2)] 释放后, $A$、$B$两木块速度相等时的瞬时速度的大小.
        \item[(3)] 释放后, 弹簧的最大伸长量.
    \end{enumerate}
    \begin{figure}[H]
        \centering
        \includegraphics[width=0.15\textheight]{fig32}
            \caption{如图}\label{Fig:32}
    \end{figure}
    \begin{solution}
        \begin{enumerate}
            \item[(1)] 设弹簧恢复到原长时滑块$B$的速度为$V_{B0}$, 由机械能守恒得: $\frac{1}{2}kx_0^2=\frac{m_2V_{B0}^2}{2}$,\  $\therefore V_{B0}=x_0\sqrt{k/3m}$.
            \item[(2)]  $A$块离墙后: $m_1v_1+m_2v_2=m_2V_{B0}$, $v_1=v_2=v$时, $mv+3mv=3mV_{B0}$, $v=\frac{3}{4}V_{B0}=\frac{3}{4}x_0\sqrt{\frac{k}{3m}}$.
            \item[(3)] $\frac{1}{2}kx_0^2=\frac{(m_1+m_2)v^2}{2}+\frac{1}{2}kx^2$, \ \ $x=\frac{x_0}{2}$ 
        \end{enumerate}
    \end{solution}
    \item 如图 \ref{Fig:33}, 质量$m=0.1kg$的小球, 拴在长度$L=0.5m$的轻绳的一端, 构成摆, 摆动时与竖直的最大夹角为$60^\circ$. 求: 在$\theta<60^\circ$的任一位置, 求小球速度$V$与$\theta$的关系式, 
    这时小球的加速度为何? 绳的张力为多大?
    \begin{figure}[H]
        \centering
        \includegraphics[width=0.13\textheight]{fig33}
            \caption{如图}\label{Fig:33}
    \end{figure}
    \begin{solution}
        摆动中机械能守恒: $-mgL\mathrm{cos}60^\circ = -mgL\mathrm{cos}\theta + \frac{mv^2}{2}$, $v=\sqrt{2gL(\mathrm{cos\theta-1/2})}$, $T-mg\mathrm{cos}\theta=ma_n=m\frac{v^2}{L}$, 带入$v$得
        $T=mg\mathrm{cos}\theta+2mg(\mathrm{cos}\theta-\frac{1}{2})=3mg\mathrm{cos}\theta-mg$.  $a_n=\frac{v^2}{L}=2g(\mathrm{cos}\theta-1/2)$, 
        $mg\mathrm{sin}\theta = ma_{\tau} \Longrightarrow a_{\tau} = g\mathrm{sin}\theta$, $\therefore \vec{a}=a_n\vec{n}+a_{\tau}\vec{\tau}$. $\vec{a}=2g(\mathrm{cos}\theta-1/2)\vec{n}+g\mathrm{sin}\vec{\tau}$.
    \end{solution}
\end{enumerate}
    \chapter{多元函数积分学}
\section{重积分的概念与性质、重积分化累次积分}
\centering{\textbf{练习题}}
\begin{enumerate}
	\item 试求$\bm{R}^2$中点集$E=\{(x,y)|0\le x\le1,0\le y\le 1, x\text{和}y\text{至少有一为有理数}\}$的内容度和外容度. 问$E$是否是可测图形?
	\item 设$A,B,C$是$\bm{R}^m$中的可测图形, 证明:
	\begin{enumerate}
		\item $V(A\backslash B)=V(A)-V(A\cap B)$;
		\item $V(A\cup B)=V(A)+V(B)-V(A\cap B)$;
		\item $V(A\cup B\cup C)=V(A)+V(B)+V(C)-V(A\cap B)-V(A\cap C)-V(B\cap C)+V(A\cup B\cup C)$.
	\end{enumerate}
\item 举例说明$\bm{R}^m$中两个点集$E_1$和$E_2$都不是可测函数, 但是$E_1\cup E_2, E_1\cap E_2$都是可测函数.问是否还可能有$E_1\backslash E_2$也是可测图形.
\item 设$\Omega$为$\bm{R}^m$中一可测图形.证明: $\Omega^\circ$和$\overline{\Omega}$为可测图形, 且$V(\Omega^\circ)=V(\Omega)=V(\overline{\Omega})$.
\item 在$\bm{R}^2$的区域$D=\{(x,y)||x|\le 1,|y|\le 1\}$上给定函数
$$ f(x,y)=\begin{cases}
1,\qquad & \text{当}x,y\text{都是有理数}, \\
2,\qquad & \text{当}x,y\text{当}x,y\text{至少有一是无理数}.
\end{cases}$$
问$f(x,y)$是否在$D$上可积.
\item 设$\bm{R}^m$中的开集$\Omega$为可测图形, $f:\Omega\rightarrow \bm{R},f\in C(\Omega)$, 且$f(\bm{x})\ge 0\ (\bm{x}\in \Omega)$, 但不恒为零. 证明: $\displaystyle{\int_{\Omega}f(\bm{x})\mathrm{d}V}>0$. 如果$\Omega$不是开集, 上述论证是否正确? 举例说明.
\item 设定义在可测图形$\Omega \subset \bm{R}^m$上的两个函数$f,g$ 有界、可积, 而且$g(\bm{x})$在$\Omega$上之值非负. 令$$
m=\underset{x\in \Omega}{\mathrm{inf}\{f(\bm{x})\}},\ M = \underset{x\in \Omega}{\mathrm{sup}}\{f(\bm{x})\}.$$
证明:
\begin{enumerate}
	\item $F(t)=\displaystyle{\int_{\Omega}[f(\bm{x}-t)]g(\bm{x})\mathrm{d}V}$是$[m,M]$上的连续函数;
	\item 存在$\mu\in [m,M]$, 使得
	$$ \displaystyle{\int_{\Omega}f\cdot g\mathrm{d}V=\mu\cdot \int_{\Omega}g\mathrm{d}V}.$$
	\item 设$f(x)\in R[-1,1]$, 证明: $f(x-y)\in R([0\times1]\times[0,1])$.
	\item 设$\Omega \subset \bm{R}^m$为测度图形, $Q$为长方体, $\Omega \subset Q^\circ,f(\bm{x})\in R(\Omega)$. 定义$$
	F(\bm{x}) = \begin{cases}
	f(\bm{x}),\qquad &\bm{x}\in \Omega\\
	0,\qquad \bm{x}\in Q\backslash \Omega.
	\end{cases}$$
	求证: $F(\bm{x})\in R(Q)$.
	
\end{enumerate}
\item 设$\Omega$为$\bm{R}^m$中点集, $Q$为长方体, $\Omega \subset Q^\circ$.定义函数
$$ \chi(\bm{x})=\begin{cases}
1,\qquad &\bm{x} \in \Omega\\
0,\qquad &\bm{x} \in \Omega\backslash \Omega.
\end{cases}$$
若$\chi(\bm{x})$在$Q$上可积, 证明: $\Omega$为可测图形.
\item 在下列积分中改变积分的顺序:
\begin{table}[H]
	\begin{tabular}{ll}
		\qquad(1)\ $\displaystyle{\int_{0}^{3}\mathrm{d}x\int_{0}^{\mathrm{ln}x}f(x,y)\mathrm{d}y}$.\qquad \qquad \qquad \qquad \qquad &(2)\ $\displaystyle{\int_{0}^{2}\mathrm{d}y\int_{y^2}^{3y}f(x,y)\mathrm{d}x}$.\\
		\qquad(3)\ $\displaystyle{\int_{-1}^{1}\mathrm{d}x\int_{-\sqrt{1-x^2}}^{1-x^2}f(x,y)\mathrm{d}y}$;\qquad \qquad \qquad \qquad \qquad &(4)\ $\displaystyle{\int_{1}^{2}\mathrm{d}x\int_{\sqrt{x}}^{2}f(x,y)\mathrm{d}y}$.
	\end{tabular}
\end{table}
\item 计算下列二重积分:
\begin{enumerate}
	\item $\Omega$是由$y^2=2px\ (p>0)$与$x=\frac{p}{2}$围成的区域, 求
	$$ \displaystyle{\underset{\Omega}{\iint }x^my^k\mathrm{d}x\mathrm{d}y}\ \ (m>0,k\text{为正整数});$$
	\item $\Omega=\{(x,y)|0\le x\le y^2,0\le y \le 2+x,x\le 2\}$, 求$\displaystyle{\underset{\Omega}{\iint}xy\mathrm{d}x\mathrm{d}y}$;
	\item $\Omega$是由$y=\sqrt{1-x^2},y=0$围成, 求$\displaystyle{\underset{\Omega}{\iint}(x^2+3xy^2)\mathrm{d}x\mathrm{d}y}$;
	\item $\Omega$是由$y=\mathrm{e}^x, y=1,x=0$及$x=1$围成, 求$\displaystyle{\underset{\Omega}{\iint}(x+y)\mathrm{d}x\mathrm{d}y}$;
	\item $\Omega$是以$(1,1),(2,3),(3,1)$和$(4,3)$为顶点的四边形, 求$$
	\displaystyle{\underset{\Omega}{\iint}(x+y)\mathrm{d}x\mathrm{d}y};$$
	\item $\Omega$是由$y=x^2,y=4x$和$y=4$围成, 求$\displaystyle{\underset{\Omega}{\iint}\mathrm{sin}x\mathrm{d}x\mathrm{d}y}$.
\end{enumerate}
\item 计算下列积分:
\begin{table}[H]
	\begin{tabular}{ll}
		\qquad	(1)\ $\displaystyle{\int_{0}^{\frac{\pi}{2}}\mathrm{d}y\int_{y}^{\frac{\pi}{2}}\frac{\mathrm{sin}x}{x}\mathrm{d}x}$;\qquad \qquad \qquad \qquad &(2)\ $\displaystyle{\int_{0}^{1}\mathrm{d}y\int_{y}^{1}\mathrm{e}^{-x^2}\mathrm{d}x}$.
	\end{tabular}
\end{table}
\item 设在$D=[a,b]\times[c,d]$上定义的二元函数$f(x,y)\in C^2(D)$, 证明:
\begin{enumerate}
		\item $\displaystyle{\underset{D}{\iint}f''_{xy}(x,y)\mathrm{d}x\mathrm{d}y}=\displaystyle{\underset{D}{\iint}f''_{yx}(x,y)\mathrm{d}x\mathrm{d}y}$;
		\item 利用(1)证明$f''_{xy}(x,y)=f''_{yx}(x,y),(x,y)\in D$(这里不准用偏导与秩序无关定理).
\end{enumerate}
\item 设$f(x),g(x)\in R[a,b], D=[a,b]\times[a,b]$, 考虑$[f(x)g(y)-g(x)f(y)]^2$在$D$上的重积分, 证明:
$\left(\displaystyle{\int_{a}^{b}f(x)g(x)\mathrm{d}x} \right)^2\le \displaystyle{\int_{a}^{b}f^2(x)\mathrm{d}x\cdot \int_{a}^{b}g^2(x)\mathrm{d}x}$.
\item 求下列立体$\Omega$的体积:
\begin{enumerate}
	\item $\Omega$是由曲线$z=xy,x+y+z=1$和$z=0$围成;
	\item $\Omega$是由$y^2+z^2=1,|x+y|=1,|x-y|=1$围成.
\end{enumerate}
\item 证明: 若$b>a>0$, 则有
\begin{table}[H]
	\begin{tabular}{ll}
	\qquad	(1)\ $\lim\limits_{T\rightarrow \infty}\displaystyle{\int_{0}^{T}\mathrm{d}x\int_{a}^{b}\mathrm{e}^{-xy}\mathrm{d}y=\mathrm{ln}\frac{b}{a}}$;\qquad \qquad \qquad \qquad & (2)\ $\displaystyle{\int_{0}^{+\infty}\frac{\mathrm{e}^{-ax}-\mathrm{e}^{-bx}}{x}\mathrm{d}x=\mathrm{ln}\frac{b}{a}}$.
	\end{tabular}
\end{table}
\item 设$f(t)$在$t\ge 0$上连续可微, 而且$\displaystyle{\int_{0}^{+\infty}\frac{f(t)}{t}\mathrm{d}t}$收敛. 证明:当$b>a>0$时, 有
\begin{enumerate}
	\item $\lim\limits_{T\rightarrow \infty}\int_{0}^{T}\mathrm{d}x\int_{a}^{b}f'(xy)\mathrm{d}y=-f(0)\mathrm{ln}\frac{b}{a}$;
	\item $\displaystyle{\int_{0}^{+\infty}\frac{f(ax)-f(bx)}{x}\mathrm{d}x=f(0)\mathrm{ln}\frac{b}{a}}$.
\end{enumerate}
\item 设$f(x,y)$在$x^2+y^2\le R^2$上可积, $0<h<R$, 令
$$	F(\xi,\eta) = \underset{(x-\xi)^2+(y-\eta)^2\le h^2}{\iint}f(x,y)\mathrm{d}x\mathrm{d}y.$$
证明:$F(\xi,\eta)$在$\xi^2+\eta^2\le (R-h)^2$上连续.
\item 证明下列三重积分化为累次积分的顺序(只写出$\mathrm{d}x,\mathrm{d}z$互换的顺序):
\begin{table}[H]
	\begin{tabular}{ll}
	\qquad	(1)\ $\displaystyle{\int_{0}^{1}\mathrm{d}x\int_{0}^{1-x}\mathrm{d}y\int_{0}^{x+y}}\mathrm{d}z$;\qquad \qquad \qquad \qquad &(2)\ $\displaystyle{\int_{-1}^{1}\mathrm{d}x\int_{-\sqrt{1-x^2}}^{\sqrt{1-x^2}}\mathrm{d}y\int_{\sqrt{x^2+y^2}}^{1}f\mathrm{d}z}$.
	\end{tabular}
\end{table}
\item 计算下列三重积分:
\begin{enumerate}
		\item $\displaystyle{\underset{\Omega}{\iiint}xy^2z^3\mathrm{d}x\mathrm{d}y\mathrm{d}z}$,\ $\Omega$是由曲面$z=xy,y=x,x=1,z=0$所围成;
		\item $\displaystyle{\underset{\Omega}{\iiint}\frac{\mathrm{d}x\mathrm{d}y\mathrm{d}z}{(1+x+y+z)^3}}$, $\Omega$是由曲面$x+y+z=1,x=0,y=0,z=0$所围成;
		\item $\displaystyle{\underset{\Omega}{\iiint}\mathrm{cos}az\mathrm{d}x\mathrm{d}y\mathrm{d}z}$, $\Omega:x^2+y^2+z^2\le R^2$;
		\item $\displaystyle{\underset{\Omega}{\iiint}(1+x^4)\mathrm{d}x\mathrm{d}y\mathrm{d}z}$, $\Omega$是由曲面$x^2=y^2+z^2,x=2,x=1$所围成.
\end{enumerate}
\item 计算三重积分
$$I=\displaystyle{\int_{0}^{1}\mathrm{d}x\int_{x}^{1}\mathrm{d}y\int_{y}^{1}y\sqrt{1+z^4}\mathrm{d}z}.$$
\end{enumerate}
\section{重积分的变换}

\section{曲线积分与格林公式}

\section{曲面积分}

\section{奥氏积分、斯托克斯公式、线积分与路径无关}

\section{场论}
    \chapter{典型综合题分析}
    \chapter{静电场}
\section{作业习题}
\subsection*{一、填空题}
\begin{enumerate}
    \item 边长为$a$的正方形的四个顶点上放置如图所示 \ref{fig:46} 的点电荷, 则中心$o$处场强为\nl.
    \begin{figure}[H]
        \centering
        \includegraphics[width=0.15\textheight]{fig46}
        \caption{如图.}\label{fig:46}
    \end{figure}
    \item 在点电荷系的电场中, 任一点的电场强度等于\nl, 这称为电场强度叠加原理.
    \item 正方形的两对角上, 各置电荷$Q$, 在其余两对角上各置电荷$q$, 若$Q$所受合力为零,则$Q$与$q$的大小关系为$\nl$.
    \item   如图所示 \ref{fig:49}, 真空中有两个点电荷, 带电量分别为$Q$和$-Q$, 相距$2R$。若以负电荷
    所在处$O$点为中心, 以$R$为半径作高斯球面$S$, 则通过该球面的电场强度通量$\varphi_e=\nl$. 
    \begin{figure}[H]
        \centering
        \includegraphics[width=0.15\textheight]{fig49}
        \caption{如图.}\label{fig:49}
    \end{figure}
    \item 如图所示\ref{fig:50}, 在场强为$E$的均匀电场中取一半球面, 其半径为$r$, 电场强度的方向与半球面的对称轴平行. 则通过这个半球面的电通量$\varphi_e=\nl$.
    \begin{figure}[H]
        \centering
        \includegraphics[width=0.15\textheight]{fig50}
        \caption{如图.}\label{fig:50}
    \end{figure}
    \item 一点电荷$q$位于一位立方体中心, 立方体边长为$a$, 则通过立方体每个表面的$\vec{E}$的通量\nl;若把这电荷移到立方体的一个顶角上, 这时通过电荷所在顶角的三个面$\vec{E}$的通量是\nl, 通过立方体另外三个面的$\vec{E}$的通量是\nl.
    \item 一均匀静电场, 场强$\vec{E}=(400\vec{i}+600\vec{j})v\cdot m^{-1}$, 则点$a(3, 2)$和点$b(1, 0)$之间的电势差$U_{ab}=\nl$.
    \item 如图所示\ref{fig:52}, 半径为$R$的均匀带电球面, 总电荷为$Q$, 设无穷远处的电势为零, 则球内距离球心为$r$的$P$点处的电势$U=\nl$. 
    \begin{figure}[H]
        \centering
        \includegraphics[width=0.15\textheight]{fig52}
        \caption{如图.}\label{fig:52}
    \end{figure}
    \item 一“无限长”均匀带电直线沿$Z$轴放置, 线外某区域的电势表达式为$U=A\mathrm{ln}(x2+y2)$, 式中$A$为常数, 该区域电场强度的两个分量为: $E_x$=\nl, $E_y$=\nl.
\end{enumerate}
\subsection*{二、选择题}
\begin{enumerate}
    \item 如图所示 \ref{fig:47}, 一电偶极子, 正点电荷在坐标$(a,0)$处, 负点电荷在坐标$(-a,0)$处, $P$点是$x$轴上的一点, 坐标为$(x,0)$. 当$x>>a$时, 该点场强的大小为(\hspace{1pc})
    \fourch{$\frac{q}{4\pi\varepsilon_0x}$;}{$\frac{q}{4\pi\varepsilon_0x^2}$;}{$\frac{qa}{2\pi\varepsilon_0x^3}$}{$\frac{qa}{\pi\varepsilon_0x^3}$.}
    \begin{figure}[H]
        \centering
        \includegraphics[width=0.15\textheight]{fig47}
        \caption{如图.}\label{fig:47}
    \end{figure}
    \item 真空中面积为$S$, 间距$d$的两平行板$S>>d_2$, 均匀带等量异号电荷+q和-q, 忽略边缘效应, 则两板间相互作用力的大小是(\hspace{1pc}).
    \twoch{$\frac{q^2}{(4\pi\varepsilon_0d^2)}$; }{$\frac{q^2}{(\varepsilon_0 s)}$;}{$\frac{q^2}{(2\varepsilon_0s)}$;}{$\frac{q^2}{(2\pi\varepsilon_0d^2)}$.}
    \item  下列哪一种说法正确 (\hspace{1pc})
    \onech{电场线上任意一点的切线方向, 代表这点的电场强度的方向;}{在某一点电荷附近的任一点, 若没放试验电荷, 则这点的电场强度为零;}{若把质量为$m$的点电荷$q$放在一电场中, 由静止状态释放, 电荷一定沿电场线运动;}{电荷在电场中某点受到的电场力很大, 该点的电场强度一定很大.}
    \item 关于高斯定理的理解有下面几种说法, 其中正确的是(\hspace{1pc})
    \onech{如果高斯面上$\vec{E}$处处为零, 则该面内必无电荷;}{如果高斯面内无电荷, 则高斯面上$\vec{E}$处处为零;}{如果高斯面上$\vec{e}$处处不为零, 则高斯面内必有电荷;}{如果高斯面内有净电荷, 则通过高斯面的电场强度通量必不为零.}
    \item 下述带电体系的场强分布可能用高斯定理来计算的是(\hspace{1pc})
    \twoch{均匀带电圆板}{有限长均匀带电棒}{电偶极子}{带电介质球(电荷体密度是离球心距离$r$的函数)}
    \item 已知某电场的电场线分布情况如图所示\ref{fig:53}. 现观察到一负电荷从$M$点移到$N$点. 有人根据这个图作出下列几点结论, 其中正确的是(\hspace{1pc})
    \twoch{电场强度$E_M$<$E_N$}{电势$U_M$<$U_N$}{电势能$W_M$<$W_N$}{电场力的功$A>0$}
    \begin{figure}[H]
        \centering
        \includegraphics[width=0.15\textheight]{fig53}
        \caption{如图.}\label{fig:53}
    \end{figure}
    \item 如图所示\ref{fig:54}, 下面表述中正确的是(\hspace{1pc})
    \twoch{$E_A$>$E_B$>$E_C$, $U_A$>$U_B$>$U_C$}{$E_A$<$E_B$<$E_C$, $U_A$>$U_B$>$U_C$}{$E_A$>$E_B$>$E_C$, $U_A$<$U_B$<$U_C$}{$E_A$<$E_B$<$E_C$, $U_A$<$U_B$<$U_C$}
    \begin{figure}[H]
        \centering
        \includegraphics[width=0.15\textheight]{fig54}
        \caption{如图.}\label{fig:54}
    \end{figure}
    \item 下列关于静电场的说法中, 正确的是(\hspace{1pc})
    \onech{电势高的地方场强就大}{带正电的物体电势一定是正的}{场强为零的地方电势一定为零}{电场线与等势面一定处处正交}
\end{enumerate}
\subsection*{三、计算题}
\begin{enumerate}
    \item 如图 \ref{fig:48}, 内半径为$R_1$, 外半径为$R_2$的环形薄板均匀带电, 电荷面密度为$\sigma$, 求: 中垂线上任一$P$点的场强及环心处0点的场强.
    \begin{figure}[H]
        \centering
        \includegraphics[width=0.15\textheight]{fig48}
        \caption{如图.}\label{fig:48}
    \end{figure}
    \item 如图 \ref{fig:51}, 无限长均匀带电圆柱体,电荷体密度为$\rho$, 半径为$R$, 求柱体内外的场强分布.
    \begin{figure}[H]
        \centering
        \includegraphics[width=0.15\textheight]{fig51}
        \caption{如图.}\label{fig:51}
    \end{figure}

    \item 如图 \ref{fig:55}, 球壳的内半径为$a$, 外半径为$b$, 壳体内均匀带电, 电荷体密度为$\rho$, 求: 空间的场强和电势分布.
    \begin{figure}[H]
        \centering
        \includegraphics[width=0.15\textheight]{fig55}
        \caption{如图.}\label{fig:55}
    \end{figure}
\end{enumerate}
    \chapter{场强、电势习题课}
\section{作业习题}
\subsection*{一、计算题}
\begin{enumerate}
    \item 如图所示 \ref{fig:56}, 有一长$l$的带电细杆.
    \begin{enumerate}
        \item[(1)] 电荷均匀分布, 线密度为$+\lambda$, 则杆上距原点$x$处的线元$\mathrm{d}x$对$P$点的点电荷$q_0$的电场力为何? $q_0$受的总电场力为何?
        \item[(2)] 若$\lambda=\varepsilon_0 l$(正电荷), $a=4l$, 则$P$点的电场强度是多少? (如图所示选择坐标系).
        
    \end{enumerate}
    \begin{figure}[H]
        \centering
        \includegraphics[width=0.15\textheight]{fig56}
        \caption{如图.}\label{fig:56}
    \end{figure}
    \item 一半径为$R$的“无限长”圆柱形带电体, 其电荷体密度为$\rho=Ar(r<R)$, 式中$A$为常数, 试求:
    \begin{enumerate}
        \item 圆柱体内, 外各点场强大小分布;
        \item 选距离轴线的距离为$R_0(R_0>R)$处为电势零点, 计算圆柱体内, 外各点的电势分布.
    \end{enumerate}
    \item 如图\ref{fig:57},半径为$R$的圆弧形细塑料棒, 两端空隙对中心张角为$2\theta_0$, 线电荷密度为$\lambda$的正电荷均匀地分布在棒上. 求:
    \begin{enumerate}
        \item 用连续带电体场强叠加原理计算圆心$O$处场强的大小和方向.
        \item 圆心处的电势.
    \end{enumerate}
    \begin{figure}[H]
        \centering
        \includegraphics[width=0.15\textheight]{fig57}
        \caption{如图.}\label{fig:57}
    \end{figure}
    \item  一环形薄片由细绳悬吊着, 环的外半径为$R$, 内半径为$R/2$, 并有电荷$Q$均匀分布在环面上. 细绳长$3R$, 也有电荷$Q$均匀分布在绳上, 如图所示\ref{fig:58}. 试求圆环中心$O$处的电场强度(圆环中心$O$在细绳延长线上)
    \begin{figure}[H]
        \centering
        \includegraphics[width=0.15\textheight]{fig58}
        \caption{如图.}\label{fig:58}
    \end{figure}
\end{enumerate}
    \chapter{导体电介质}
\section{作业习题}
\subsection*{一、填空}
\begin{enumerate}
    \item 选无穷远处为电势零点,半径为$R$的导体球带电后,其电势为$V_0$,则球外离球心距离为$r$~($r>R$)~处的电场强度的大小为\nl,球内离球心距离$r(r<R)$处的电势$U=\nl$.
    \item 半径为$R$的金属球与地连接,在与球心$O$相距$d$处有一电荷为$q$的点电荷,如图所示。设地的电势为零,则球上的感应电荷在球心$O$点处产生的电势$U_0$=\nl.    
    \item 一孤立带电导体球,其表面处场强的方向垂直于导体表面,当把另一带电体放在这个导体球附近时,该导体球表面处场强的方向\nl.
    \item 在两板间距为$d$的平行板电容器中,平行地插入一块厚度为$d/2$的金属大平板,则电容变为原来的\nl 倍;如果插入的是厚度为$d/2$的相对电容率为$\varepsilon_r = 4$的大介质平板,则电容变为原来的\nl 倍.
    \item 一平行板电容器两极板间电压为$U$,其间充满相对电容率为$\varepsilon_r$的各向同性均匀电介质,电介质厚度为$d$。则电介质中的电场能量密度$W_e$=\nl.
    \item 一平行板电容器二极板间充满相对介电常数$\epsilon_r$的电介质,则电容变为原来的\underline{\makebox[3em]{}}倍.
    \item 真空中两块互相平行的无限大均匀带电平面。其电荷密度分别为$+\sigma$和$+2\sigma$,两板之间的距离为$d$,两板间的电场强度大小$E=\nl$.
    \item 电荷$q_1$、$q_2$、$q_3$和$q_4$在真空中的分布如图所示, 
    其中~$q_2$~是半径为R的均匀带电球体, $S$为闭合曲面,
    则通过闭合曲面$S$的电通量$\displaystyle{\oiint_S {\vec{E}\cdot \mathrm{d}\vec{S}}=\nl}$.
    
\end{enumerate}
\subsection*{二、选择}
\begin{enumerate}
    \item  在一个孤立的导体球壳内,若在偏离球中心处放一个点电荷,则在球壳内、外表面上将出现感应电荷,其分布将是~\space
    \twoch{内表面均匀,外表面也均匀.}{内表面不均匀,外表面均匀.}{内表面均匀,外表面不均匀.}{内表面不均匀,外表面也不均匀.}
    \item 对于处在静电平衡下的导体,下面的叙述中,正确的是~\space
    \onech{导体内部无净电荷,电荷只能分布在导体表面}{导体表面上各处的面电荷密度与表面紧邻处的电场强度的大小成反比}{孤立的导体处于静电平衡时,表面各处的面电荷密度与各处表面的曲率成反比}{沿导体表面移动点电荷,静电场力做功}
\end{enumerate}
    \chapter{稳定磁场}
\section{作业习题}
\subsection*{一、填空题}
\begin{enumerate}
    \item 如图~\ref{fig:65},一个带电$Q$的粒子($Q>0$),以速度$V$向右运动,求距粒子$r$处产生的磁感应强度为$B=\nl$,方向为\nl.
    \begin{figure}[H]
        \centering
        \includegraphics[width=0.25\textwidth]{fig65}
        \caption{如图所示}\label{fig:65}
    \end{figure}
    \item 一个无限长的圆形螺线管,单位长度匝数为$n$,通有电流强度$I$,则在螺线管内部的磁感应强度大小$B=\nl$.
    \item 如图~\ref{fig:66},一平面线圈由半径为~$R$~的1/4圆弧和相互垂直的二直线组成,通以电流$I$,把它放在磁感强度为~$B$~的均匀磁场中,则线圈磁矩大小$p_m=\nl$.
    \begin{figure}[H]
        \centering
        \includegraphics[width=0.25\textwidth]{fig66}
        \caption{如图所示}\label{fig:66}
    \end{figure}
    \item 如图~\ref{fig:67}, 真空中环绕两根通有电流为$I$的导线的两种环路,则对环路$L_1$有$\oint_{l1}\vec{B}\cdot \mathrm{d}\vec{l}=\nl$,对环路$L_2$有$\oint_{l2}\vec{B}\cdot \mathrm{d}\vec{l}=\nl$.
    \begin{figure}[H]
        \centering
        \includegraphics[width=0.25\textwidth]{fig67}
        \caption{如图所示}\label{fig:67}
    \end{figure}
    \item 在安培环路定理$\oint_{l} \vec{B}\cdot \mathrm{d}\vec{l}$中, $\sum \mathbf{I_i}$是指\nl;$\vec{B}$~是指\nl;它是由\nl 决定的.
    \item 一半径为$a$的无限长直载流导线,沿轴向均匀地流有电流$I$,若作一个半径为$R=5a$,高为$L$的柱形曲面,已知柱形曲面的轴与载流导线的轴平行且相距$3a$,如图~\ref{fig:68},则$B$在圆柱侧面$S$上的积分~$\int_S \vec{B}\cdot \mathrm{d} \vec{S}$.
    \begin{figure}[H]
        \centering
        \includegraphics[width=0.25\textwidth]{fig68}
        \caption{如图所示}\label{fig:68}
    \end{figure}
    \item 通有电流$I$ 的长直导线在一平面内被弯成如图~\ref{fig:69}~形状($R$为已知),放于垂直进入纸面的均匀磁场$\vec{B}$中,则整个导线所受的安培力大小$F=\nl$.
     \begin{figure}[H]
        \centering
        \includegraphics[width=0.25\textwidth]{fig69}
        \caption{如图所示}\label{fig:69}
    \end{figure} 
    \item 如图所示~\ref{fig:70},平行放置在同一平面内的三条载流长直导线,电流依次是$I$,$I$,$2I$要使导线$AB$所受的安培力等于零,则
    $x=\nl$.
    \begin{figure}[H]
        \centering
        \includegraphics[width=0.25\textwidth]{fig70}
        \caption{如图所示}\label{fig:70}
    \end{figure}
\end{enumerate}
\subsection*{二、选择题}
\begin{enumerate}
    \item 长直导线通有电流$I$,将其弯成如图所示\ref{fig:71}形状,则$O$点处的磁感应强度大小为~\spaces
    \twoch{$\frac{\mu_0I}{2\pi R}+\frac{\mu_0 I}{4R}$}{$\frac{\mu_0 I}{4\pi R}+\frac{\mu_0 I}{8R}$}{$\frac{\mu_0 I}{2\pi R}+\frac{\mu_0 I}{8 R}$}{$\frac{\mu_0 I}{4\pi R}+\frac{\mu_0 I}{4 R}$}
    \insertfig{0.25}{fig71}{fig:71}
    \item 一根载有电流$I$的无限长直导线,在$A$处弯成半径为$R$的圆形,由于导线外有绝缘层,在$A$处两导线并不短路,则在圆心处磁感应强度$\vec{B}$的大小为~\spaces
    \twoch{$I(\mu_0 +1 )/(2\pi R)$}{$\mu_0 \pi I / (2\pi R)$}{$\mu I (1+\pi R)$}{$\mu_0 I (1+\pi)/(4\pi R)$}

    \item 四条相互平行的载流长直导线电流强度均为$I$,方向如图\ref{fig:73}。设正方形的边长为$2a$,则正方形中心的磁感应强度为~\spaces
    \fourch{$\frac{2\mu_0}{\pi a}I$}{$\frac{2\mu_0}{\sqrt{2}\pi a}I$}{$\frac{\mu_0}{\pi a}I$}{$0$}
    \insertfig{0.25}{fig72}{fig:72}
    \item 如图\ref{fig:73}, 在磁感强度为$\vec{B}$的均匀磁场中作一半径为$r$的半球面$S$,$S$边线所在平面的法线方向单位矢量$\vec{n}$与$\vec{B}$的夹角为$\alpha$,则通过半球面$S$的磁通量(取弯面向外为正)为~\spaces
    \fourch{$\pi r^2 B$}{$2\pi r^2 B$}{$-\pi r^2 B \mathrm{sin} \alpha$}{$-\pi r^2 B\mathrm{cos}\alpha$}
    \insertfig{0.25}{fig73}{fig:73}
    \item 如图\ref{fig:74},两根直导线~$ab$~和~$cd$~沿半径方向被接到一个截面处处相等的铁环上,稳恒电流$I$从$a$端流入而从$d$端流出,则磁感强度$\vec{B}$沿图中包围铁环截面的闭合路径$L$的积分$\oint_L \vec{B}\cdot \dd \vec{l}$等于~\spaces
    \fourch{$\mu_0I$}{$-\mu_0 I/3$}{$2\mu_0I/3$}{$-2\mu_0 I/3$}
    \insertfig{0.25}{fig74}{fig:74}
    \item 如图所示\ref{fig:75},流出纸面的电流为$2I$,流进纸面的电流为$I$,则下述式中哪一个是正确的是~\spaces
    \twoch{$\oint_{L1}\vec{B}\cdot \dd \vec{l}=2\mu_0 I$}{$\oint_{L2}\vec{B}\cdot \dd \vec{l}=\mu_0 I$}{$\oint_{L3}\vec{B}\cdot \dd \vec{l}=-\mu_0 I$}{$\oint_{L4}\vec{B}\cdot \dd \vec{l}=-\mu_0 I$}
    \insertfig{0.25}{fig75}{fig:75}
    \item 如图~\ref{fig:76}~六根互相绝缘导线,通以电流强度均为$I$,区域$I$、$II$、$III$、$IV$均为正方形,那么指向纸内的磁通量最大的区域是~\spaces
    \fourch{$I$区域}{$II$区域}{$III$区域}{$IV$区域}
    \insertfig{0.25}{fig76}{fig:76}
    \item 无限长直导线通有电流$I$,右侧有两个相连的矩形回路,分别是$S_1$和$S_2$,则通过两个矩形回路$S_1$、$S_2$的磁通量之比为~\spaces
    \fourch{1:2}{1:1}{1:4}{2:1}
    \insertfig{0.25}{fig77}{fig:77}
\end{enumerate}
\subsection*{三、计算题}
\begin{enumerate}
    \item 载有电流为$I$的无限长导线,弯成如图~\ref{fig:78}~形状,其中一段是半径为$a$的半圆,求圆心处的磁感应强度$\vec{B}$的大小.
    \insertfig{0.25}{fig78}{fig:78}
    \item 如图~\ref{fig:79}~半径为$R$的带电圆盘,电荷面密度为$\sigma$,圆盘以角速度$\omega$绕过盘心并垂直盘面的轴旋转,求中心$O$处的磁感应强度$\vec{B}$。
    \insertfig{0.25}{fig79}{fig:79}
    \item 如图~\ref{fig:80}一半径为$R$的无限长圆柱形导体,现有电流$I$均匀地流过导体横截面,且电流方向与导体轴线平行,求空间的磁场分布。
    \insertfig{0.25}{fig80}{fig:80}
    \item 如图~\ref{fig:81}~在载流为$I_1$的长直导线旁,共面放置一载流为$I_2$的等腰直角三角形线圈$abc$,腰长$ab=ac=L$,边长$ab$平行于长直导线,相距$L$,求线圈各边受的磁力$F$.
    \insertfig{0.25}{fig81}{fig:81}
    \item 如图~\ref{fig:82}无限长直导线和半径为$R$的圆形线圈,彼此绝缘,共面放置,且圆线圈直径和直导线重合,直导线与圆线圈分别通以电流$I_1$和$I_2$,求
    \begin{enumerate}[label=(\arabic*)]
        \item 长直导线对半圆弧$abc$所作用的磁力;
        \item 整个圆形线圈所受的磁力.
    \end{enumerate}
    \insertfig{0.25}{fig82}{fig:82}
\end{enumerate}
    \input{chapter/chap12.tex}
    \hypersetup{pageanchor=true}
\end{document}