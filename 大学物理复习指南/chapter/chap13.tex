\chapter{电磁感应}
\section{作业习题}
\subsection*{一、填空题}
\begin{enumerate}
    \item 如果导体不是闭合的,即使导体在磁场里做切割磁力线运动也不会产生感应电流,但在导体的两端产生\nl.
    \item 在磁感强度为~$\vec{B}$~的均匀磁场中,以速率$v$垂直切割磁力线运动的一长度为$L$的金属杆,相当于一个电源,它的电动势~$\varepsilon=\nl$.
    \item 半径为$R$的圆形回路,放在均匀磁场中,回路平面与$\vec{B}$垂直,当回路半径以恒定的速率$\frac{\dd R{\dd t}$收缩,刚开始时回路中的感应电动势大小$\varepsilon=\nl$.
    \item 动生电动势计算公式为~$\varepsilon=\nl$.
    \item 如图:长为$L$的金属杆$OA$,在方向竖直向上,磁感应强度大小为$B$的均匀磁场中,以匀角速度$\omega$绕$OO'$轴逆时针(从上往下看)方向转动,转动过程中$OA$与$OO'$轴的夹角始终保持$60^\circ$不变,则$OA$上的动生电动势为\nl.
    \item 半径为~$a$~的无限长密绕螺线管,单位长度上的匝数为$n$,通以交变电流$I=I_0\mathrm{cos}\omega t$,则围在管外的同轴圆形回路(半径为~$r$~)上的感生电动势大小$\varepsilon=\nl$.
    \item 金属杆$ABC$处于磁感强度$B=0.1 T$的匀强磁场中,磁场方向垂直纸面向里(如图所示)。已知$AB=BC=0.2 \mathrm{m}$,当金属杆在图中标明的速度方向运动时,测得$A,C$两点间的电势差是~$3.0V$,则可知$A,B$两点间的电势差$V_{ab}=\nl$.
    \item 对于一自感为$L$的线圈来说,当其通电流为$I$时,磁场的能量$W=\nl$.
    \item 反映电磁场基本性质和规律的积分形式的麦克斯韦方程组为:
    \twoch{$\oint_S \vec{D}\cdot \dd \vec{S}=\sum\limits_{i=1}^n q_i$}{$\int_l \vec{E}\cdot \dd \vec{l}=-\frac{\dd \phi}{\dd t}$}{$\oint_S \vec{B}\cdot \dd \vec{S}=0$}{$\oint_l \vec{H}\cdot \vec{l}=\sum\limits_{i=1}^{n}I_I+\frac{\dd \phi_e}{\dd t}$}
    试判断:变化的磁场一定伴随有电场,这个结论是包含于或等效于上述第\nl个麦克斯韦方程式。磁感应线是无头无尾的,这个结论是包含于或等效于上述第\nl个麦克斯韦方程式。
    \item 麦克斯韦电磁场理论的两个基本假设为:(1)\underline{\makebox[2.5cm]{}};  (2)\underline{\makebox[2.5cm]{}};
\end{enumerate}