\chapter{导体电介质}
\section{作业习题}
\subsection*{一、填空}
\begin{enumerate}
    \item 选无穷远处为电势零点,半径为$R$的导体球带电后,其电势为$V_0$,则球外离球心距离为$r$~($r>R$)~处的电场强度的大小为\nl,球内离球心距离$r(r<R)$处的电势$U=\nl$.
    \item 半径为$R$的金属球与地连接,在与球心$O$相距$d$处有一电荷为$q$的点电荷,如图所示。设地的电势为零,则球上的感应电荷在球心$O$点处产生的电势$U_0$=\nl.    
    \item 一孤立带电导体球,其表面处场强的方向垂直于导体表面,当把另一带电体放在这个导体球附近时,该导体球表面处场强的方向\nl.
    \item 在两板间距为$d$的平行板电容器中,平行地插入一块厚度为$d/2$的金属大平板,则电容变为原来的\nl 倍;如果插入的是厚度为$d/2$的相对电容率为$\varepsilon_r = 4$的大介质平板,则电容变为原来的\nl 倍.
    \item 一平行板电容器两极板间电压为$U$,其间充满相对电容率为$\varepsilon_r$的各向同性均匀电介质,电介质厚度为$d$。则电介质中的电场能量密度$W_e$=\nl.
    \item 一平行板电容器二极板间充满相对介电常数$\epsilon_r$的电介质,则电容变为原来的\underline{\makebox[3em]{}}倍.
    \item 真空中两块互相平行的无限大均匀带电平面。其电荷密度分别为$+\sigma$和$+2\sigma$,两板之间的距离为$d$,两板间的电场强度大小$E=\nl$.
    \item 电荷$q_1$、$q_2$、$q_3$和$q_4$在真空中的分布如图所示, 
    其中~$q_2$~是半径为R的均匀带电球体, $S$为闭合曲面,
    则通过闭合曲面$S$的电通量$\displaystyle{\oiint_S {\vec{E}\cdot \mathrm{d}\vec{S}}=\nl}$.
    
\end{enumerate}
\subsection*{二、选择}
\begin{enumerate}
    \item  在一个孤立的导体球壳内,若在偏离球中心处放一个点电荷,则在球壳内、外表面上将出现感应电荷,其分布将是~\space
    \twoch{内表面均匀,外表面也均匀.}{内表面不均匀,外表面均匀.}{内表面均匀,外表面不均匀.}{内表面不均匀,外表面也不均匀.}
    \item 对于处在静电平衡下的导体,下面的叙述中,正确的是~\space
    \onech{导体内部无净电荷,电荷只能分布在导体表面}{导体表面上各处的面电荷密度与表面紧邻处的电场强度的大小成反比}{孤立的导体处于静电平衡时,表面各处的面电荷密度与各处表面的曲率成反比}{沿导体表面移动点电荷,静电场力做功}
\end{enumerate}